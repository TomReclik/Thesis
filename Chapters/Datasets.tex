\chapter{Datasets}

\section{Data Collection - First Stage}
At first damage sites in the recorded SEM micrographs were classified using the conventional class definitions for damage mechanisms in dual phase steels, as explained in \ref{cha:DualPhaseSteels}. Additionally one class was added, namely evolved damage sites. These sites have evolved into the surrounding ferrite, and the information of the underlying mechanism is not unique. \\

Two researchers were involved in the creation of a dataset using labelImg \cite{labelImg}. This tool enables its user to zoom into images and create rectangular bounding boxes around objects in the image that need to be classified. Then a label was added to the marked object. Additionally the tool gives its user the option to mark objects as "difficult". This option was additionally used for damage sites, that did not evolve into the surrounding ferrite but were not clearly assignable to a certain class. These objects can be emmitted from the training of a network, as the need for represantative samples were needed for the training of a classifier. A labeled micrograph using labelImg can be seen in figure \ref{fig:labelImg}. \\

\begin{figure}
\centering
\includegraphics[width=\textwidth]{labelImg.png}
\caption{Damage site classification using labelImg.}
\label{fig:labelImg}
\end{figure}

In order to have a consistently labeled dataset, the researcher further more would correspond if instances appeared that were difficult to assign a class. If a consensus was reached the class agreed upon was chosen as the class for this particular instance, otherwise the "difficult" label was added. During labeling difficulties arose for the assignment of boundary decohesions and evolved damage sites. Therefore an internal study was performed assessing whether the class definitions were clear enough. 

\section{Consistency of Class Definitions}

The participants of the study received the following class definitions together with $25$ instances belong to either of the classes.\\
\begin{itemize}[label={}]
\item "\textbf{Inclusion}: Inclusions, either holes left from preparation or actual inclusions."
\item "\textbf{Martensite Cracking}: Brittle cracked martensite islands."
\item "\textbf{Interface Decohesion}: Damage to the martensite/ferrite boundary."
\item "\textbf{Boundary Decohesion}: Damage to ferrite grain boundaries."
\item "\textbf{Evolved Damage}: More than one active damage mechanism, e.g. martensite cracking evolved into ductile damage in" ferrite.
\end{itemize}

\subsection{Martensite Cracking}

\begin{figure}
\centering
\includegraphics[width=\textwidth]{/Classification/Martensite.pdf}
\end{figure}

\begin{figure}[H]
\begin{subfigure}{.25\textwidth}
\centering
  \includegraphics[width=.8\linewidth]{/Classification/Martensite0.pdf}
  \caption{MC}
  \label{fig:MC}
\end{subfigure}%
\begin{subfigure}{.25\textwidth}
\centering
  \includegraphics[width=.8\linewidth]{/Classification/Martensite1.pdf}
  \caption{ID}
  \label{fig:Interface_scalebar}
\end{subfigure}%
\centering
\begin{subfigure}{.25\textwidth}
\centering
  \includegraphics[width=.8\linewidth]{/Classification/Martensite2.pdf}
  \caption{NE}
  \label{fig:Notch_scalebar}
\end{subfigure}%
\begin{subfigure}{.25\textwidth}
\centering
  \includegraphics[width=.8\linewidth]{/Classification/Martensite3.pdf}
  \caption{Inclusion}
  \label{fig:Inclusion_scalebar}
\end{subfigure}%
\caption{Abstracted damage mechanism for martensite cracking (MC) in subfigure (a), interface decohesion (ID) in subfigure (b), notch effect (NE) in subfigure (c),  and inclusions in subfigure (d). The martensite islands, the ferrite matrix, and the foreign body are labeled as M, F, and I respectively.}
\label{fig:classes}
\end{figure}

\begin{figure}
\centering
\includegraphics[width=\textwidth]{/Classification/Inclusion.pdf}

\end{figure}

Investigating the agreement with the predetermined classes for each sample is shown in table \ref{tab:Reliability}. As can be seen similar difficulties were found for the classes boundary decohesion and evolved damages. Furthermore it was found that due to the high martensite concentration, ferrite grains rarely if at all were directly adjacent to each other. Rather ferrite grains were seperated by martensite grains. Indicating that damage sites marked as boundary decohesions result from different mechanisms. \\

Additionally it was found that damage sites appeared often between the tips of two martensite islands. Therefore a new class was introduced, termed notch effect. This category is classified by its location in the microstructure exhibiting a special geometry. Due to the hardness of the martensite at this point in the geometry a stress concentration is present, resulting the formation of a damage site. In contrast to the other classes the underlying physical process, leading to the formation of a void, is not determined. It is possible that the two martensite islands were connected, prior to the crack occuring, or that a ferrite bridge existed between two separate martensite islands and an interface decohesion took place. \\

The new classes agreed upon are shown diagrammatically in figure \ref{fig:classes}. Some of the already labeled damage sites were therefore relabeled using the new class definitions. 

\begin{table}[H]
 \begin{center}
  \begin{tabular}{@{} *2l @{}} \toprule[2pt]
   Damage Category & Accuracy \\\midrule
   Martensite Cracking & $85 \%$   \\ 
   Inclusion  & $84 \%$ \\ 
   Interface Decohesion  & $76 \% $ \\
   Evolved & $56\%$ \\
   Boundary Decohesion & $44 \%$ \\ \bottomrule[2pt]

  \end{tabular}
 \end{center}
 \caption{Agreement for the classification of damage sites by hand. }
 \label{tab:Reliability}
\end{table}

%
%During labeling, some difficulties arose. Mainly the classes boundary decohesion and evolved damage site posed particularly difficult. An internal study was therefore performed in order to assert whether the class definitions were narrow enough. The study showed that other researchers also had difficulties in classifying damage sites predetermined to belong to these classes. 
%The agreement with the damage classes determined beforehand can be seen in table \ref{tab:Reliability}. Major problems arose with boundary decohesions and evolved damage sites. Due to the high concentration of martensite in the dual-phase steel used, few ferrite grains are adjacent to each other. Therefore boundary decohesions will occur rarely, and sites resembling boundary decohesions probably result from other mechanisms. Problems with evolved damage sites come from the lack of a clear threshold after which a damage site falls into this class. Due to this, boundary decohesions and evolved damages were not used for the classification algorithm to distinguish between.\\
%
%\begin{table}[H]
% \begin{center}
%  \begin{tabular}{@{} *2l @{}} \toprule[2pt]
%   Damage Category & Accuracy \\\midrule
%   Martensite Cracking & $85 \%$   \\ 
%   Inclusion  & $84 \%$ \\ 
%   Interface Decohesion  & $76 \% $ \\
%   Evolved & $56\%$ \\
%   Boundary Decohesion & $44 \%$ \\ \bottomrule[2pt]
%
%  \end{tabular}
% \end{center}
% \caption{Agreement for the classification of damage sites by hand. }
% \label{tab:Reliability}
%\end{table}

\section{Data Collection - Second Stage}

Since the information of the location of the damage site inside of the micrograph is not necessary for the training of a classifier, a new method was used for the classification of damage sites. The goal is to eliminate the need to search for damage sites by hand. This was performed by using DBSCAN, as explained in chapter \ref{cha:CNN}. 




undersampling, oversampling, additional weights



A consistently labeled dataset is crucial for the training of a classifier. When elements in the training set are classified using different class definitions or are mislabeled, the convergence and the performance of a network used for classification is hindered. \\

In a first approximation the effect of using different class definitions for labeling or misclassification can be explained as follows. Assuming the network has already been trained to some extent on a correctly labeled dataset, two similar samples, both belonging to class $c_0$ but one of them incorrectly labeled as class $c_1$, will be mapped by the feature extraction to close proximity of each other in the feature space. The fully connected layer in the end of the CNN will then have to learn to map these close points in the feature space to the maximal distance in the output space, requiring large weight corrections, therefore slowing down training. Furthermore, if the two points in the feature space are arbitrarily small to each other the effect of a mislabeled training sample will then nullify the adaptation of the network to a correctly labeled sample. This mapping problem can be seen in figure \ref{fig:featuremapping}.

\begin{figure}[H]
\centering
\includegraphics[width=\textwidth]{featuremapping.pdf}
\caption{The classification of two similar images in the input space to two different classes. Given a feature mapping this will on the one hand lead to a close distance in the feature space, while on the other hand to a desired maximal distance in the output space. The possible outputs are indicated by a dashed line.}
\label{fig:featuremapping}
\end{figure}

%Due to the necessary time for the labeling of damage sites, this task is performed by multiple researchers. Unclear class definitions will lead to different conclusions for similar damage sites, leading the neural network to adapt its weights once in order to learn the mapping $o_0 \mapsto c_0$ and for a similar object $o_1 \mapsto c_1$, while the true class may be either $c_0$ or $c_1$. This will nullify the correction of the networks weights. \\

\section{Reliability of Training Data}

Due to the need of consistently labeled datasets, for the training process of the classifier, a study was performed internally beforehand. Five experts were asked to classify $25$ damage sites given the following descriptions. 
\begin{itemize}[label={}]
\item \textbf{Inclusion}: Inclusions, either holes left from preparation or actual inclusions.
\item \textbf{Martensite Cracking}: Brittle cracked martensite islands.
\item \textbf{Interface Decohesion}: Damage to the martensite/ferrite boundary.
\item \textbf{Boundary Decohesion}: Damage to ferrite grain boundaries.
\item \textbf{Evolved Damage}: More than one active damage mechanism, e.g. martensite cracking evolved into ductile damage in ferrite.
\end{itemize}
The agreement with the damage classes determined beforehand can be seen in table \ref{tab:Reliability}. Major problems arose with boundary decohesions and evolved damage sites. Due to the high concentration of martensite in the dual-phase steel used, few ferrite grains are adjacent to each other. Therefore boundary decohesions will occur rarely, and sites resembling boundary decohesions probably result from other mechanisms. Problems with evolved damage sites come from the lack of a clear threshold after which a damage site falls into this class. Due to this, boundary decohesions and evolved damages were not used for the classification algorithm to distinguish between.\\

\begin{table}[H]
 \begin{center}
  \begin{tabular}{@{} *2l @{}} \toprule[2pt]
   Damage Category & Accuracy \\\midrule
   Martensite Cracking & $85 \%$   \\ 
   Inclusion  & $84 \%$ \\ 
   Interface Decohesion  & $76 \% $ \\
   Evolved & $56\%$ \\
   Boundary Decohesion & $44 \%$ \\ \bottomrule[2pt]

  \end{tabular}
 \end{center}
 \caption{Agreement for the classification of damage sites by hand. }
 \label{tab:Reliability}
\end{table}

Furthermore damage sites located in a special geometrical constellation appeared often. These are classified by being located between the tips of two martensite islands. The underlying mechanisms of the nucleation of these damage sites is unknown, and might result from the cracking of one formerly connected martensite or the decohesion of a ferrite bridge between two martensite islands. Characteristic for this constellation is a stress concentration at this particular point in the microstructure, this class is called notch effect. The final classes used for classification in this work are shown in figure \ref{fig:classes}.

\begin{figure}[H]
\begin{subfigure}{.25\textwidth}
\centering
  \includegraphics[width=.8\linewidth]{MartensiteCracking_abstraction_scalebar.pdf}
  \caption{MC}
  \label{fig:MC}
\end{subfigure}%
\begin{subfigure}{.25\textwidth}
\centering
  \includegraphics[width=.8\linewidth]{InterfaceDecohesion_abstraction_scalebar.pdf}
  \caption{ID}
  \label{fig:Interface_scalebar}
\end{subfigure}%
\centering
\begin{subfigure}{.25\textwidth}
\centering
  \includegraphics[width=.8\linewidth]{NotchEffect_abstraction_scalebar.pdf}
  \caption{NE}
  \label{fig:Notch_scalebar}
\end{subfigure}%
\begin{subfigure}{.25\textwidth}
\centering
  \includegraphics[width=.8\linewidth]{Inclusion_abstraction_scalebar.pdf}
  \caption{Inclusion}
  \label{fig:Inclusion_scalebar}
\end{subfigure}%
\caption{Abstracted damage mechanism for martensite cracking (MC) in subfigure (a), interface decohesion (ID) in subfigure (b), notch effect (NE) in subfigure (c),  and inclusions in subfigure (d). The martensite islands, the ferrite matrix, and the foreign body are labeled as M, F, and I respectively.}
\label{fig:classes}
\end{figure}



\section{Influence of an Incorrectly Labeled Dataset}

The influence of an incorrectly labeled dataset was studied using the CIFAR-10 dataset, with the EERACN network, see \ref{app:EERACN}. The network was trained for $30$ epochs with an Adam optimizer \cite{Sharma2017} with a learning rate of $0.0005$ and other parameters set to default. The baseline accuracy for a correctly labeled dataset with constant seeds among experiments and fluctuations resulting from the use of GPU is $0.73\pm 0.01$. Purposely mislabeling $30 \%$, corresponding to the average mislabeling rate from \ref{tab:Reliability}, of the training data leads to an accuracy of $0.66\pm 0.01$ evaluated on a dataset with correct labels, leading to a drop in the accuracy of $0.07 \pm 0.01$. For a realistic setting the test dataset will also be mislabeled. Evaluating the trained network on a mislabeled dataset leads to a further drop in the accuracy to $0.43\pm 0.01$, since correctly labeled inputs will be evaluated as wrong predictions. The mislabeled data has therefore two effects. Firstly the performance of the network drops significantly. Secondly the measured accuracy of the network will be skewed towards a network using random predictions. 


%The network was once trained with a correctly labeled dataset and then with a dataset with $30\%$ of the datapoints mislabeled, corresponding to the average mislabeling rate from \ref{tab:Reliability}. The baseline accuracy for a correctly labeled dataset with constant seeds among experiments and fluctuations resulting from the use of GPU is $0.73\pm 0.01$. Training the network with mislabeled data and evaluating the trained network on a dataset labeled correctly the accuracy drops to $0.66\pm 0.01$. Because the data used for the training of a network distinguishing between damage mechanisms is completely labeled by hand the test dataset would also be labeled incorrectly. The accuracy of a network evaluated on a test dataset with inconstistent labeling leads to a further drop in the accuracy to $0.43\pm 0.01$, due to correct predictions being labeled as wrong.\\


\section{Datasets}
The datasets were obtained from SEM micrographs by two researchers. 



The datasets were obtained in two ways. At first labelImg \cite{labelImg} was used. With this tool it is possible to mark damage sites in SEM panoramas, requiring the user to search for them by zooming into the micrograph. Later once the localization algorithm was implemented as explained in chapter \ref{cha:Localization}, significantly reducing the required time for the creation of a dataset by negating the need to localize damage sites by hand. \\

\begin{figure}[H]
\begin{subfigure}{.5\textwidth}
\centering
  \includegraphics[width=.8\linewidth]{Artifact_scalebar.png}
\end{subfigure}
\begin{subfigure}{.5\textwidth}
\centering
  \includegraphics[width=.8\linewidth]{Artifact_2_scalebar.png}
\end{subfigure}
\caption{Artifact showing on the surface of the sample.}
\label{fig:artifacts}
\end{figure}

During deformation artifacts can form on the surface of the micrograph. A few examples of such artifacts are shown in figure \ref{fig:artifacts}. Due to the preparation of the sample these do not show in ex-situ experiments and very rarely in early stages of in-situ experiments. The dataset is therefore split into two categories. The first one consists of surface micrographs not containing artifacts on the sample's surface, while the second one consists of surface micrographs containing said artifacts. \\

The available data for each damage mechanism and category after labeling is shown in table \ref{tab:Dataset}.

%The data is split into two categories. The first category are surface micrographs that were taken right after preparing the sample. Due to the preparation these do not contain artifacts from the deformation of the sample. This first category consists of ex-situ experiments and the first stage of in-situ experiments. Later stages of in-situ experiments fall into the second category, 
%The data is split into three parts. Firstly data from ex-situ experiments, with different grades of stress. The dual-phase steel was prepared before each recording with the SEM, therefore no artifacts can be seen on the surface of the sheet. Secondly data from the first stage of in-situ experiments, this data is  The second round of experiments was performed in-situ. Due to the in-situ nature of those experiments at higher rates of deformation, recordings contain artifacts on the surface of the probe. The datasets with the number of damage sites in each category can be seen in table \ref{tab:Dataset}. \\ 

\begin{table}[H]
\begin{center}
\begin{tabular}{@{} *5l @{}} \toprule[2pt]
Training set &  \multicolumn{4}{c}{Damage Mechanism}   \\\midrule
 & Inc & MC & ID & NE   \\ 
ex-situ  & 379 & 691 & 788 & 443\\ 
in-situ  & 193 & 796 & 586 & 418 \\ \bottomrule
all  & 572 & 1487 & 1374 & 861\\\bottomrule[2pt]

\end{tabular}
 \caption{The number of damage sites found in the ex-situ and in-situ experiments per class (inclusion (Inc), martensite cracking (MC), interface decohesion (ID), and notch effect (NE)).}
 \label{tab:Dataset}
\end{center}
\end{table}