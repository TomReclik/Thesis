\chapter{Dual-Phase Steel}

Dual-phase steels are a class of advanced high-strength steels. On a microstructural level dual-phase steels consist of a ductile ferrite matrix with islands of brittle martensite. It furthermore can contain austenite, pearlite, bainite, carbides and acicular ferrites. From an engineering point of view it exhibits multiple advantageous properties besides its rather straightforward thermodynamical processing. Dual-phase steels combine a high ultimate tensile strength, caused by the brittle martensite, a low initial yielding stress, enabled by the ductile ferrite, high fracture strain, and a high hardening ratio. Combined with their low weight they are of particular interest in the automotive industry, where they are used for steel sheets. \\
%Dual-phase steels are a class of advanced high-strength steels. On a microstructural level they consist of a ductile ferrite matrix with islands of brittle martensite. Furthermore they can contain, austenite remaining from the processing of the material, pearlite, bainite carbides, and acicular ferrite. The hard martensite is responsible for the materials high ultimate strength, while the ductile ferrite results in a low initial yield strength. Those advantageous properties together with a high fracture strain, high hardening ratio, and their light weight makes them interesting from a engineering point of view, especially in the automotive industry, where dual-phase steels are used for steel sheets. \\

While the microstructure of dual-phase steels is seemingly simple, its influence on the mechanical properties of dual-phase steels is only partially understood and research is ongoing. This is due to the fact that multiple microstructural parameters, like the martensite volume ratio, the size of martensite islands, the morphology, etc. play an important role for the macroscopic behavior. The microstructure at the surface of a sample is shown in figure \ref{fig:DPMicrostructure}. \\

\begin{figure}
\centering
  \includegraphics[width=\linewidth]{DPSteelMicrostructure.png}
  \caption{Dual-phase steel SEM micrograph}
  \label{fig:DPMicrostructure}
\end{figure}

\section{Damage Mechanisms}

While under tensile stress, the material exhibits irreversible deformation at stresses above the yield point. A stress-strain curve is shown in figure \ref{fig:DPStressStrain}. Beyond this point voids can nucleate in the microstructure. These voids are classified based on their relative position to the constituents of the dual-phase steel. The classes are as follows
\begin{itemize}[label={}]
\item \textbf{Martensite Cracking}: A crack nucleating inside a martensite island.
\item \textbf{Interface Decohesion}: Nucleation of a void at the interface between a martensite island and a ferrite grain.
\item \textbf{Boundary Decohesion}: Nucleation of a void at the interface between two adjacent ferrite grains.
\item \textbf{Notch Effect}: A void nucleating between the tips of two martensite islands. 
\end{itemize}
Additionally influencing the materials behavior are
\begin{itemize}[label={}]
\item \textbf{Inclusions}: A foreign particle remaining from the processing of the material. 
\end{itemize}
The different possible damage mechanisms are shown in figure \ref{fig:DamageCategories_abstraction}. \\
Due to the high concentration of martensite in the dual-phase steel used in this study, boundary decohesions occur rarely or not at all. In in-situ experiments, most of the sites classified at a certain stage, without knowledge of the previous state, as boundary decohesions turned out to be evolved notch effect sites. Examples for each damage category can be seen in figure \ref{fig:DamageCategories}.\\

\begin{figure}
\begin{subfigure}{.2\textwidth}
\centering
  \includegraphics[width=.8\linewidth]{MartensiteCracking_abstraction.pdf}
  \caption{MC}
  \label{fig:MC}
\end{subfigure}%
\begin{subfigure}{.2\textwidth}
\centering
  \includegraphics[width=.8\linewidth]{InterfaceDecohesion_abstraction.pdf}
  \caption{ID}
  \label{fig:Interface_scalebar}
\end{subfigure}%
\centering
\begin{subfigure}{.2\textwidth}
\centering
  \includegraphics[width=.8\linewidth]{BoundaryDecohesion_abstraction.pdf}
  \caption{BD}
  \label{fig:Inclusion_scalebar}
\end{subfigure}%
\begin{subfigure}{.2\textwidth}
\centering
  \includegraphics[width=.8\linewidth]{NotchEffect_abstraction.pdf}
  \caption{NE}
  \label{fig:Notch_scalebar}
\end{subfigure}%
\centering
\begin{subfigure}{.2\textwidth}
\centering
  \includegraphics[width=.8\linewidth]{Inclusion_abstraction.pdf}
  \caption{Inclusion}
  \label{fig:Inclusion_scalebar}
\end{subfigure}%
\caption{Abstracted damage mechanism for martensite cracking (MC) in subfigure (a), interface decohesion (ID) in subfigure (b), boundary decohesion (BD) in subfigure (c), notch effect (NE) in subfigure (d), and inclusions in subfigure (e). The martensite islands, the ferrite matrix, and the foreign body are labeled as M, F, and I respectively.}
\label{fig:DamageCategories_abstraction}
\end{figure}

\section{Experimental Setup}

\textbf{Preparation} \\
Before the experiments were performed, the surface of the dual-phase steel was prepared in order to be able to distinguish the martensite islands from the ferrite matrix, in surface micrographs. The preparation process consists of grinding the surface with up to $4000$ grit sandpaper, polishing it with oxide polishing suspension in steps of $6\mu m$, $3\mu m$, and $1\mu m$, and finally etching it with $1 \%$ nital. \\

\noindent \textbf{Probe Geometry} \\
The probe geometry was chosen in such a way that the center of the probe experiences almost homogeneous stress. It can be seen in figure \ref{fig:ProbeGeometry}. \\

\noindent \textbf{Tensile Tests}\\
Tensile tests were performed with engineering stresses ranging from $4$ to $8$. \\

\noindent \textbf{SEM Specifications} \\
The specifications and settings of the SEM can be seen in table \ref{tab:SEM}.\\

\begin{table}[H]
 \begin{center}
  \begin{tabular}{@{} *2l @{}} \toprule[2pt]
   Model & Zeiss LEO 1530 \\\midrule
   Horizontal Field Width & $100\mu m$   \\ 
   Vertical Field Width  & $75\mu m$ \\ 
   Resolution  & $3072\times 2304$ \\
   Detector Type & Secondary Electrons \\
   Electron Source & Field-Emitter Cathode \\ \bottomrule[2pt]

   \label{tab:SEM}
  \end{tabular}
 \end{center}
 \caption{Details of the SEM used for the surface micrographs.}
\end{table}
