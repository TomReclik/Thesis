\chapter{Dual-Phase Steel}
\label{cha:DualPhaseSteels}

Dual-phase steels are a class of advanced high-strength steels. They combine a high ultimate tensile strength, a low initial yielding stress, high fracture strain, and a high hardening ratio. These advantageous properties together with their rather straightforward thermodynamical processing and light weight make them particularly interesting for the automotive industry, where they are mostly used for sheet metal parts. \\

%They exhibit multiple advantageous properties besides its rather straightforward thermodynamical processing.Combined with their low weight they are of particular interest in the automotive industry, where they are used for steel sheets. \\
%Dual-phase steels are a class of advanced high-strength steels. On a microstructural level they consist of a ductile ferrite matrix with islands of brittle martensite. Furthermore they can contain, austenite remaining from the processing of the material, pearlite, bainite carbides, and acicular ferrite. The hard martensite is responsible for the materials high ultimate strength, while the ductile ferrite results in a low initial yield strength. Those advantageous properties together with a high fracture strain, high hardening ratio, and their light weight makes them interesting from a engineering point of view, especially in the automotive industry, where dual-phase steels are used for steel sheets. \\


On a microstructural level dual-phase steels consist of a ductile ferrite matrix with islands of brittle martensite. While the microstructure is seemingly simple, its influence on the materials mechanical properties is not fully understood and research since the patent of dual-phase steels in 1968 \cite{dualphassteel} is still ongoing. The microstructural parameters include the martensite volume fraction, the size of martensite islands, and the overall morphology. The configuration of these parameters depends on the chemical composition and processing of the material. 

%On a microstructural level dual-phase steels consist of a ductile ferrite matrix with islands of brittle martensite. They furthermore can contain austenite, pearlite, bainite, carbides and acicular ferrites. While the microstructure of dual-phase steels is seemingly simple, consisting mostly of two While the microstructure of dual-phase steels is seemingly simple, its influence on the mechanical properties of dual-phase steels is only partially understood and research is ongoing PAPER. This is due to the fact that multiple microstructural parameters, like the martensite volume ratio, the size of martensite islands, the morphology, etc. play an important role in the macroscopic behavior. The microstructure at the surface of a sample is shown in figure \ref{fig:DPMicrostructure}. \\

\begin{figure}[H]
\centering
  \includegraphics[width=\linewidth]{DPMicrograph.png}
  \caption{Dual-phase steel SEM micrograph.}
  \label{fig:DPMicrostructure}
\end{figure}

\section{Damage Mechanisms}



 

%In dual-phase steels, with two constituents, these result in damage sites at the interface between ferrite and ferrite grains, called boundary decohesion, and at the interface between ferrite and martensite, called interface decohesion. Since the martensite is embedded in a ferrite matrix, grain boundaries between two martensite grains do not occur and therefore no voids accumulate at the grain boundary between two martensite grains. Due to the hardness of the martensite, cracks can occur in a single martensite grain, called martensite cracking. These are heavily influenced by the stress distribution and the geometry of the martensite grain.\\


%During deformation dual-phase steels firstly deform strictly elastically. If the applied stress is released the sample returns to its original state. Increasing the stress further up to the yield stress the material will be deformed permanently but upon releasing the applied stress it will return elastically to a different geometry. Beyond the yield stress damages nucleate inside the material permanently damaging the material. The yield stress together with the tensile strength and ultimate elongation are marked in the stress strain that can be seen in the appendix \ref{fig:DPStressStrain}.


%While under tensile stress, the material exhibits irreversible deformation at stresses above the yield point. A stress-strain curve is shown in figure \ref{fig:DPStressStrain}. Beyond this point voids can nucleate in the microstructure. These voids are classified based on their relative position to the constituents of the dual-phase steel. The classes are as follows
During deformation, materials exhibit different behaviors depending on the applied stress. At first, a material reacts elastically, returning to its original state when the stress is released. Increasing the stress further, the regime of plastic deformation is reached, and permanent changes to the microstructure of the material are introduced. Additionally to the plastic deformation, in dual-phase steels voids nucleate in the microstructure, becouse of the stress and strain partitioning between the two phases and their plastic incompatibility. These voids can act as sources and propagators of a critical crack leading to the failure of the material if the stress is further increased. The emerging voids in the materials microstructure are classified based on their relative position to the constituents of the dual-phase steel. A common classification is as follows
\begin{itemize}[label={}]
\item \textbf{Martensite Cracking}: A brittle crack nucleating inside a martensite island.
\item \textbf{Interface Decohesion}: Nucleation of a void at the interface between a martensite island and a ferrite grain.
\item \textbf{Boundary Decohesion}: Nucleation of a void at the grain boundary between two adjacent ferrite grains.
\end{itemize}
Furthermore, important for the materials behavior are foreign particles. These do not nucleate under stress but are introduced to the material during its processing. In SEM micrographs, these inclusions can also be seen as voids, albeit commonly of larger size.
\begin{itemize}[label={}]
\item \textbf{Inclusions}: A foreign particle remaining from the processing of the material. 
\end{itemize}
Abstracted examples of the possible damage mechanisms are shown in figure \ref{fig:DamageCategories_abstraction}. \\


%Due to the high concentration of martensite in the dual-phase steel used in this study, boundary decohesions occur rarely or not at all. In in-situ experiments, most of the sites classified at a certain stage, without knowledge of the previous state, as boundary decohesions turned out to be evolved notch effect sites. Examples for each damage category can be seen in figure \ref{fig:DamageCategories}.\\

\begin{figure}[H]
\begin{subfigure}{.25\textwidth}
\centering
  \includegraphics[width=.8\linewidth]{MartensiteCracking_abstraction_scalebar.pdf}
  \caption{MC}
  \label{fig:MC}
\end{subfigure}%
\begin{subfigure}{.25\textwidth}
\centering
  \includegraphics[width=.8\linewidth]{InterfaceDecohesion_abstraction_scalebar.pdf}
  \caption{ID}
  \label{fig:Interface_scalebar}
\end{subfigure}%
\centering
\begin{subfigure}{.25\textwidth}
\centering
  \includegraphics[width=.8\linewidth]{BoundaryDecohesion_abstraction_scalebar.pdf}
  \caption{BD}
  \label{fig:Inclusion_scalebar}
\end{subfigure}%
%\begin{subfigure}{.2\textwidth}
%\centering
%  \includegraphics[width=.8\linewidth]{NotchEffect_abstraction_scalebar.pdf}
%  \caption{NE}
%  \label{fig:Notch_scalebar}
%\end{subfigure}%
\centering
\begin{subfigure}{.25\textwidth}
\centering
  \includegraphics[width=.8\linewidth]{Inclusion_abstraction_scalebar.pdf}
  \caption{Inclusion}
  \label{fig:Inclusion_scalebar}
\end{subfigure}%
\caption{Abstracted damage mechanism for martensite cracking (MC) in subfigure (a), interface decohesion (ID) in subfigure (b), boundary decohesion (BD) in subfigure (c),  and inclusions in subfigure (d). The martensite islands, the ferrite matrix, and the foreign body are labeled as M, F, and I respectively.}
\label{fig:DamageCategories_abstraction}
\end{figure}

\section{Experimental Setup}

In order to study the mechanisms behind the deformation of a material and its failure, experiments have to be performed investigating the materials response to different kinds and levels of plastic strain under a certain stress state. By studying the response of the microstructure, specifically the formation of voids, a deeper understanding of the materials mechanic properties can be generated. While it is possible to use three dimensional imaging, these methods are limited by their time intensity and are usually limited to small portions of a probe. Another possibility is to use surface micrographs, these are limited by voids nucleating at or evolving to the surface of the specimen, but larger portions of the probe can be imaged. Furthermore two types of experiments can be performed, explained in the following. \\

\noindent \textbf{Ex-Situ}\\
In ex-situ experiments a sample is deformed, then metallographically prepared in order to be able to distinguish the constituents on the surface of the material. Afterwards, the surface of the specimen is scanned and investigated. This method is restricted to temporal snapshots of damage mechanisms. However, a view of the damage in the bulk of the material is possible, as material is ground during the metallographic preparation.\\

\noindent \textbf{In-Situ}\\
In in-situ experiments, the surface of the sample is prepared beforehand and the deformation experiment is performed afterwards with different stages of stress, enabling the study of the time evolution of damage sites. The downside to this method is the influence of the preparation of the materials surface on its deformation behavior. Furthermore, the nucleation and evolution of damage sites at the surface of the sample differs from the bulk properties as the stress states at th surface are altered. \\

The data used in this thesis stems from uniaxial tensile tests with surface micrographs from both ex-situ and in-situ experiments. Detail of the experimental setup can be found in the appendix \ref{app:experiment}.

%In order to investigate the mechanisms behind the deformation of a material and its failure, the material is put under stress and its response is investigated on a microstructural level. Such tests can for example involve putting a probe of the material uniaxial or biaxial tensile tests, or bending tests. The data recorded for this thesis stems from unixial tensile tests. While it is possible to investigate the microstructures response with three dimensional imaging, these methods are very time intensive and therefore restricted to a small portion of the probe. A simpler method is to record the surface of the probe using an electron microscope, restricting insights to damages either nucleating or evolving to the surface at the benefit of being able to investigating a larger portion of the probe. \\

%Furthermore it is possible to either deform a material and then prepare its surface in order to enable the distinction between its constituents at the cost of recording only snapshots of damage sites, or to prepare the material and record micrographs of the same material at different levels of stress at the cost of influencing the materials behavior through the preparation of its surface, but enabling the recording of the evolution of damage sites. These types of experiments are called ex-situ and in-situ experiments respectively. \\

%The details of the experiments are explained in the appendix \ref{app:experiment}



%\textbf{Preparation} \\
%Before the experiments were performed, the surface of the dual-phase steel was prepared in order to be able to distinguish the martensite islands from the ferrite matrix, in surface micrographs. The preparation process consists of grinding the surface with up to $4000$ grit sandpaper, polishing it with oxide polishing suspension in steps of $6\mu m$, $3\mu m$, and $1\mu m$, and finally etching it with $1 \%$ nital. \\
%
%\noindent \textbf{Probe Geometry} \\
%The probe geometry was chosen in such a way that the center of the probe experiences almost homogeneous stress. It can be seen in figure \ref{fig:ProbeGeometry}. \\
%
%\noindent \textbf{Tensile Tests}\\
%In order to study the damage mechanisms inside the materials microstructure, stresses were applied beyond its yield stress. \\
%
%\noindent \textbf{SEM Specifications} \\
%The surface of the material after deformation was recorded using an SEM. Its specifications and settings can be seen in table \ref{tab:SEM}. \\
%
%\begin{table}[H]
% \begin{center}
%  \begin{tabular}{@{} *2l @{}} \toprule[2pt]
%   Model & Zeiss LEO 1530 \\\midrule
%   Horizontal Field Width & $100\mu m$   \\ 
%   Vertical Field Width  & $75\mu m$ \\ 
%   Resolution  & $3072\times 2304$ \\
%   Detector Type & Secondary Electrons \\
%   Electron Source & Field-Emitter Cathode \\ \bottomrule[2pt]
%
%  \end{tabular}
% \end{center}
% \caption{Details of the SEM used for the surface micrographs.}
%   \label{tab:SEM}
%\end{table}
