\chapter{Dual-Phase Steel}

Dual-phase steels are a class of advanced high-strength steels. On a microstructural level dual-phase steels consist of a ductile ferrite matrix with islands of brittle martensite. It furthermore can contain austenite, pearlite, bainite, carbides and acicular ferrites. From an engineering point of view it exhibits multiple advantageous properties besides its rather straightforward thermodynamical processing. Dual-phase steels combine a high ultimate tensile strength, caused by the brittle martensite, a low initial yielding stress, enabled by the ductile ferrite, high fracture strain, high hardening ratio. Combined with their low weight they are of particular interest in the automotive industry, where they are used for steel sheets. \\
%Dual-phase steels are a class of advanced high-strength steels. On a microstructural level they consist of a ductile ferrite matrix with islands of brittle martensite. Furthermore they can contain, austenite remaining from the processing of the material, pearlite, bainite carbides, and acicular ferrite. The hard martensite is responsible for the materials high ultimate strength, while the ductile ferrite results in a low initial yield strength. Those advantageous properties together with a high fracture strain, high hardening ratio, and their light weight makes them interesting from a engineering point of view, especially in the automotive industry, where dual-phase steels are used for steel sheets. \\

While the microstructure of dual-phase steels is seemingly simple, its influence on the mechanical properties of dual-phase steels is only partially understood and research is ongoing. This is due to the fact that multiple microstructural parameters, like the martensite volume ratio, the size of martensite islands, the morphology, etc. play an important role for the macroscopic behavior. The microstructure at the surface of a sample is shown in figure \ref{fig:DPMicrostructure}. \\

\begin{figure}
\centering
  \includegraphics[width=\linewidth]{DPSteelMicrostructure.png}
  \caption{Dual-phase steel SEM micrograph}
  \label{fig:DPMicrostructure}
\end{figure}

\section{Damage Mechanisms}

While under tensile stress, the material exhibits irreversible deformation at stresses above the yield point. A stress-strain curve is shown in figure \ref{fig:DPStressStrain}. Beyond this point voids can nucleate in the microstructure. These voids are classified based on their relative position to the constituents of the dual-phase steel. The classes are as follows
\begin{itemize}[label={}]
\item \textbf{Martensite Cracking}: A crack nucleating inside a martensite island.
\item \textbf{Interface Decohesion}: Nucleation of a void at the interface between a martensite island and a ferrite grain.
\item \textbf{Boundary Decohesion}: Nucleation of a void at the interface between two adjacent ferrite grains.
\item \textbf{Notch Effect}: A void nucleating between the tips of two martensite islands. 
\end{itemize}
Additionally influencing the materials behavior are
\begin{itemize}[label={}]
\item \textbf{Inclusions}: A foreign particle remaining from the processing of the material. 
\end{itemize}
Due to the high concentration of martensite in the dual-phase steel used in this study, boundary decohesions occur rarely or not at all. In in-situ experiments, most of the sites classified at a certain stage, without knowledge of the previous state, as boundary decohesions turned out to be evolved notch effect sites. Examples for each damage category can be seen in figure \ref{fig:DamageCategories}.\\

\begin{figure}
\centering
\begin{subfigure}{.25\textwidth}
\centering
  \includegraphics[width=.8\linewidth]{Inclusion_scalebar.png}
  \caption{Inclusion}
  \label{fig:Inclusion_scalebar}
\end{subfigure}%
\begin{subfigure}{.25\textwidth}
\centering
  \includegraphics[width=.8\linewidth]{Martensite_scalebar.png}
  \caption{MC}
  \label{fig:Martensite_scalebar}
\end{subfigure}%
\begin{subfigure}{.25\textwidth}
\centering
  \includegraphics[width=.8\linewidth]{Interface_scalebar.png}
  \caption{ID}
  \label{fig:Interface_scalebar}
\end{subfigure}%
\begin{subfigure}{.25\textwidth}
\centering
  \includegraphics[width=.8\linewidth]{Notch_scalebar.png}
  \caption{NE}
  \label{fig:Notch_scalebar}
\end{subfigure}%
\caption{Examples of damage sites.}
\label{fig:DamageCategories}
\end{figure}

\section{Experimental Setup}

Before the tensile tests were performed, the samples were polished and its surface was etched in order to be able to distinct martensite and ferrite in the recorded micrographs. The probe geometry can be seen in figure \ref{fig:ProbeGeometry}. Stress was applied uniaxial. For the recording of the micrographs a scanning electron microscope (SEM) was used. The parameter settings can be seen in table \ref{tab:SEMSettings}.