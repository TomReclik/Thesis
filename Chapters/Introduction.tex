\chapter{Introduction}

The continuous drive in the transportation industry to use lighter components that still meet safety regulations, for economical and ecological reasons, requires a deeper understanding of the underlying mechanisms responsible for the irreversible deformation and failure of materials. Due to the emerging complexity of macroscopic properties from microscopic features, ab initio theories only give crude approximations, and experiments are needed to observe the response of the microstructure to external deformation. By understanding the mechanisms behind the irreversible deformation and failure, new materials can be designed to exhibit desired mechanical properties.\\

%Due to the emerging complexity of the macroscopic properties of a material from their microstructure, experiments are needed, exploring the response of the microstructure to the deformation of the macroscopic material. A promising approach to improve the materials properties then is to reversely design a microstructure more resistant to the mechanisms behind irreversible deformations and failure. \\

%In materials with a heterogeneous microstructure, before the onset of failure under tensile stress, voids can nucleate inside the material. For the formation of a critical crack in the material and therefore its failure those voids act as sources for its nucleation and propagation. Understanding the damage mechanisms is therefore crucial in order to improve the materials resilience against failure, by e.g. engineering a microstructure, suppressing the predominant damage mechanism. A material improved in that way can on the one hand be used in order to meet the demand in modern transportation engineering by the reduction of a components weight resulting in lower fuel consumption and on the other hand in general create a component less prone to failure and therefore with a longer lifetime, resulting in lower costs in maintenance.\\

The deformation of a material leads to changes in the microstructure, e.g. emerging voids between different constituents in a material. Classifying those changes without computational aid leads to a focus on a few sites or extended periods of time required for the collection of statistically relevant data. Developing such a classification algorithm is therefore crucial for a deeper understanding of the deformation behavior. \\

%In heterogeneous materials its deformation can result in the formation of voids between the different constituents of the material on a microscopic level, which at higher levels of stress act as sources and propagators of a critical crack leading to the failure of the material. Because of the emerging properties of the macroscopic material from its microscopical features, statistically relevant data on the microstructure's response to the externally applied stress is needed. Evaluating such a statistically relevant dataset without an automated algorithmic aid is very time intensive. \\

While it is easy for humans to classify objects in images, it is extremely difficult to formulate the classification task into strict algorithmic rules for realistic problems. Machine learning algorithms are capable of closing this semantic gap, by classifying images based on extracted features, i.e. properties of the object represented in the image, either by hand engineering them, or by letting the machine learning algorithm learn the relevant features by themselves. \\

%While it is easy for a human to classify objects based on images, formulating strict algorithmic rules for a classification task is infeasible for realistic problems. Usually machine learning is used for classification tasks. There are generally two approaches. In the first features are hand engineered and extracted and then given to some machine learning algorithm, such as an decision tree actually classifying. The second approach is to let the algorithm directly learn the relevant features and classify the object using the features it learned by itself, such as in convolutional neural networks (CNNs). \\


%Gathering the necessary data is very time intensive, consisting of preparing and deforming the material, recording the micrograph, and evaluating these micrographs. The last step is the most promising point of attack for the reduction of needed human supervision using minimal resources. The evaluation itself consists of two steps, localization and classification of damage sites. Automating this process would enable the collection of statistically relevant data for ex-situ and in-situ experiments, and therefore the determination of ,for example, the predominant damage mechanism. Furthermore this makes it possible to track the evolution of a large number of damage sites in in-situ experiments through different stages of stress, gaining a deeper understanding of the physical processes of damage nucleation and growth.\\

%Currently localization is either done by searching manually or using some form of algorithmic rule together with a clustering algorithm reducing the overall space needed to be searched, while the classification has to be performed entirely by hand. This comes with two major problems. Firstly the amount of time needed in order to gather a dataset with statistical relevancy is unjustifiable and secondly the concentration while searching and classifying over long periods of time decreases, resulting in the increasing of the misclassification rate.\\

%The goal of this thesis is to automate this process. Therefore enabling an automatic collection of statistically relevant data for ex-situ and in-situ experiments, finding for example the predominant mechanism during deformation. Additionally making it possible to track the evolution of a large number of damage sites in in-situ experiments through different stages of stress, gaining a deeper understanding of the physical processes of damage nucleation and growth.\\

While other works in material science have implemented algorithms for the classification of damage sites by using hand tailored features \cite{Chen2013}, \cite{Zapata2011}, these methods are restricted to the specific material in use. The aim of this master thesis is to implement an algorithm applicable to a variety of materials. \\

%, e.g. Chen et al. extracted hand tailored features and used a decision tree for the localization and classification of engineering ceramic grinding surface damages \cite{Chen2013} and Zapata et al. similarly extracted features and classified using an artificial neural network (ANN) for the classification of weld defects in radiographic images \cite{Zapata2011}, the aim of this thesis is to implement an algorithm needing minimal preprocessing in order to be easily applicable for damage mechanisms in a variety of materials. \\

%While other works have implemented such an algorithm for certain materials, e.g. Chen et al. extracted hand tailored features and used a decision tree for the localization and classification of engineering ceramic grinding surface damages \cite{Chen2013} and Zapata et al. similarly extracted features and classified using an artificial neural network (ANN) for the classification of weld defects in radiographic images \cite{Zapata2011}, the aim of this thesis is to implement an algorithm needing minimal preprocessing in order to be easily applicable for damage mechanisms in a variety of materials. \\

%While other works have achieved this goal for certain materials, e.g. Chen et al. extracted hand tailored features and used a decision tree for the localization and classification of engineering ceramic grinding surface damages \cite{Chen2013} and Zapata et al. similarly extracted features and classified using an artificial neural network (ANN) for tWhile other works have implemented such an algorithm for certain materials, e.g. Chen et al. extracted hand tailored features and used a decision tree for the localization and classification of engineering ceramic grinding surface damages \cite{Chen2013} and Zapata et al. similarly extracted features and classified using an artificial neural network (ANN) for the classification of weld defects in radiographic images \cite{Zapata2011}, the aim of this thesis is to implement an algorithm needing minimal preprocessing in order to be easily applicable for damage mechanisms in a variety of materials. \\
%he classification of weld defects in radiographic images \cite{Zapata2011}, the aim of this thesis is to implement an algorithm needing minimal preprocessing in order to be easily applicable for damage mechanisms in a variety of materials. \\

Convolutional neural networks (CNNs) are a class of classification algorithms that need minimal to no preprocessing and are capable of automatically learning relevant features for the needed classification task. Furthermore they have proven to be reliable classification tools, e.g. winning the ImageNet Large Scale Visual Recognition Challenge \cite{imagenet_cvpr09} annually since 2012. Therefore they find heavy use in other fields of science and engineering, like in medicine for the automatic classification of skin cancer \cite{CNNSkinCancer}. \\

%In other fields of science and engineering it is commonplace to use convolutional neural networks (CNNs) for the task of object classification in images, e.g. the automatic classification of skin cancer by Esteva et al. \cite{CNNSkinCancer}. The advantage of CNNs is that they do not need preprocessing of inputs and hand engineered features. On common classification challenges like the ImageNet Large Scale Visual Recognition Challenge \cite{imagenet_cvpr09}, conducted annually since 2010, since 2012 CNNs take the first place each year. Showing that they are reliable classification tools. \\

Dual-phase steel was used as a sample material in this work, firstly because it is one of the most widely used advanced high-strength steels and secondly due to the fact that damage sites right after nucleation are clearly distinguishable on the surface of the material. The deformation behavior was investigated by applying uniaxial stress to the material and recording surface micrographs using a scanning electron microscope (SEM). Even though different methods exist that gain a deeper insight into the material than surface micrographs, e.g. electron backscatter diffraction, these methods are limited by the time required to record the necessary data. \\

The goal of this thesis is to automate the process of localizing and classifying damage sites in surface micrographs of dual-phase steels under stress, using a combination of machine learning algorithms and CNNs. This tool would enable the automated collection of statistically relevant data, for determining the predominant damage mechanism. Furthermore making it possible to track the evolution of a large number of damage sites, resulting in a deeper understanding of the underlying mechanisms. Ultimately this algorithm can be easily applied to the classification of damage sites in different materials, that appear as voids in the microstructure. \\

%Furthermore making it possible to aid the tracking of the evolution of damage sites, thereby 
%
% and also the automated tracking of the evolution of damage sites, gaining a deeper understanding into the underlying mechanisms, while also being easily applicable to classifying damage sites in materials, that show themselves as voids in the microstructure. \\

%The goal of this thesis is to automate the process of localizing and classifying damage sites in ex-situ and in-situ experiments using a combination of machine learning algorithms and CNNs. As a sample material dual-phase steel is used, since the damage mechanisms right after nucleation are clearly distinguishable from each other. But ultimately the created tool should be applicable to other materials as well. Due to the time intensity of data collection, one of the main challenges is the collection of a suitable dataset for the training of the CNNs. \\


%In the following ....