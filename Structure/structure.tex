\documentclass[10pt,a4paper]{report}
\usepackage[utf8]{inputenc}
\usepackage[english]{babel}
\usepackage{amsmath}
\usepackage{amsfonts}
\usepackage{amssymb}
\usepackage{graphicx}
\usepackage{subcaption}\usepackage{bm}
\usepackage{booktabs}
\usepackage{float}
\usepackage[bookmarks]{hyperref}
\restylefloat{table}
\graphicspath{{../Images/}}
\newcommand\tab[1][0.5cm]{\hspace*{#1}}
\renewcommand{\baselinestretch}{1.25} 


\def\boxit#1{%
  \smash{\fboxsep=0pt\llap{\rlap{\fbox{\strut\makebox[#1]{}}}~}}\ignorespaces
}

\begin{document}
\chapter{Introduction}
\begin{itemize}
\item CNNs powerful tool for the classification of objects in images (ImageNet challenge)
\item Evaluation of damage sites in micrographs accompanied by a lot of damage analysis by hand $\rightarrow$ most of the time restriction to a few damage sites + errors performed by human evaluation
\item Goal: automate evaluation of large panoramas, in order to get statistically relevant data for ex-situ experiments + automated tracking of interesting damage sites in in-situ experiments
\end{itemize}

\chapter{Dual-Phase Steel}
\begin{itemize}
\item What is dual phase steel
\item What is it used for (automotive industry, steel sheets)
\item Why is it a good sample material for this problem (damage sites right after emergence clearly assignable to a certain class)
\item What kind of dual-phase steel was used: dual-phase steel 800
\begin{figure}
\centering
  \includegraphics[width=\linewidth]{PanoramaSmall2X_1Y_2.png}
  \caption{Dual-phase steel SEM micrograph}
  \label{fig:ANN}
\end{figure}

\end{itemize}
\section{Damage Mechanisms}
\begin{itemize}
\item Classes the network is trained to distinguish: inclusion, martensite cracking, interface decohesion, and notch effect
\item Damages at the grain boundary between two adjacent ferrite grains not considered, due to the high proportion of martensite + experiments by conducted by Carl most damage sites looking like boundary damages are evolved notch effects
\item Other types of damage categories, goal of finding the mechanisms behind evolved damage sites
\end{itemize}
\section{Experimental Setup}
\begin{itemize}
\item Experimental setup: uniaxial experiments, resolution etc
\end{itemize}

\chapter{Convolutional Neural Networks}
\section{Artificial Neurons}
\begin{itemize}
\item Basic building block of an artificial neural network
\item Inspired by the biological neuron
\end{itemize}
\begin{figure}
\centering
  \includegraphics[width=0.7\linewidth]{Perceptron_cropped.pdf}
  \caption{Perceptron}
  \label{fig:Neuron}
\end{figure}
\section{Artificial Neural Networks}
\begin{itemize}
\item Collection of neurons
\item Topology, e.g. first approaches were to connect every neuron in each layer to every neuron in the next layer not applicable to image recognition
\item Trainable weights
\end{itemize}
\begin{figure}
\centering
  \includegraphics[width=0.9\linewidth]{SimpleNeuron-crop.pdf}
  \caption{Artificial Neural Network}
  \label{fig:ANN}
\end{figure}
\section{Linear Image Filter}
\begin{itemize}
\item Assigning label to images possible for humans $\rightarrow$ some function mapping the image to its class should exist
\item Neural networks as function approximators
\item How to find the right topology for the network to perform well
\item Motivation using the visual cortex of animals, based on the experiments performed by Wiesel and Hubel
\item Discrete convolutions
\item In image processing convolutions are used to transform images and find features like edges (Sobel filter)
\end{itemize}

\begin{figure}
\centering
  \includegraphics[width=0.9\linewidth]{vis_mnist_conv.png}
  \caption{First level features learned for the recognition of digits}
  \label{fig:ANN}
\end{figure}

\section{Convolutional Neural Network - Topology}
\begin{itemize}
\item Change topology appropriately to the inherent topology of images
\item Introduce perceptive field for neurons
\item Weight sharing among neurons
\item Results in the action of a convolution
\item Use multiple such convolutions in each layer, leading to multiple feature maps

\end{itemize}
\begin{figure}
\centering
  \includegraphics[width=0.9\linewidth]{tDogClassMapping.png}
  \caption{Intuitive mapping between an image and its class}
  \label{fig:ANN}
\end{figure}

\section{Additional Layers}
\begin{itemize}
\item Pooling
\item Dropout
\item Batch normalization
\item Output layer (softmax)
\end{itemize}

\section{Training}
\begin{itemize}
\item Training using examples, train/validation/test splits
\item Backpropagation
\item Early stopping as a further method of preventing overfitting
\end{itemize}

\chapter{Localization}

\begin{figure}
\begin{subfigure}{.5\textwidth}
\centering
  \includegraphics[width=.8\linewidth]{DBScan_firststep.png}
  \caption{Gray scale cutoff}
  \label{fig:Inclusion_scalebar}
\end{subfigure}
\begin{subfigure}{.5\textwidth}
\centering
  \includegraphics[width=.8\linewidth]{DBScan_secondstep.png}
  \caption{Clusters found using DBSCAN}
  \label{fig:Inclusion_scalebar}
\end{subfigure}
\caption{Localization of damage sites using DBSCAN. (Use smaller images)}
\end{figure}

\chapter{Architecture}
\begin{itemize}
\item Inherent size difference between different damage classes
\item Use two CNNs which differ in the size of their perceptive fields
\end{itemize}

\begin{figure}
\centering
\begin{subfigure}{.25\textwidth}
\centering
  \includegraphics[width=.8\linewidth]{Inclusion_scalebar.png}
  \caption{Inclusion}
  \label{fig:Inclusion_scalebar}
\end{subfigure}%
\begin{subfigure}{.25\textwidth}
\centering
  \includegraphics[width=.8\linewidth]{Martensite_scalebar.png}
  \caption{MC}
  \label{fig:Martensite_scalebar}
\end{subfigure}%
\begin{subfigure}{.25\textwidth}
\centering
  \includegraphics[width=.8\linewidth]{Interface_scalebar.png}
  \caption{ID}
  \label{fig:Interface_scalebar}
\end{subfigure}%
\begin{subfigure}{.25\textwidth}
\centering
  \includegraphics[width=.8\linewidth]{Notch_scalebar.png}
  \caption{NE}
  \label{fig:Notch_scalebar}
\end{subfigure}%
\caption{The different categories of damage sites, that the classifiers are trained to distinguish from each other together with a scale bar.}
\label{fig:SizeDifference}
\end{figure}

\begin{figure}
\begin{center}
  \includegraphics[width=\linewidth]{Architecture.pdf}
\caption{Architecture}
\label{fig:Architecture}
\end{center}
\end{figure}

\chapter{Evaluation and Results}
\section{Datasets}
\begin{itemize}
\item Amount of data per class split into ex-situ and in-situ data
\item Training parameters and dataset splits used
\end{itemize}
\section{First Classifier}
\subsection{Network Architecture}
\begin{itemize}
\item Tests on most popular implemented CNN architectures
\end{itemize}
\subsection{Generalization to new Datasets}
\subsection{Choosing a Threshold}
\subsection{Misclassified Damage Sites}
\subsection{Robustness}

\section{Second Classifier}
\subsection{Network Architecture}
\begin{itemize}
\item Simpler architecture due to the smaller size of the damage sites
\item Best performing network from previous studies on the CIFAR-10 dataset: EERACN
\end{itemize}
\subsection{Further Split of Hierarchy}
\begin{itemize}
\item Firstly distinguish between brittle (martensite cracking) and ductile mechanisms (interface decohesion and notch effects)
\item Then distinguish the remaining two classes
\item Experiment shows no improvement
\end{itemize}
\subsection{Generalization to new Datasets}
\subsection{Choosing a Threshold}
\subsection{Misclassified Damage Sites}
\subsection{Robustness}

\section{Combined Classifier}
\subsection{Performance}
\subsection{In-situ Analysis with Position Tracking}

\chapter{Proposed Workflow}
\begin{itemize}
\item Train classifier
\item Review purity efficiency
\item Choose threshold high enough for the desired accuracy
\item Label not classified damage sites by hand and reintroduce them to the training since they represent instances promising to increases the classifiers performance
\end{itemize}

\chapter{Conclusion}
\begin{itemize}
\item Classification of inclusions works fine
\item Classification of Martensite crackings also fine
\item Classification of interface decohesions and notch effects problematic
\item Need more data
\item Significant reduction of the amount of work required for the evaluation of damage sites in dual-phase steels
\item Possible increase in accuracy by using additional information like phase information (Ferrite and Martensite marked) as in figure \ref{fig:Phases}
\item On beam camera (reduce shadows)
\end{itemize}

\begin{figure}
\centering
  \includegraphics[width=0.9\linewidth]{AwesomeClassifier.png}
  \caption{Classifier marking martensite grains and ferrite matrix}
  \label{fig:Phases}
\end{figure}


\end{document}