\documentclass[12pt,a4paper]{report}
\usepackage[utf8]{inputenc}
\usepackage[english]{babel}
\usepackage{amsmath}
\usepackage{amsfonts}
\usepackage{amssymb}
\usepackage{graphicx}
\usepackage{subcaption}\usepackage{bm}
\usepackage{booktabs}
\usepackage{float}
\usepackage[bookmarks]{hyperref}
\usepackage{enumitem}
\usepackage{multicol}
\usepackage{relsize}
\usepackage{mathtools}
\usepackage[font={small}]{caption}
\usepackage{fancyhdr}
\usepackage{gensymb}

\pagestyle{fancy} %eigener Seitenstil
\fancyhf{} %alle Kopf- und Fußzeilenfelder bereinigen
\fancyhead[R]{\smaller{\rightmark}} %Kopfzeile links
\fancyhead[C]{} %zentrierte Kopfzeile
\fancyhead[L]{\ifnum\value{chapter}>0 \chaptername\ \thechapter. \fi} %Kopfzeile rechts
\fancyfoot[C]{\thepage}


\DeclarePairedDelimiter\ceil{\lceil}{\rceil}
\DeclarePairedDelimiter\floor{\lfloor}{\rfloor}

\restylefloat{table}
\graphicspath{{./Images/}}
\newcommand\tab[1][0.5cm]{\hspace*{#1}}
\renewcommand{\baselinestretch}{1.5} 

\title{Application of Deep Convolutional Neural Network for the Classification of Damage Sites in Dual-Phase Steels}
\author{Tom Markus Reclik}

\def\boxit#1{%
  \smash{\fboxsep=0pt\llap{\rlap{\fbox{\strut\makebox[#1]{}}}~}}\ignorespaces
}

\begin{document}
%\maketitle
%\begin{abstract}
%To this day, various methods are used in order to reveal the micromechanical mechanisms of damage in materials. Post-mortem analysis at different stages of stress reveals only snapshots of the material, while in-situ methods are spatially limited to observing the evolution of only a few damage events. A limiting factor in all those methods is the amount of work involved in controlling the microscope and the image analysis. \\
%\tab[0.4cm] In this work, we implement different structures and architectures of neural networks for the localization and classification of damages. As a sample material dual-phase steels are chosen, due to the different responses of the ductile ferrite matrix and the brittle martensite islands to stress, resulting in the formation of damage sites belonging to distinct classes at early stages of deformation. \\
%\tab[0.4cm] The developed algorithms can on the one hand be used in order to automate the statistical evaluation of post-mortem micrographs, while on the other hand enabling in-situ experiments to generate statistically relevant data. Due to the computational nature of this method, a high throughput of data is possible, enabling a more complete understanding of failure mechanisms in many materials.
%
%
%
%
%
%%In material science, failure mechanisms in metals are classically studied based on the evolution of damages in micrographs observed at different stages of stress. The collection of statistically meaningful data is therefore an important step in the understanding of these processes. A limiting factor is the amount of work involved in the control of the microscope and image analysis. 
%%The aim of this master thesis is to automate the process of gathering data by employing machine learning algorithms. As a sample application, dual-phase steels are studied. Due to their light weight and high strength, they play an important role in the automotive industry. Consisting of a soft ferrite matrix with islands of hard martensite, different kinds of damages can emerge under stress, e.g. martensite islands can crack, damages between two ferrite grains can emerge, etc. 
%%In a first step, convolutional neural networks, and derivatives thereof, are used in order to find and classify damages. Due to the limited amount of relevant data, two challenges arise. On the one hand, data has to be collected experimentally and afterwards labeled by hand. On the other hand, a network architecture has to be found capable of achieving proper accuracy on small data sets. 
%%While the architecture to be used in the end has to be designed for the experimental data, a few preliminary studies will be performed on standard sets of images, namely the CIFAR-10 data set and the ImageNet data set.
%
%
%%Collecting statistically meaningfull data is limited by the large amount of work involved in the image analysis and control of the microscope. 
%%The aim of this master thesis is to automate this process and overcome those limitations by employing machine learning algorithms. Convolutional neural networks are used in order to classify different categories of damages. Specifically dual-phase steels will be studied in this thesis.
%%Dual-phase steel plays a prominent role in advanced high strength steel. Due to its light weight and high strength they are often used in the automotive industry.
%%Due to limited amount of relevant data, two challanges arise. On the one hand data has to be collected and labeled on the experimental side. On the other hand network architectures have to be found capable of achieving proper accuracy on small data sets. 
%%While the architecture to be used for the classification problem has to be designed on the experimental data, a few preliminary studies will be peformed on standard set of images, namely the CIFAR-10 data set and the ImageNet data set.
%
%%Dual phase steel plays a prominent role in advanced high strength steel. Its light weight and high strength make it interesting for fields like the automotive industry. 
%%On the mesoscopic scale different kinds of damages can emerge under uniaxial and biaxial tension. Gathering statisticly meaningful data is an important means for understanding (and optimizing?) its mechanical properties. Up to this point the process of collecting data is performed, either manually or additionally aided by computational means, by inspecting scanning electron microscope pictures. This poses a bottleneck for the collection of large volumes of data for later analysis. In this thesis we ask whether it is possible to automatize this process of locating and categorizing damages by means of artificial convolutional neural networks.
%%Approaching this task multiple problems emerge. Firstly there is only a low volume of data for the training of the constructed convolutional neural network. In a first effort of tackling this problem, we investigate the influence of the size of the training data on a sample data set, namely the CIFAR-10 \cite{CIFAR-10} data set. This gives us a first idea of what to expect in the later process and what network architectures are most fit to work on small data sets. Secondly damages are defined by their surroundings rather than by themselves, making this an unique challange to tackle. (Here later more about different architectures)
%
%%Different kinds of damages can emerge in dual phase steel under uniaxial and biaxial tension. Gathering statisticly meaningful data is an important means for understanding its mechanical properties. The process of collecting data to this point is performed either manually or partly assisted by computational means. In this thesis we ask whether it is possible to automatize this process of locating and categorizating damages by means of artificial neural networks. Solving this problem we face to problems. On the one hand a low volume of training data compared to classical categorization problems. Secondly damages are objects mostly defined by their surroundings rather than by their own properties.
%\end{abstract}
%\chapter{Introduction}

The continuous drive in the transportation industry to use lighter components that still meet safety regulations, for economical and ecological reasons, requires a deeper understanding of the underlying mechanisms responsible for the irreversible deformation and failure of materials. Due to the emerging complexity of macroscopic properties from microscopic features, ab initio theories only give crude approximations, and experiments are needed to observe the response of the microstructure to external deformation. By understanding the mechanisms behind the irreversible deformation and failure, new materials can be designed to exhibit desired mechanical properties.\\

%Due to the emerging complexity of the macroscopic properties of a material from their microstructure, experiments are needed, exploring the response of the microstructure to the deformation of the macroscopic material. A promising approach to improve the materials properties then is to reversely design a microstructure more resistant to the mechanisms behind irreversible deformations and failure. \\

%In materials with a heterogeneous microstructure, before the onset of failure under tensile stress, voids can nucleate inside the material. For the formation of a critical crack in the material and therefore its failure those voids act as sources for its nucleation and propagation. Understanding the damage mechanisms is therefore crucial in order to improve the materials resilience against failure, by e.g. engineering a microstructure, suppressing the predominant damage mechanism. A material improved in that way can on the one hand be used in order to meet the demand in modern transportation engineering by the reduction of a components weight resulting in lower fuel consumption and on the other hand in general create a component less prone to failure and therefore with a longer lifetime, resulting in lower costs in maintenance.\\

The deformation of a material leads to changes in the microstructure, e.g. emerging voids between different constituents in a material. Classifying those changes without computational aid leads to a focus on a few sites or extended periods of time required for the collection of statistically relevant data. Developing such a classification algorithm is therefore crucial for a deeper understanding of the deformation behavior. \\

%In heterogeneous materials its deformation can result in the formation of voids between the different constituents of the material on a microscopic level, which at higher levels of stress act as sources and propagators of a critical crack leading to the failure of the material. Because of the emerging properties of the macroscopic material from its microscopical features, statistically relevant data on the microstructure's response to the externally applied stress is needed. Evaluating such a statistically relevant dataset without an automated algorithmic aid is very time intensive. \\

While it is easy for humans to classify objects in images, it is extremely difficult to formulate the classification task into strict algorithmic rules for realistic problems. Machine learning algorithms are capable of closing this semantic gap, by classifying images based on extracted features, i.e. properties of the object represented in the image, either by hand engineering them, or by letting the machine learning algorithm learn the relevant features by themselves. \\

%While it is easy for a human to classify objects based on images, formulating strict algorithmic rules for a classification task is infeasible for realistic problems. Usually machine learning is used for classification tasks. There are generally two approaches. In the first features are hand engineered and extracted and then given to some machine learning algorithm, such as an decision tree actually classifying. The second approach is to let the algorithm directly learn the relevant features and classify the object using the features it learned by itself, such as in convolutional neural networks (CNNs). \\


%Gathering the necessary data is very time intensive, consisting of preparing and deforming the material, recording the micrograph, and evaluating these micrographs. The last step is the most promising point of attack for the reduction of needed human supervision using minimal resources. The evaluation itself consists of two steps, localization and classification of damage sites. Automating this process would enable the collection of statistically relevant data for ex-situ and in-situ experiments, and therefore the determination of ,for example, the predominant damage mechanism. Furthermore this makes it possible to track the evolution of a large number of damage sites in in-situ experiments through different stages of stress, gaining a deeper understanding of the physical processes of damage nucleation and growth.\\

%Currently localization is either done by searching manually or using some form of algorithmic rule together with a clustering algorithm reducing the overall space needed to be searched, while the classification has to be performed entirely by hand. This comes with two major problems. Firstly the amount of time needed in order to gather a dataset with statistical relevancy is unjustifiable and secondly the concentration while searching and classifying over long periods of time decreases, resulting in the increasing of the misclassification rate.\\

%The goal of this thesis is to automate this process. Therefore enabling an automatic collection of statistically relevant data for ex-situ and in-situ experiments, finding for example the predominant mechanism during deformation. Additionally making it possible to track the evolution of a large number of damage sites in in-situ experiments through different stages of stress, gaining a deeper understanding of the physical processes of damage nucleation and growth.\\

While other works in material science have implemented algorithms for the classification of damage sites by using hand tailored features \cite{Chen2013}, \cite{Zapata2011}, these methods are restricted to the specific material in use. The aim of this master thesis is to implement an algorithm applicable to a variety of materials. \\

%, e.g. Chen et al. extracted hand tailored features and used a decision tree for the localization and classification of engineering ceramic grinding surface damages \cite{Chen2013} and Zapata et al. similarly extracted features and classified using an artificial neural network (ANN) for the classification of weld defects in radiographic images \cite{Zapata2011}, the aim of this thesis is to implement an algorithm needing minimal preprocessing in order to be easily applicable for damage mechanisms in a variety of materials. \\

%While other works have implemented such an algorithm for certain materials, e.g. Chen et al. extracted hand tailored features and used a decision tree for the localization and classification of engineering ceramic grinding surface damages \cite{Chen2013} and Zapata et al. similarly extracted features and classified using an artificial neural network (ANN) for the classification of weld defects in radiographic images \cite{Zapata2011}, the aim of this thesis is to implement an algorithm needing minimal preprocessing in order to be easily applicable for damage mechanisms in a variety of materials. \\

%While other works have achieved this goal for certain materials, e.g. Chen et al. extracted hand tailored features and used a decision tree for the localization and classification of engineering ceramic grinding surface damages \cite{Chen2013} and Zapata et al. similarly extracted features and classified using an artificial neural network (ANN) for tWhile other works have implemented such an algorithm for certain materials, e.g. Chen et al. extracted hand tailored features and used a decision tree for the localization and classification of engineering ceramic grinding surface damages \cite{Chen2013} and Zapata et al. similarly extracted features and classified using an artificial neural network (ANN) for the classification of weld defects in radiographic images \cite{Zapata2011}, the aim of this thesis is to implement an algorithm needing minimal preprocessing in order to be easily applicable for damage mechanisms in a variety of materials. \\
%he classification of weld defects in radiographic images \cite{Zapata2011}, the aim of this thesis is to implement an algorithm needing minimal preprocessing in order to be easily applicable for damage mechanisms in a variety of materials. \\

Convolutional neural networks (CNNs) are a class of classification algorithms that need minimal to no preprocessing and are capable of automatically learning relevant features for the needed classification task. Furthermore they have proven to be reliable classification tools, e.g. winning the ImageNet Large Scale Visual Recognition Challenge \cite{imagenet_cvpr09} annually since 2012. Therefore they find heavy use in other fields of science and engineering, like in medicine for the automatic classification of skin cancer \cite{CNNSkinCancer}. \\

%In other fields of science and engineering it is commonplace to use convolutional neural networks (CNNs) for the task of object classification in images, e.g. the automatic classification of skin cancer by Esteva et al. \cite{CNNSkinCancer}. The advantage of CNNs is that they do not need preprocessing of inputs and hand engineered features. On common classification challenges like the ImageNet Large Scale Visual Recognition Challenge \cite{imagenet_cvpr09}, conducted annually since 2010, since 2012 CNNs take the first place each year. Showing that they are reliable classification tools. \\

Dual-phase steel was used as a sample material in this work, firstly because it is one of the most widely used advanced high-strength steels and secondly due to the fact that damage sites right after nucleation are clearly distinguishable on the surface of the material. The deformation behavior was investigated by applying uniaxial stress to the material and recording surface micrographs using a scanning electron microscope (SEM). Even though different methods exist that gain a deeper insight into the material than surface micrographs, e.g. electron backscatter diffraction, these methods are limited by the time required to record the necessary data. \\

The goal of this thesis is to automate the process of localizing and classifying damage sites in surface micrographs of dual-phase steels under stress, using a combination of machine learning algorithms and CNNs. This tool would enable the automated collection of statistically relevant data, for determining the predominant damage mechanism. Furthermore making it possible to track the evolution of a large number of damage sites, resulting in a deeper understanding of the underlying mechanisms. Ultimately this algorithm can be easily applied to the classification of damage sites in different materials, that appear as voids in the microstructure. \\

%Furthermore making it possible to aid the tracking of the evolution of damage sites, thereby 
%
% and also the automated tracking of the evolution of damage sites, gaining a deeper understanding into the underlying mechanisms, while also being easily applicable to classifying damage sites in materials, that show themselves as voids in the microstructure. \\

%The goal of this thesis is to automate the process of localizing and classifying damage sites in ex-situ and in-situ experiments using a combination of machine learning algorithms and CNNs. As a sample material dual-phase steel is used, since the damage mechanisms right after nucleation are clearly distinguishable from each other. But ultimately the created tool should be applicable to other materials as well. Due to the time intensity of data collection, one of the main challenges is the collection of a suitable dataset for the training of the CNNs. \\


In the following ....
%\chapter{Dual-Phase Steel}

Dual-phase steels are a class of advanced high-strength steels. On a microstructural level dual-phase steels consist of a ductile ferrite matrix with islands of brittle martensite. It furthermore can contain austenite, pearlite, bainite, carbides and acicular ferrites. From an engineering point of view it exhibits multiple advantageous properties besides its rather straightforward thermodynamical processing. Dual-phase steels combine a high ultimate tensile strength, caused by the brittle martensite, a low initial yielding stress, enabled by the ductile ferrite, high fracture strain, and a high hardening ratio. Combined with their low weight they are of particular interest in the automotive industry, where they are used for steel sheets. \\
%Dual-phase steels are a class of advanced high-strength steels. On a microstructural level they consist of a ductile ferrite matrix with islands of brittle martensite. Furthermore they can contain, austenite remaining from the processing of the material, pearlite, bainite carbides, and acicular ferrite. The hard martensite is responsible for the materials high ultimate strength, while the ductile ferrite results in a low initial yield strength. Those advantageous properties together with a high fracture strain, high hardening ratio, and their light weight makes them interesting from a engineering point of view, especially in the automotive industry, where dual-phase steels are used for steel sheets. \\

While the microstructure of dual-phase steels is seemingly simple, its influence on the mechanical properties of dual-phase steels is only partially understood and research is ongoing. This is due to the fact that multiple microstructural parameters, like the martensite volume ratio, the size of martensite islands, the morphology, etc. play an important role for the macroscopic behavior. The microstructure at the surface of a sample is shown in figure \ref{fig:DPMicrostructure}. \\

\begin{figure}
\centering
  \includegraphics[width=\linewidth]{DPSteelMicrostructure.png}
  \caption{Dual-phase steel SEM micrograph}
  \label{fig:DPMicrostructure}
\end{figure}

\section{Damage Mechanisms}

While under tensile stress, the material exhibits irreversible deformation at stresses above the yield point. A stress-strain curve is shown in figure \ref{fig:DPStressStrain}. Beyond this point voids can nucleate in the microstructure. These voids are classified based on their relative position to the constituents of the dual-phase steel. The classes are as follows
\begin{itemize}[label={}]
\item \textbf{Martensite Cracking}: A crack nucleating inside a martensite island.
\item \textbf{Interface Decohesion}: Nucleation of a void at the interface between a martensite island and a ferrite grain.
\item \textbf{Boundary Decohesion}: Nucleation of a void at the interface between two adjacent ferrite grains.
\item \textbf{Notch Effect}: A void nucleating between the tips of two martensite islands. 
\end{itemize}
Additionally influencing the materials behavior are
\begin{itemize}[label={}]
\item \textbf{Inclusions}: A foreign particle remaining from the processing of the material. 
\end{itemize}
The different possible damage mechanisms are shown in figure \ref{fig:DamageCategories_abstraction}. \\
Due to the high concentration of martensite in the dual-phase steel used in this study, boundary decohesions occur rarely or not at all. In in-situ experiments, most of the sites classified at a certain stage, without knowledge of the previous state, as boundary decohesions turned out to be evolved notch effect sites. Examples for each damage category can be seen in figure \ref{fig:DamageCategories}.\\

\begin{figure}
\begin{subfigure}{.2\textwidth}
\centering
  \includegraphics[width=.8\linewidth]{MartensiteCracking_abstraction.pdf}
  \caption{MC}
  \label{fig:MC}
\end{subfigure}%
\begin{subfigure}{.2\textwidth}
\centering
  \includegraphics[width=.8\linewidth]{InterfaceDecohesion_abstraction.pdf}
  \caption{ID}
  \label{fig:Interface_scalebar}
\end{subfigure}%
\centering
\begin{subfigure}{.2\textwidth}
\centering
  \includegraphics[width=.8\linewidth]{BoundaryDecohesion_abstraction.pdf}
  \caption{BD}
  \label{fig:Inclusion_scalebar}
\end{subfigure}%
\begin{subfigure}{.2\textwidth}
\centering
  \includegraphics[width=.8\linewidth]{NotchEffect_abstraction.pdf}
  \caption{NE}
  \label{fig:Notch_scalebar}
\end{subfigure}%
\centering
\begin{subfigure}{.2\textwidth}
\centering
  \includegraphics[width=.8\linewidth]{Inclusion_abstraction.pdf}
  \caption{Inclusion}
  \label{fig:Inclusion_scalebar}
\end{subfigure}%
\caption{Abstracted damage mechanism for martensite cracking (MC) in subfigure (a), interface decohesion (ID) in subfigure (b), boundary decohesion (BD) in subfigure (c), notch effect (NE) in subfigure (d), and inclusions in subfigure (e). The martensite islands, the ferrite matrix, and the foreign body are labeled as M, F, and I respectively.}
\label{fig:DamageCategories_abstraction}
\end{figure}

\section{Experimental Setup}

\textbf{Preparation} \\
Before the experiments were performed, the surface of the dual-phase steel was prepared in order to be able to distinguish the martensite islands from the ferrite matrix, in surface micrographs. The preparation process consists of grinding the surface with up to $4000$ grit sandpaper, polishing it with oxide polishing suspension in steps of $6\mu m$, $3\mu m$, and $1\mu m$, and finally etching it with $1 \%$ nital. \\

\noindent \textbf{Probe Geometry} \\
The probe geometry was chosen in such a way that the center of the probe experiences almost homogeneous stress. It can be seen in figure \ref{fig:ProbeGeometry}. \\

\noindent \textbf{Tensile Tests}\\
Tensile tests were performed with engineering stresses ranging from $4$ to $8$. \\

\noindent \textbf{SEM Specifications} \\
The specifications and settings of the SEM can be seen in table \ref{tab:SEM}.\\

\begin{table}[H]
 \begin{center}
  \begin{tabular}{@{} *2l @{}} \toprule[2pt]
   Model & Zeiss LEO 1530 \\\midrule
   Horizontal Field Width & $100\mu m$   \\ 
   Vertical Field Width  & $75\mu m$ \\ 
   Resolution  & $3072\times 2304$ \\
   Detector Type & Secondary Electrons \\
   Electron Source & Field-Emitter Cathode \\ \bottomrule[2pt]

   \label{tab:SEM}
  \end{tabular}
 \end{center}
 \caption{Details of the SEM used for the surface micrographs.}
\end{table}

%
\chapter{Machine Learning for Image Analysis} % Main chapter title

\label{CNN} % For referencing the chapter elsewhere, use \ref{Chapter1} 

The task of classification of objects in images is a highly non-trivial problem, as the information about the object is encoded in a digitized form, where both the values and their relative position matter. As no straightforward way for object classification exists, intermediate steps have to be taken, firstly extracting relevant features from the image, that a classifier can use in order to identify the object in the image. Convolutional neural networks are todays state of the art algorithms for the classification of images. They differ from classical approaches to computer vision, as they incorporate selection of features into the learning process. The complexity of these algorithms requires high computational resources, that have been made available for realistic problems only in the recent decade, in part through the demand for ever more powerful personal computer and the usage of tensor calculation on the graphics processing unit. \\

%The original idea for CNNs was inspired by the groundbreaking work of Hubel and Wiesel in 1958 \cite{Hubel1959}, recognizing the orientation dependence of perceptive fields and the hierarchical structure of the visual cortex of cats. In a similar manner, CNNs learn and recognize objects through a hierarchical representation and make use of the idea of the orientation dependent perceptive field. \\

%Within the machine learning approach to image analysis, two general methods exist. Firstly one can approach the problem of object classification by hand engineering features that the researcher deems to be relevant and use a machine learning algorithm learn how to distinguish between the object classes using the extracted features. This approach comes with the downside that the hand engineered features may exclude relevant features and include irrelevant features leading to a poor performance. Furthermore an algorithm constructed in such a manner will not generalize well to new problems, as new features have to be hand engineered for the problem at hand. In the past this approach was heavily used, as it is less resource intensive. Today the problem of object classification in images is tackled by including the determination of relevant features into the machine learning task. This creates an algorithm that only receives an image without the need of preprocessing and feature extraction and trains both the relevant features and the classifier simultaneously. Due to the necessary large computational resources needed for this approach it has only recently been applied to major problems, as the available computational power has increased tremendously and graphic processing units (GPUs) have been used for tensor calculations.

% The "traditional" approach is based on feature engineering. At first relevant features are selected and an extraction of these features is hand engineered. Using these features a classifier is then trained and learns depending on the features given to which class the object in the image belongs. While this approach works well for certain problems, it lacks the capability of being easily applicable to new problems, since it might be necessary to engineer new features. Furthermore for complex problems the selection and engineering of features becomes increasingly difficult. Today the state of the art is to include the selection of features into the training. A classifier both learns what features are relevant and at the same time learns, depending on the extracted features, the mapping of the object shown in image to its corresponding class. \\

%Even though this approach exists since 1989, its wide range application was hindered by the necessary computational power for the training such a classifier. Due to the increase in computational power, due to Moores law, and the use of graphic processing units (GPUs) for tensor arithmetic the use of convolutional networks was made possible.

%\section{Introduction}
%
%Artificial neural networks (ANNs) are generally able to approximate any smooth function, as proven by Hornik et al. in 1989 \cite{Hornik1989}. In many cases the underlying function of a problem is unknown and the function can only be probed at certain points. ANNs can be trained on those samples and are particularly capable of finding the underlying non-linearities. Crucial for the performance of an ANN is firstly the collection of a representative training dataset. Once a sufficiently large dataset is available the selection of an appropriate architecture becomes increasingly important. \\
%One crucial step in finding an ANN performing well for this task is the selection of an architecture and a representative training dataset. \\

%In the following the basic building blocs of ANNs, artificial neurons, will be introduced in section \ref{sec:Neurons}. Then the basic structure of feedforward networks will be introduced in \ref{sec:feedforwardNetworks}. Afterwards convolutional neural networks will be motivated by linear image filter \ref{sec:DiscreteConvolutions} and then constructed from fully connected feedforward networks in section \ref{sec:CNN}


\section{Neurons}
\label{sec:Neurons}
The basic building block of every artificial neural network (ANN) is the neuron. For real valued inputs it maps $n: \mathbb{R}^n \rightarrow \mathbb{R}$ given internal parameters $w \in \mathbb{R}^d$, the weights, and $b \in \mathbb{R}$, the bias.
\begin{equation}
x \mapsto g(w \cdot x + b)
\end{equation}
with some suitable activation function $g$. A single neuron acts as a binary classifier, introducing a decision hypersurface with the normal vector given by the weight vector and its offset given by the bias in the opposite direction of the weight vector. An example of such a hypersurface is shown for a two dimensional problem in figure \ref{fig:binaryclassifier}. \\

\begin{figure}[H]
\centering
  \includegraphics[width=0.4\linewidth]{binaryclassifier.pdf}
  \caption{An example of a binary classifier in two dimensions. The hypersurface, in this case a line, distinguishes between elements belonging to class $0$ and $1$.}
  \label{fig:binaryclassifier}
\end{figure}

In figures \ref{fig:Neuron}  and \ref{fig:ArtNeuron} the resemblence between the biological neuron and its artificial counterpart is is illustrated. The biological neuron receives signals from adjacent neurons and fires if the internal potential exceeds its threshold, similarly to the action of the artificial neuron.
% While this would be more accurately represented using a step function as the non-linear activation function, which was historically the first to be implemented by Rosenblatt \cite{Rosenblatt1957}, problems arise, e.g. the lacking of a training algorithm for neurons arranged in layers. 
The choice of an activation function is a free parameter in the design of the architecture. Today in most deep architectures the most common choice for activation functions are rectifying linear units (ReLU) and derivatives thereof like the exponential linear unit (ELU) \cite{Clevert2015} or the leaky rectified linear unit (leakyReLU) \cite{Maas2013}.

\begin{figure}[H]
\centering
  \includegraphics[width=0.9\linewidth]{brain-2022398.pdf}
  \caption{An abstracted biological neuron.}
  \label{fig:Neuron}
  \includegraphics[width=0.9\linewidth]{Neuron_2.pdf}
  \caption{A single artificial neuron with $6$ inputs.}
  \label{fig:ArtNeuron}
\end{figure}


%%The basic building block of each artificial neural network is the neuron. Given an input vector $x$ it performs a weighted sum and adds a bias
%%\begin{equation}
%%u_i = W_i \cdot x + b_i 
%%\end{equation}
%%afterwards applies a non-linear activation function
%%\begin{equation}
%%z_i = g(u_i)
%%\end{equation}
%%This closely resembles the inner workings of a biological neuron. It receives signals from adjacent neurons, the weights correspond to the dendrites connecting the neurons. If the potential inside the neuron exceeds a certain potential, here represented by a bias, the neuron fires. The firing of the neuron in the artificial case is performed by applying the activation function. While the closest analogy to a biological neuron would be by using a step function, this approach was discarded, mostly due to the lacking of an automated training algorithm for neurons arranged in a layered fashion. Today one of the most common activation function is the rectifying linear units and derivatives thereof.

\section{Feedforward Neural Networks}
\label{sec:feedworwardNUsing a threshold function for $g$, points are given a hard label, either belonging to class $0$ or $1$, depending on their relative position to the decision hypersurface.etworks}

\subsection{Possibly leave out}
As can be seen in figure \ref{fig:binaryclassifier}, single neurons are capable of distinguishing between two classes that are linearly separable. When more classes exist or the classes are not linearly separable more neurons and more complex arrangements are necessary. In feedforward neural networks the neurons are arranged in a layered fashion were each layer consists of multiple neurons. With this layered structure problems can be solved that are impossible for a single neuron like the XOR problem. This problem can be solved by using a layered structure with two neurons in its first layer and a single neuron in its single layer assigning a label, as can be seen in figure \ref{fig:XOR}. \\

\begin{figure}
\centering
\includegraphics[width=0.8\textwidth]{XOR.pdf}
\caption{The transformation of the XOR function from the input space to the space of the output of the first two neurons, indicated by the two lines. The classification problem of the XOR function can be then solved by a single neuron in this transformed space.}
\label{fig:XOR}
\end{figure}

%For more complicated feedforward neural networks, the action of the network can be best described using tensor notation
%
%In feedforward neural networks neurons are arranged in a layered fashion. Each layer processes the output of its preceding layer, where the first layer constitutes the input to the network, see figure \ref{fig:ANN}. 

Overall the network acts as a function $N: \mathbb{R}^n \rightarrow \mathbb{R}^m$ with internal parameters $\theta$, where the dimensionality of $\theta$ depends on the internal structure of the network. The action of a layer $l$ can best be described by using matrix notation. 
\begin{equation}
W^l = 
\begin{pmatrix}
w_{0,0}^l & w_{0,1}^l & \dots & w_{0,n_{l-1}-1}^l \\
w_{1,0}^l & w_{1,1}^l & \dots & w_{1,n_{l-1}-1}^l \\
\vdots & \vdots & \vdots & \vdots \\
w_{n_l-1,0}^l & w_{n_l-1,1}^l & \dots & w_{n_l-1,n_{l-1}-1}^l \\
\end{pmatrix}
\end{equation}
where $w_i^l$ is the weight vector of neuron $i$ in layer $l$, $n_l$ is the number of neurons in layer $l$ and $n_{l-1}$ is the number of neurons in its previous layer. $W^l$ is matrix of real valued numbers of dimension $n_l \times n_{l-1}$. By collecting the biases of all neurons in layer $l$ in a bias vector $b^l$, the mapping of layer $l$ can be expressed as%By now also collecting the biases of all neurons in layer $l$ in a bias vector $b_l$.  and letting the layer dependent activation function act elementwise the action of a layer on its input can be defined as
\begin{equation}
x^{l-1} \mapsto g_l(W^l x^{l-1} + b^l)
\end{equation}
where $g_l$ is the layer dependent activation function that is applied elementwise to the input vector. The number of layers and the number of neurons in each layer are free parameters in the design of the neural network architecture. \\
\begin{figure}[H]
\centering
  \includegraphics[width=0.8\linewidth]{SimpleNeuron-crop.pdf}
  \caption{An ANN with an input vector of $5$ elements, $3$ hidden layers, containing $6$ neurons each, outputting a single real number. Each edge corresponds to the weight connection between the connected neurons. }
  \label{fig:ANN}
\end{figure}

Once a network is constructed its weights are initialized using pseudo-random numbers, e.g. using either a normal or uniform distribution with different variances depending on the activation function used in the network, e.g. He initialization \cite{He2015} for ReLU functions and Xavier initialization \cite{Glorot2010} for sigmoid functions. After the weights have been initialized the training is performed using the backpropagation algorithm \cite{Rumelhart1986}. For the backpropagation algorithm, a cost function needs to be defined, e.g. the quadratic deviation of the desired output $(y-\hat{y})^2$, with $y$ being the desired output and $\hat{y}$ being the networks prediction. Since the networks prediction $\hat{y}$ depends on the weights and biases of the network through the processing of the input, the cost function depends on those internal parameters. The backpropagation algorithm calculates the contribution of each weight and bias to the cost function and adjusts the weights by employing a numerical optimizer, i.e. the algorithm searches for the global minimum of the cost function.  The simplest optimizer is the gradient descent algorithm adjusting the weights by correcting them in the direction of the steepest descend of the cost function hypersurface. Todays optimizers are still based on the gradient descent method, but also use additional information, from approximates of higher order derivatives and training in batches \cite{Duchi2010},\cite{Sharma2017}. \\

%Feedforward networks are commonly used to approximate the mapping from some object represented in an input space to its corresponding class. %Assuming such a mapping exists these networks are generally capable of approximating the mapping, as mentioned before. %While this mapping can be realized using only one hidden layer with a large number of neurons, this approach is infeasible since each possible object in the input space has to be used in order to train the network.
% Choosing a topology for a neural network becomes crucial if the input space itself has in intrinsic topology. E.g. the information of an object represented in an image is not only represented by the pixel values but also their position inside the image.\\

%\subsection{Weight Initialization}
%The weights inside the network are usually initialized randomly. Choosing the right distribution and parameters is crucial for rate at which the network learns. For deep neural networks using sigmoid activation functions Xavier initialization is the most efficient while for ReLUs He initialization has proven to be the most efficient. The distribution from which the weights are chosen are the uniform distribution or the normal distribution. The details for the parameter initialization can be seen in table \ref{tab:initialization}
%
%\begin{table}[H]
% \begin{center}
%  \begin{tabular}{@{} *2l @{}} \toprule[2pt]
%   Xavier Initialization\\\midrule
%   Weights & Accuracy \\
%   Bias & $85 \%$   \\ 
%   Xavier Initialization\\\midrule
%   Weights & Accuracy \\
%   Bias & $85 \%$   \\ 
%  \end{tabular}
% \end{center}
% \caption{Agreement for the classification of damage sites by hand. }
% \label{tab:Reliability}
%\end{table}

%\subsection{Training}
%At first an error function $C$ has to be defined, that satisfies $C>0$ and is minimal only when the network returns the desired output. By doing so a measure is introduced for how far the prediction is from the truth. $C$ depends on all weights and biases in the network. By using backpropagation it is possible to calculate the contribution of each weight and bias to the prediction. 
%\section{Cost Functions}
%\begin{itemize}
%\item How a human learns - punishment?
%\item Define what is right and what is wrong
%\item Mathematical foundation to find optimal weights
%\item Properties of cost functions:
%\subitem $C>0$
%\subitem Output of network close to desired output then cost function close to minimum
%\end{itemize}
%\subsection{Squared Error}
%\begin{itemize}
%\item Linear regression
%\item Classical error function
%\end{itemize}
%\subsection{Categorical Cross Entropy}
%% http://neuralnetworksanddeeplearning.com/chap3.html#the_cross-entropy_cost_function
%\begin{itemize}
%\item Problems with squared error: small gradient
%\item Definition of cross entropy
%\item Using definition of sigmoid function shows that the system learns faster the further it is away from the true solution
%\end{itemize}

One approach to construct a topology in a network is to connect each neuron in one layer with every neuron in its preceding layer, a so called fully connected neural network, which can be seen in figure \ref{fig:ANN}. These kinds of networks are used for classification or regression problems depending on parameters living in separate spaces, e.g. the prediction of the price of a house depending on its numbers of rooms, the number of square meters, etc. Applying this network to the classification of images comes with two major downsides. Firstly the number of trainable parameters grows quickly as the size of the image and the size of the network grows, e.g. classifying an image of size $256\times 256$ with $100$ neurons in the networks first layer would already lead to $6553600$ trainable parameters. Secondly such a network treats all inputs as belonging to independent axis, disregarding the topology of the input space.
%Secondly such a network does not regard the topology of the input space. %In the next section a motivation for a special kind of topology, a convolutional neural network, will be motivated using practices for image processing using discrete convolutions.

\section{CNN}\label{sec:DiscreteConvolutions}
Image recognition is easy for humans, due to our evolution and sensory development but hard for computer algorithms. Therefore it is safe to assume that there exists an underlying function, mapping from the input space to the class space. As ANNs are generally capable of approximating any smooth funciton \cite{Hornik1989}, an ANN should therefore be capable of performing image analysis efficiently, given sufficient relevant training data and a suitable architecture. The basis for a ANNs with a special internal structure so called convolutional neural networks are convolutions that will be explained in the following section. \\

%ANN should therefore be able to perform image analysis efficiently, given sufficient relevant training data and a suitable architecture. \\

% By now adjusting the architecture of our network we hope that the constructed network will be more effective at approximating this function. \\


%While inspiration was originally taken from the inner workings of the visual cortex of mammals, based on the pioneering work D. Hubel and T. Wiesel \cite{Hubel1959}, in this work CNNs will be motivated by techniques used in image processing, specifically discrete convolutions. \\
%An image is represented as an array with shape $M\times N\times C$ with $M$, height $N$ and channels $C$. In an RGB image the number of channels equals $3$ representing the different colours, while in a black and white only one channel is present. Each element takes a value between $0$ and $255$ representing the intensity at its position inside the image. \\
\subsection{Convolutions}
A convolution is a linear operation acting on two functions $f$ and $g$, resulting in a new function. Commonly one of the functions is called the convolution kernel, say $g$, and the resulting function is a modification of the original function $f$. The operation is given by
\begin{equation} \label{eq:convolutionContinuous}
(f*g)(x) = \int_{\mathbb{R}^n} f(x)g(x-y)dy
\end{equation}
where $f$ and $g$ act on $\mathbb{R}^n$.  \\

When working with digitalized data, equation \ref{eq:convolutionContinuous} needs to be adjusted for functions acting on the discrete space $\mathbb{Z}^n$, resulting in
\begin{equation}\label{eq:convolutionDiscrete}
(f*g)(i_1,\dots ,i_n) = \sum_{j_1} \cdots \sum_{j_n} f(i_1,\dots ,i_n) g(i_1-j_1,\dots ,i_n-j_n)
\end{equation}
$g$ is usually restricted to have non-zero values only in a window of a certain size. For a two-dimensional object an example is shown in figure \ref{fig:Convolution}. \\

\begin{figure}[H]
\centering
\includegraphics[width=\linewidth]{DiscreteConvolution.pdf}
\caption{Discrete convolution for a kernel of size $3\times 3$ (gray) and an input of size $5\times 5$ (blue). The output is shown in green. Taken from \cite{RajaKishor2016}.}
%\caption{Discrete convolution for a kernel of size $3\times 3$ (gray) and an input of size $5\times 5$ (blue), with stride $s=1$ and no padding. The output is shown in green. Taken from \cite{RajaKishor2016}.}
\label{fig:Convolution}
\end{figure}

A digitalized image is represented internally as pixel values in the shape of an array $I_{m,n,c}$ with $m=0,\dots ,M-1$ and $n=0,\dots ,N-1$ being the spatial dimensions of the image for an image of size $M\times N$. The additional index $c$ is used in order to represent colour. An image encoded in the RGB colour space will have three channels, where each channel corresponds to the contribution of each colour, red, green, and blue. \\

In image processing discrete convolutions are used in order to find or amplify features in images. One example is the Sobel operator, which is used for the detection of edges in images. It approximates the gradient of the pixel values of an image. Assuming there is an underlying continuous function and an image $I$ is its discretization, the gradient in horizontal direction can be approximated by $G_x$ corresponding to edges in vertical direction and the derivative in vertical direction by $G_y$ corresponding to edges in horizontal direction. $G_x$ and $G_y$ are given by
\begin{align}
  \begin{split}
G_x =
\begin{pmatrix}
+1 & 0 & -1 \\
+2 & 0 & -2 \\
+1 & 0 & -1 \\
\end{pmatrix}
\end{split}
\begin{split}
G_y = 
\begin{pmatrix}
+1 & +2 & +1 \\
0 & 0 & 0 \\
-1 & -2 & -1
\end{pmatrix}
\end{split}
\end{align}
The overall gradient can then be calculated by
\begin{equation}
G = \sqrt{G_x^2+G_y^2}
\end{equation}
in the sense that $G_x$ and $G_y$ are used to convolve the original image, the resulting elements are squared, added, and the square root is applied elementwise. A sample application can be seen in figure \ref{fig:Sobel}. By doing so the relevant feature, the position of the edges, is extracted from the original image. \\
%In the next section adjustments to ANNs will be described in order to get a topology, that reflects the extraction of features in images using convolutions.
\begin{figure}
\centering
\begin{subfigure}{.5\textwidth}
  \centering
  \includegraphics[width=\linewidth]{Bike.png}
  \caption{Original image}
  \label{fig:sub1}
\end{subfigure}%
\begin{subfigure}{.5\textwidth}
  \centering
  \includegraphics[width=\linewidth]{Bike_Sobel.png}
  \caption{Edge amplified image}
  \label{fig:sub2}
\end{subfigure}
\caption{Edge amplification}
\label{fig:Sobel}
\end{figure}

\section{Incorporation of Convolutions into ANNs}
\label{sec:CNN}

As shown for the Sobel operator, it is possible to extract features from an image using an appropriate convolution kernel. Discrete convolutions take the spatial informations of their input into account. The idea of a convolutional neural network is to employ multiple discrete convolutions in them in order to extract the features that are relevant for the classification task at hand. The selection of the relevant features is performed by the network itself, as the weights of the convolution kernel are adjustable through backpropagation. In the end of a convolutional neural network, the subsequent extraction of features results in a feature vector that is then used by a fully connected neural network for classification.\\

Starting from a fully connected neural network the action of a discrete convolution can be replicated by rearranging the connections inside the neural network. \\ %At first this transformation will be described for one layer of the network extracting only one feature for an input with only spatial dimensions. The generalization to an image with multiple channels and a layer extracting more than one feature then becomes straightforward. Having the action of one such convolutional layer a convolutional neural network can constructed. \\


% Moreover, a network is used for the purpose of the best amplification method theory adviser controller overview. A heuristic approach to the mining of displayed pixels is achieved through the usage of tree based paper trails, using an coffee implementation. Therefore, the supply chain has insignificant variations which lead to unexpected unobservable universial changing events. However, determining the structure of a theory with underlying hidden droplets is not feasible in humankinds lifespan. The juggling of pods provides the known solution to the structure and comparing the theory among strange thoughts however beetles survive. From the perspective of a researcher active in the field of biosophy the advantegous properties of watersolutions in mud have no significant contribution to the appetite of a great ape. Apart from human neural networks, different kinds slowly evolve and master pieces of operations as long the labor force concentrates heavily on quantums. As a conclusion the convolution resolves the evolution of the solution of any ution to provide the answer. Having found the answer all that remains is finding the corresponding question. 

At first the transformation of an ANN to incorporate discrete convolutions will be explained for images having only one colour channel, corresponding to grayscale images, extracting only a single feature. The channel index will therefore be ommitted. In order to preserve the topology of the input, its shape is retained taking the form of a tensor $I_{m,n}$ with $m=0,\dots ,M-1$ and $n=0,\dots ,N-1$. The weights of neuron $i$ then also take the form $W^i_{m,n}$, where each neuron can still be connected to each neuron from its input.
%In order to preserve the topology of the image the input now is an tensor $I_{m,n}$ with $m=0,\dots ,M-1$ and $n=0,\dots ,N-1$. The connections of a neuron therefore also becomes tensor $W_{m,n}^i$ with $m=0,\dots ,M-1$, $n=0,\dots ,N-1$, and $i$ being the index of the neuron. The bias of neuron $i$ is $b^i$. The output of neuron $i$ then is

\begin{equation}
v^i = g\left( \sum_{m=0}^{M-1} \sum_{n=0}^{N-1} I_{m,n} W_{m,n}^i + b^i \right)
\end{equation} \\
with $b^i$ being the bias of neuron $i$. By then restricting the receptive field of neuron $i=0$ to a window of size $M'\times N'$ anchored at $i'=0,j'=0$ its weight matrix takes the form

\begin{align}
\begin{split}
W^0 = 
\begin{pmatrix}
\tilde{W}^0 & \boldsymbol{0} \\
\boldsymbol{0} & \boldsymbol{0} \\
\end{pmatrix}
\end{split}
\begin{split}
\tilde{W}^0 = 
\begin{pmatrix}
\tilde{W}_{0,0}^0 & \tilde{W}_{0,1}^0 & \dots & \tilde{W}_{0,N'-1}^0 \\
\tilde{W}_{1,0}^0 & \tilde{W}_{1,1}^0 & \dots & \tilde{W}_{1,N'-1}^0 \\
\vdots & \vdots & \ddots & \vdots \\
\tilde{W}_{M'-1,0}^0 & \tilde{W}_{M'-1,1}^0 & \dots & \tilde{W}_{M'-1,N'-1}^0 \\
\end{pmatrix}
\end{split}
\end{align}
Applying this weight matrix to the input would result in a value depending only on values in the input inside of the receptive field of the weight matrix, namely the coordinates at which $W^0$ is non-zero. The output of this multiplication is identical to the output of a discrete convolution at the coordinates $0,0$ with $\tilde{W}^0$ being the convolution kernel. By now creating copies of this neuron, having the same weights but anchored at different positions, such that the entire image is covered the action of a discrete convolution is recreated, as each neuron performs an elementwise multiplication with the convolution kernel $\tilde{W}^0$ but anchored at a different coordinate. A one dimensional example can be found in \ref{fig:CNNTransformation}.\\

\begin{figure}[H]
\centering
\includegraphics[width=0.7\textwidth]{CNNTransformation.pdf}
\caption{The transformation of a fully connected network, with only one neuron. In the second graph the first neuron is only connected to a section of the input. In the last image the first neuron is copied covering the entire input. By now going from left to right indicated by the arrow, the one dimensional version of \ref{fig:Convolution} is replicated.}
\label{fig:CNNTransformation}
\end{figure}

Generalizing the input to also incorporate multiple channels, i.e. $I_{m,n,c}$, the weight tensors of all neurons also take the form $W^i_{m,n,c}$. The output of this layer now takes the form of equation \ref{eq:convolutionDiscrete}. At this point the extraction of one feature has been explained. By applying multiple such convolution operations, where each convolution is independent from all other convolutions, multiple features can be extracted in each layer. Through backpropagation such a layer is then capable to learn which features are relevant for the classification problem at hand. \\

Such a network topology overcomes the shortcomings of a fully connected network. Firstly the number of trainable parameters is significantly reduced as the weight matrix of each neuron is only sparsely occupied and the weights between neurons in each feature map are restricted to the same value. Furthermore, the internal structure of the input space is taken into account in each layer through the usage of convolutions. \\

Using discrete convolutions comes with new hyperparameters to choose. These are listed in the following.

\subsubsection{Kernel Size}
The size of the convolution window $M' \times N' \times c'$ is one of the most important hyperparameters, as it determines the perceptive window of each neuron. This in turn determines at what scales features live in. It is possible that for a feature in a layer only the directly adjacent values are important. For the extreme case where the convolution window is of size $1\times 1 \times c$ where c is the number of channels of the input, a layer would only transform an entire image independently of the spatial relations of values in the input. Most commonly $M'=N'$ and $c'=c$ are chosen and very rarely deviate from this rule. Popular choices for the kernel size are $2,3,5,7$ where it has been argued that kernel sizes larger than $3$ should be replaced by subsequent kernels of size $3$ \cite{Szegedy2015}. 

%The size of the convolution window $M' \times N' \times c'$ is one of the most important hyperparameters. Most commonly $M'=N'$ and $c'=c$ are chosen and very rarely deviate from this rule. The size of the convolution window determines the scale at which features live in each layer. For example a convolution window of size $1\times 1 \times c$, the spatial relation between the pixels is not considered and each pixel of the input is transformed in the same manner. A layer employing a convolution kernel of this size is called a bottleneck layer \cite{Lin2013} and is often times used in order to reduce the dimensionality inside a network. Popular choices for the kernel size are $2,3,5,7$ where it has been argued that kernel sizes larger than $3$ should be replaced by subsequent kernels of size $3$ \cite{Szegedy2015}. 

%One of the most important hyperparameters is the size of the convolution window $M' \times N' \times c'$. For the spatial dimensions $M' = N'$ is chosen, while the depth of the kernel is equal to the depth of the input. In a CNN layers can have different values of $N'$. Commonly these have the values of $1$ for bottleneck layers reducing the dimensionality inside the CNN \cite{Lin2013}, $2,3,5,7$ where it has been argued that kernel sizes larger than $3$ should be replaced by subsequent kernels of size $3$ \cite{Szegedy2015}. 
 
\subsubsection{Number of Convolutional Kernels}
The number of convotuinal kernels determine the number of features to be extracted in each layer. Together with the kernel size, this determines the number of trainable parameters in each layer. \\

As an image is processed by the network its size usually decreases. This is often compensated by the number of convolutional kernels. The information about the specific position of the object in the original image is therefore transformed into information about the object itself. \\

Contrary to this general trend of increasing number of convolutional kernels, sometimes in between layers bottleneck layers are introduced. These layers have a kernel size of $1\times 1$ and less convolutional kernels than its preceding layer. This is used in order to reduce dimensionality before further processing \cite{Lin2013}.

% Furthermore the extreme case mentioned before of a kernel size of $1\times 1 \times c$ is sometimes used as well in so called bottleneck layers , in order to reduce the dimensionality inside a network. 
%
%The number of convolution kernels determines the number of features to be extracted in each layer. It together with the kernel size and the number of layers it determines the number of trainable parameters in the network. \\
%
%It is possible and commonly used to have an increasing number of convolutional kernels in order to compensate for the fact that the image size in typical network decreases. 
%
%Furthermore given the output of a layer with $l$ features a bottleneck layer, as described before, can be used with $l'$ convolutional kernels in order to reduce the dimensionality inside of the network. 


\subsubsection{Padding}
As the kernel is moved across the image, it is possible that the convolution window falls outside the input image. The treatment of the boundary of the image is called padding. The main ways are to either ignore the boundary, leading to an output smaller in size $M\times N \mapsto M-M' \times N-N'$, to expand the input with zeros (zero padding), with the outermost values (same padding), or to introduce periodic boundary conditions at the border.

\subsubsection{Stride}
The increments with which the convolutional window is moved across the input is called the stride, $s_1$ and $s_2$ for each spatial dimension. Most commonly $s_1 = s_2 = s$ is chosen. Furthermore the stride is restricted by the size of the convolution kernel $s\leq N'$ such that the entire input is still covered. $s=1$ and $s=2$ are popular choices for the stride.

\section{Additional Layers}
Besides convolutional layers CNNs also consist of other layers. These will be described in the following.

\subsubsection{Pooling}
As a network grows, adding multiple layers with multiple channels per layer, the required memory and the number of operations increases alongside. In order to reduce memory consumption and computation times in between layers pooling layers are added. These downsample the output of one layer for the following layer, thereby tackling these issues.\\

%As a network grows and the size of the input increases, the number of values to be tracked and need to be calculated increases, leading to a high consumption of memory and time. Pooling layers address this issue by downsampling the input in between layers in order to save memory. Furthermore the use of downsampling leads to a desired invariance to small translations as the exact position of an object is not important for the task of classification. \\

Pooling layers work similarly to a convolutional layers except that their is predetermined. Just like the convolutional layer pooling layers also have the hyperparameters window size $N^P_1 \times N^P_2$, and stride $s_1^P,s_2^P$, where again $N^P_1 = N^P_2 = N^P$ and $s_1^P=s_2^P=s^P$ are most commonly chosen. An input with the  dimensions $M \times N \times c$ is then mapped to  $M \mapsto \ceil{(M-N^P)/s^P} $ and $N \mapsto \ceil{(N-N^P)/s^P}$ while the channel dimension remains unchanged $c \mapsto c$.\\

The most popular choices for the pooling function are either to return the maximum value (max pooling) or the average of the values in the window (average pooling). An example of max pooling for an input of size $6\times 4$ and $N^P = 2, s^P = 2$  can be seen in figure \ref{fig:MaxPooling}.\\

\begin{figure}[H]
\centering
\includegraphics[width=1\linewidth]{MaxPooling.pdf}
\caption{Max pooling}
\label{fig:MaxPooling}
\end{figure}


\subsubsection{Dropout}
Multiple approaches exist to prevent overfitting, e.g. introducing additional penalties to the error function, restricting the weights from growing indefinitely. One popular approach is "dropout". First introduced by Srivastava et. al. \cite{DropoutOriginal}, this approach selects a portion of neurons in each layer during training at random and removes them from the network temporarily. Dropping out units from a neural network corresponds to creating a new neural network that shares the existing weights from the original network. After units have been dropped out the network is trained. This process is repeated and neurons are selected again by random. The weights of the final trained network are then averages from the weights during training. Effectively, this corresponds to training $N$ similar but not identical networks at the same time. The final prediction is the calculated from the average of this ensemble. \\ %For a neural network consisting of $n$ neurons there are $2^n$ such possible thinned networks, of which each realization will rarely if at all be trained. The trained weights will then be averaged resulting in a network of the original size. \\

\begin{figure}[H]
	\centering
	\includegraphics[width=0.8\linewidth]{dropout.jpeg}
	 \caption{Dropout Neural Net Model taken from \cite{DropoutOriginal}.}
	 %taken from:http://blog.christianperone.com/2015/08/convolutional-neural-networks-and-feature-extraction-with-python/
	 %ueberlegen andere grafik zu benutzen
 \label{DropoutDiagram}
\end{figure}

%In a standard neural network using backpropagation the error function is minimized by using the influence of each parameter, leading to neurons adapting to one another and possibly compensating errors made by those neurons. This co-adaption leads to overfitting since it does not generalize to to unseen data. By randomly dropping out units this effect is suppressed. 

%In their original paper Srivastava et. al. showed this by looking at the first level features of a neural network trained on the MNIST \cite{MNIST} with and without dropout.

\subsubsection{Batch Normalization}

Another, in modern architectures, often employed layer type is a batch normalization layer. In their paper \cite{Ioffe2015} the Ioffe et al. argue that the shift in the distribution of each layer during training leads to a diminished learning rate, called an internal covariance shift. In order to reduce this shift, the input $x_i$ of a batch $B$ is firstly normalized such that it has zero mean and a unity variance
\begin{equation}
\hat{x}_i = \frac{x_i-\mu_B}{\sqrt{\sigma_B^2+\epsilon}}
\end{equation}
where $\mu_B$ is the expectation value of the batch, $\sigma_B$ is the standard deviation of the batch, and $\epsilon$ is a regularization parameter preventing a division by zero. Since this normalization may change what a network is capable of learning, the possibility is left to rescale the input such that different regions of the activation function can be used,
\begin{equation}
y_i = \gamma \hat{x}_i + \beta
\end{equation}
where $\gamma$ and $\beta$ are trainable parameters. The authors reported that such a normalization is capable of increasing the accuracy, decrease the number of epochs necessary for training, and act as a regularization method.

\subsubsection{Softmax Layer}

As the input is transformed and features are extracted through the subsequent convolution and pooling operations, the spatial information is lost. The features are then flattened into a feature vector. As fully connected neural networks have proven as a powerful tool for the classification based on inputs having no direct relationship, the last layer of a CNN is a fully connected layer, using the extracted information in order to assign a class. \\

The output of this layer can take any value before an activation function is applied. However in order to be capable to interpret the output of the network in a probabilistic manner, it is desired that the output should fall into the interval $[0,1]$. An activation function that satisfies this condition is the softmax function defined by
\begin{equation}
\sigma(z)_j = \frac{e^{z_j}}{\sum_i e^{z_i}}
\end{equation}
Where $z_j$ is the output of neuron $j$ before the application of an activation function. 

%For classification tasks, the output of the layer should represent a probability or confidence of the network that the object processed belongs to class $j$. Therefore it should return normalized values in the interval $[0,1]$. For multi-class problems this is achieved by using the softmax function defined by

%After the application of subsequent convolution and pooling operations the resulting features are collected flattened into a feature vector. The last layer of the network then decides to which class the object belongs to. Desired properties of the output of the network are that if the object belongs to class $j$ the output at node $j$ should be close to one in order to have some form of probability or confidence of the network in its decision. Therefore the output should be in the interval $(0,1]$. One popular function exhibiting these properties is the softmax function defined by
%\begin{equation}
%\sigma(z)_j = \frac{e^{z_j}}{\sum_i e^{z_i}}
%\end{equation}

\section{Typical Architecture}

Generally the structure of a CNN is built up of consecutive convolutional layers with pooling layers, dropout, and batch normalization layers in between, with a fully connected network in the end assigning classes based on the extracted features. By denoting the convolutional layer as a combination of normalization and regularization
\begin{equation}
Conv = Conv \rightarrow Dropout \rightarrow BatchNorm
\end{equation}
the typical architecture of a CNN can be depicted as
\begin{equation*}
\begin{split}
Input \rightarrow
Conv \rightarrow {Conv} \rightarrow {Pool} \rightarrow {Conv} & \rightarrow \\
 \rightarrow {Conv} \rightarrow {Pool} \rightarrow \dots & \rightarrow {Fully \text{ } Connected} \rightarrow Output
\end{split}
\end{equation*}

Through the subsequent application of convolutional layers, each extracted feature is influenced by an increasing number of pixels in the input. The first layers will therefore extract low level features, while at the end of the network these features will be transformed into high level information. The information from the input, that influences the feature extracted by a later layer is illustrated in figure \ref{fig:featureHierarchy}. 

\begin{figure}[H]
\centering
\includegraphics[width=0.7\textwidth]{CNNHierarchie2.pdf}
\caption{The first two layers of an abstracted CNN for a one dimensional problem, with a kernel size of $3$ and no padding. As can be seen while both layers have the same kernel size, the feature map in the first layer is only dependent on only three input values, while the second is influenced, through the first layer by five input values.}
\label{fig:featureHierarchy}
\end{figure}

%
%\section{Convolutional Neural Networks - Architecture}
%
%Reflecting the topology of the input the neurons are now arranged in rectangular grid. 
%
%
%
%
%While at first each neuron is still connected to each input neuron, those connections are now restricted to a smaller window of size $m\times n$, in such a manner that the entire input is covered in a regular way. This can be seen in figure \ref{fig:CNN_topology}.
%Due to the special topology of images the neurons have to be rearranged as can be seen in figure \ref{fig:CNN_topology_notrestricted}. By restricting the connections of each neuron to a specific window of size $m\times n$, such that each neuron sees a different part of the original image and heavy weight sharing between all neurons, this layer of neurons performs a convolution on the original image. The result then is a feature map. In a CNN multiple such feature extractors are used in each layer. Additionally to the convolution operation, each to each output a bias is added and a non-linear activation function is applied.\\
%\begin{figure}
%\centering
%\begin{subfigure}{.5\textwidth}
%  \centering
%  \includegraphics[width=\linewidth]{CNN_topology.pdf}
%  \caption{Original image}
%  \label{fig:CNN_topology_notrestricted}
%\end{subfigure}%
%\begin{subfigure}{.5\textwidth}
%  \centering
%  \includegraphics[width=\linewidth]{CNN_topology_restricted.pdf}
%  \caption{Edge amplified image}
%  \label{fig:sub2}
%\end{subfigure}
%\caption{Edge amplification}
%\label{fig:Sobel}
%\end{figure}
%
%
%\begin{figure}
%\includegraphics[width=\linewidth]{CNN_topology_multiple_feature_maps.pdf}
%  \caption{Original image}
%  \label{fig:CNN_topology_notrestricted}
%\caption{Edge amplification}
%\label{fig:Sobel}
%\end{figure}
%
%\section{Different Layers Used In CNNs}
%While the convolutional layer is the most essential part of a network to be defined as a CNN, other layers are used in them that will be described in this section.
%
%\subsection{Pooling Layer}
%Usually a pooling layer follows a convolutional layer. A pooling layer takes, similarly to the convolutional layer, a rectangular as its input. It performs a predefined operation on this window, e.g. it returns the average or the maximum value inside the window. This results in a network that is shift invariant to minor local changes
%
%\subsection{Normalization Layer}
%Batch Normalization
%
\section{Performance Analysis}

In order to asses how well a classifier performs and generalizes to new data, the dataset available is split into a training dataset and a testing dataset. Once the classifier is trained, its predictions on the test dataset are investigated. \\

The testing dataset will be denoted as $\{I_i,y_i\}_{i=1,\dots ,N} $, with $I_i$ being a image with index $i$, $y_i$ being the correct class for image $i$, and $N$ being the size of the testing dataset, and the prediction of the network for $I_i$ will be denoted as $\hat{y}_i$. It is now possible to study the performance of the network in varying degrees of detail, listed in the following with decreasing levels of detail.

\begin{itemize}
\item \textbf{Inspection by hand}: With this method all available resources will be used by the researcher in order to study the networks performance, namely the image itself $I_i$, the correct prediction $y_i$, and the networks prediction $\hat{y}_i$. This can be done for the entire dataset, which becomes increasingly time intensive as the dataset grows, or by restricting the dataset to samples of cases were the prediction of the network differs from the true class label.
\item \textbf{Confusion matrix}: By using a confusion matrix the information of the image itself $I_i$ is disregarded and the predictions of the network and the true classes are collected in a matrix, in the form that each entry corresponds to the number of times a class $i$ was predicted and class $j$ was the true class.
\item \textbf{Metrics}: The information at hand can be reduced further by now collecting the predictions of the network and/or the true classes into one scalar, the metric. 
\end{itemize}

Commonly firstly the metrics are studied in order to have an first intuition whether the network performs well overall. If the network performs sub-optimal the confusion matrix can be used in order to determine difficult classes that then can be investigated by hand. In the following confusion matrices and different metrics will be discussed in detail.


\subsection{Confusion Matrix}

As mentioned above a confusion matrix is the collection of the prediction of the network and the true classes into a matrix. Its elements $c_{m,n}$ are defined as the number that class $m$ was predicted for an instance when class $n$ was true, i.e. it is given by
\begin{equation}
c_{m,n} = | \{y_i = m \quad \textrm{and} \quad \hat{y}_i = n \quad | \quad i=1,\dots,N  \} |
\end{equation}
where $|\cdot |$ denotes the size of a set. It is therefore a matrix of size $k\times k$, where $k$ is the number of classes, of integer values. The diagonal elements are the cases in which the classifiers prediction were correct and the off-diagonal elements are wrong predictions of the classifier. \\

The confusion matrix can be used in order to investigate if two classes are often confused with each other, since in that case the entries $c_{m,n}$ for some fixed classes $m$ and $n$ will be large in comparison to other off-diagonal elements. Two cases of confusion matrices are shown in \ref{fig:ConfusionMatrix}, with an example of an optimal case and an example for a realistic case. \\

\begin{figure}[H]
\centering
\includegraphics[width=\textwidth]{ConfusionMatrix.pdf}
\caption{An example of confusion matrices of an optimal classifier and of a realistic classifier.}
\label{fig:ConfusionMatrix}
\end{figure}

%\begin{minipage}{.5\linewidth}
%\begin{equation*}
%\begin{pmatrix}
%1000 & 0 & 0 & 0 \\
%0 & 1000 & 0 & 0 \\
%0 & 0 & 1000 & 0 \\
%0 & 0 & 0 & 1000 
%\end{pmatrix}
%\end{equation*}
%\end{minipage}
%\begin{minipage}{.5\linewidth}
%\begin{equation}
%\begin{pmatrix}
%898 & 13 & 41 & 20 \\
%68 & 959 & 16 & 17 \\
%23 & 17 & 937 & 47 \\
%11 & 11 & 6 & 916 
%\end{pmatrix}
%\end{equation}
%\label{eq:ConfusionMatrix}
%\end{minipage}


\subsection{Metrics}

From the confusion matrix, different metrics can be defined. These are listed in the following
\begin{itemize}[label={}]
\item \textbf{True positive rate}: The true positive rate is the ratio of samples correctly identified as belonging to class $m$ and the number of all samples belonging to this class, i.e.
\begin{equation}
TPR_m = \frac{c_{m,m}}{\sum_{n=0}^{k-1} c_{n,m}}
\end{equation}
\item \textbf{Positive predictive value}: The positive predictive value is the ratio of samples correctly identified as belonging to class $m$ and the number of all samples classified as belonging to class $m$, i.e. 
\begin{equation}
PPV_m = \frac{c_{m,m}}{\sum_{n=0}^{k-1} c_{m,n}}
\end{equation} 
The positive predictive value plays an important role for the usage of a classifier in order to generate statistical data, as the predictions statistical error is related to the positive predictive value.
\item \textbf{Accuracy}: The accuracy is defined as the ratio between all correctly identified samples and the number of overall samples, i.e.
%
% One of the most commonly used metrics for the assessment of a networks performance. It is defined by the number of correct predictions divided by the number of all predictions. Its values therefore fall in the interval $[0,1]$, where an accuracy of $1$ corresponds to a perfect classifier and a value of $0$ to a classifier never predicting the correct class. It can be defined from the confusion matrix as 
\begin{equation}
a = \frac{\sum_{m=0}^{k-1} c_{m,m}}{\sum_{m=0}^{k-1} \sum_{n=0}^{k-1} c_{m,n}}
\end{equation} 

\end{itemize}



\subsection{Purity vs Efficiency}
When a classifier does not achieve sufficient accuracy, e.g. when the error introduced by the classifier on its predictions is too high therefore preventing an interpretation of its results, it is possible to increase the number of correctly classified objects at the cost of a lower classification rate. This is performed by introducing a threshold $\theta$ and only classify if the confidence of the network, the outputted probability like value, exceeds it. The metrics discussed earlier can then be evaluated only in cases where a classification could be performed, the accuracy on this set will be called purity in the following. In figure \ref{fig:PURvsEFF} purity efficiency curves for a classifier working optimally, a classifier assigning classes at random, and a suboptimal classifier are shown. 

\begin{figure}[H]
\begin{center}
\includegraphics[width=\linewidth]{ACCvsEFF_example.pdf}

\end{center}
\caption{PurityEfficiency}
\label{fig:PURvsEFF}
\end{figure}

\subsection{LIME}
While investigating the predictions of a classifier by hand, the "reasoning" of the classifier for arriving at the assigned class is not directly accessible. In order to assist a researcher at understanding the classifiers prediction is the "Local Interpretable Model-agnostic Explanations (LIME)" tool, developed by Ribeiro et al. \cite{Ribeiro2016}. For image classification the region of interest, decisive for the resulting class prediction, is extracted from the image. As discussion exists even for researchers at arriving at the same damage category, a classifier thats prediction can be understood can therefore be preferable over a classifier scoring a higher accuracy, without interpretability. 
%\begin{figure}[H]
%\centering
%\includegraphics[width=\textwidth]{Lime.pdf}
%\end{figure}





\section{Clustering Algorithms}

For the localization of objects in images many sophisticated machine learning methods exist, like R-CNNs combining localization with classification. For the case of damage sites the defining feature for a void is its darkness compared to its surrounding. Candidates for damage sites can therefore easily be found by using a binarization of the image depending on the values of the pixels in the micrograph. These candidates have then to be clustered, in order to find the location and size of the void. A damage site can then be found by using only clusters of a certain size, thereby discarding noise and small voids. \\

As the number of damage sites in a SEM micrograph is previously unknown, the clustering algorithm should require minimum domain specific knowledge. Furthermore the damage sites are differently shaped, therefore clusters of different shapes should be recognized by the clustering algorithm. The density-based spatial clustering of applications with noise (DBSCAN) \cite{Ester:1996:DAD:3001460.3001507}, fullfills these requirements while also working efficiently on large datasets. 

\subsection{DBSCAN}
DBSCAN is a density based clustering algorithm, with two parameters. Firstly, $\epsilon$ defining the notion of a neighborhood, i.e. the neighbourhood of a point $p$ is defined by $N(p) = \{ q \in D | norm(p,q)<\epsilon\}$. Secondly, $M$ defining core points as points with at least $M$ points in their neighbourhood, i.e. a point $p$ is a core point if $|N(p)| \geq M|$. Clustering is now performed by grouping all points that are reachable to some core point, where two points $p,q$ are reachable if a path exists with $p_1=q,p_2,\dots,p_n=p$ with adjacent points being in their neighbourhood, i.e. $p_i\in N(p_{i+1})$. Every other point is considered to be noise. \\


%DBSCAN is a density based clustering algorithm, that takes two parameters. Firstly, the maximal distance $\epsilon$ at which two points can be separated while still being considered neighboring. Secondly, the minimum number of neighbors for a point to be considered a core point. The clustering algorithm then decides between three kinds of points
%\begin{itemize}
%\item \textbf{Core Points}: A point with at least $m$ points in its $\epsilon$ neighbourhood.
%\item 
%\end{itemize}
%It takes two parameters $\epsilon $, defining the maximum distance between two points to be considered neighbours, and min_samples, defining how many points a certain point needs to be neighboring in order to consider it a core point. All core points define a cluster, and all points directly connected to these core points are also considered to belonging to this cluster. All points not 

\begin{figure}[H]
\centering
\includegraphics[width=0.6\textwidth]{DBSCAN.pdf}
\caption{DBSCAN with $M=4$. Core points are coloured green, reachable points are coloured blue, and noise is coloured black.}
\end{figure}

%In order to localize objects in images many sophisticated machine learning methods exist. In the case of damage mechanisms the defining feature is the darkness of the voids. These can be found by applying a binarization of the original micrograph, finding all possible candidates for a damage site. These have then to be clustered, finding the location of the damage site and its size. For this task many useful clustering algorithms exist. 
%In order to identify clusters in images a useful group of machine learning algorithms are clustering algorithms. 

%In order to identify clusters in images a useful group of machine learning algorithms are clustering algorithms. Having objects represented by a particular feature, in the case of damage sites these consist of the darkness of pixels, the original image can be transformed using predefined transformations in order to transform the image only containing the interesting features. Once the image is transformed the clusters of features can be localized using clustering algorithms. One particularly useful clustering algorithm is the Density-based spatial clustering of applications with noise (DBSCAN) \cite{DBSCAN}, which was introduced in 1999 and is still in use. What makes this algorithm so powerful is that the number of clusters in the image does not have to be known beforehand, as for other clustering algorithms as k-means. Furthermore DBSCAN not only assigns each pixel its corresponding cluster but also identifies all pixels belonging to noise. As the name indicates DBSCAN is a density based clustering algorithm. It takes two parameters, the maximal distance two points can be separated by in a predefined metric and the number of points constituting a cluster. By looping through all pixels and grouping them together clusters are retrieved.



















%%%%\chapter{Preliminary Studies}

Before the necessary data for the training of a classifier, distinguishing between different damage mechanisms, was available, preliminary studies on different datasets were performed. The aim was to find an architecture performing well on a relatively small dataset distinguishing between up to five classes. Therefore the influence of the size of the training dataset, different network architectures, and architectural details were investigated. \\

\noindent The datasets were chosen from the most popular available datasets. \\
\noindent \textbf{CIFAR-10} \cite{Krizhevsky2009}: (Canadian Institute for Advanced Research 10) consisting of $60,000$ $32\times 32$ colour images distributed among 10 classes. \\
\noindent \textbf{ILSVRC} \cite{imagenet_cvpr09}: (ImageNet Large Scale Visual Recognition Challenge) consisting of over $14,000,000$ colour images of varying sizes distributed among over $20,000$ classes.

\section{CIFAR-10}

\subsection{Architectures}
Due to the small window size of the images in the CIFAR-10 datasets, some rather simple architectures were tested. The networks decided upon are \\
\noindent\textbf{Classical CNN}: A CNN with a few convolutional layers with a fully connected layer at the end. \\
\noindent\textbf{Graham simplified}: A CNN modeled after the winning CNN in the 2015 CIFAR-10 competition. Two convolutional layers alternate with a pooling layer, where the number of starts with $320$ and increases in each layer by $320$. In order to avoid excessive padding the number of layers was reduced. \\
\noindent\textbf{EERACN}: A CNN that uses convolutional layers followed by bottleneck layers, inspired by \cite{Xu2015}. \\
The details of each architecture can be seen in \ref{cha:Appendix_architectures}

\subsection{Influence of the Size of the Dataset}
Expecting to collect datasets with up to $1000$ examples per damage mechanism, $5$ categories of the CIFAR-10 dataset were chosen randomly and the networks performance was evaluated with up to $5000$ samples overall. The resulting accuracies can be seen in figure \ref{fig:Accuracy_Comparison_CIFAR10}. \\
\begin{figure}
  \includegraphics[width=\linewidth]{Accuracy_CIFAR10.pdf}
\caption{Accuracy on training sets of different sizes.}
\label{fig:Accuracy_Comparison_CIFAR10}
\end{figure}

While the classical CNN, performs comparably well for small datasets, its accuracy does not seem to improve much with an increasing amount of data. This is possibly caused by the rather small number of trainable parameters, with most trainable parameters in the fully connected last layer. The Graham simplified network, oscillates and has a rather low accuracy, probably caused by the large number of trainable parameters. The EERACN network has a smaller standard deviation from its mean and achieves an appropriate accuracy with an increasing trend for larger training sizes. With a similar amount of trainable parameters as the classical CNN, but located in its convolutional layers instead of the fully connected layer. \\

\section{ILSVRC}



\chapter{Datasets}

\section{Data Collection - First Stage}
At first damage sites in the recorded SEM micrographs were classified using the conventional class definitions for damage mechanisms in dual phase steels, as explained in \ref{cha:DualPhaseSteels}. Additionally one class was added, namely evolved damage sites. These sites have evolved into the surrounding ferrite, and the information of the underlying mechanism is not unique. \\

Two researchers were involved in the creation of a dataset using labelImg \cite{labelImg}. This tool enables its user to zoom into images and create rectangular bounding boxes around objects in the image that need to be classified. Then a label was added to the marked object. Additionally the tool gives its user the option to mark objects as "difficult". This option was additionally used for damage sites, that did not evolve into the surrounding ferrite but were not clearly assignable to a certain class. These objects can be emmitted from the training of a network, as the need for represantative samples were needed for the training of a classifier. A labeled micrograph using labelImg can be seen in figure \ref{fig:labelImg}. \\

\begin{figure}
\centering
\includegraphics[width=\textwidth]{labelImg.png}
\caption{Damage site classification using labelImg.}
\label{fig:labelImg}
\end{figure}

In order to have a consistently labeled dataset, the researcher further more would correspond if instances appeared that were difficult to assign a class. If a consensus was reached the class agreed upon was chosen as the class for this particular instance, otherwise the "difficult" label was added. During labeling difficulties arose for the assignment of boundary decohesions and evolved damage sites. Therefore an internal study was performed assessing whether the class definitions were clear enough. 

\section{Consistency of Class Definitions}

The participants of the study received the following class definitions together with $25$ instances belong to either of the classes.\\
\begin{itemize}[label={}]
\item "\textbf{Inclusion}: Inclusions, either holes left from preparation or actual inclusions."
\item "\textbf{Martensite Cracking}: Brittle cracked martensite islands."
\item "\textbf{Interface Decohesion}: Damage to the martensite/ferrite boundary."
\item "\textbf{Boundary Decohesion}: Damage to ferrite grain boundaries."
\item "\textbf{Evolved Damage}: More than one active damage mechanism, e.g. martensite cracking evolved into ductile damage in" ferrite.
\end{itemize}

\subsection{Martensite Cracking}

\begin{figure}
\centering
\includegraphics[width=\textwidth]{/Classification/Martensite.pdf}
\end{figure}

\begin{figure}[H]
\begin{subfigure}{.25\textwidth}
\centering
  \includegraphics[width=.8\linewidth]{/Classification/Martensite0.pdf}
  \caption{MC}
  \label{fig:MC}
\end{subfigure}%
\begin{subfigure}{.25\textwidth}
\centering
  \includegraphics[width=.8\linewidth]{/Classification/Martensite1.pdf}
  \caption{ID}
  \label{fig:Interface_scalebar}
\end{subfigure}%
\centering
\begin{subfigure}{.25\textwidth}
\centering
  \includegraphics[width=.8\linewidth]{/Classification/Martensite2.pdf}
  \caption{NE}
  \label{fig:Notch_scalebar}
\end{subfigure}%
\begin{subfigure}{.25\textwidth}
\centering
  \includegraphics[width=.8\linewidth]{/Classification/Martensite3.pdf}
  \caption{Inclusion}
  \label{fig:Inclusion_scalebar}
\end{subfigure}%
\caption{Abstracted damage mechanism for martensite cracking (MC) in subfigure (a), interface decohesion (ID) in subfigure (b), notch effect (NE) in subfigure (c),  and inclusions in subfigure (d). The martensite islands, the ferrite matrix, and the foreign body are labeled as M, F, and I respectively.}
\label{fig:classes}
\end{figure}

\begin{figure}
\centering
\includegraphics[width=\textwidth]{/Classification/Inclusion.pdf}

\end{figure}

Investigating the agreement with the predetermined classes for each sample is shown in table \ref{tab:Reliability}. As can be seen similar difficulties were found for the classes boundary decohesion and evolved damages. Furthermore it was found that due to the high martensite concentration, ferrite grains rarely if at all were directly adjacent to each other. Rather ferrite grains were seperated by martensite grains. Indicating that damage sites marked as boundary decohesions result from different mechanisms. \\

Additionally it was found that damage sites appeared often between the tips of two martensite islands. Therefore a new class was introduced, termed notch effect. This category is classified by its location in the microstructure exhibiting a special geometry. Due to the hardness of the martensite at this point in the geometry a stress concentration is present, resulting the formation of a damage site. In contrast to the other classes the underlying physical process, leading to the formation of a void, is not determined. It is possible that the two martensite islands were connected, prior to the crack occuring, or that a ferrite bridge existed between two separate martensite islands and an interface decohesion took place. \\

The new classes agreed upon are shown diagrammatically in figure \ref{fig:classes}. Some of the already labeled damage sites were therefore relabeled using the new class definitions. 

\begin{table}[H]
 \begin{center}
  \begin{tabular}{@{} *2l @{}} \toprule[2pt]
   Damage Category & Accuracy \\\midrule
   Martensite Cracking & $85 \%$   \\ 
   Inclusion  & $84 \%$ \\ 
   Interface Decohesion  & $76 \% $ \\
   Evolved & $56\%$ \\
   Boundary Decohesion & $44 \%$ \\ \bottomrule[2pt]

  \end{tabular}
 \end{center}
 \caption{Agreement for the classification of damage sites by hand. }
 \label{tab:Reliability}
\end{table}

%
%During labeling, some difficulties arose. Mainly the classes boundary decohesion and evolved damage site posed particularly difficult. An internal study was therefore performed in order to assert whether the class definitions were narrow enough. The study showed that other researchers also had difficulties in classifying damage sites predetermined to belong to these classes. 
%The agreement with the damage classes determined beforehand can be seen in table \ref{tab:Reliability}. Major problems arose with boundary decohesions and evolved damage sites. Due to the high concentration of martensite in the dual-phase steel used, few ferrite grains are adjacent to each other. Therefore boundary decohesions will occur rarely, and sites resembling boundary decohesions probably result from other mechanisms. Problems with evolved damage sites come from the lack of a clear threshold after which a damage site falls into this class. Due to this, boundary decohesions and evolved damages were not used for the classification algorithm to distinguish between.\\
%
%\begin{table}[H]
% \begin{center}
%  \begin{tabular}{@{} *2l @{}} \toprule[2pt]
%   Damage Category & Accuracy \\\midrule
%   Martensite Cracking & $85 \%$   \\ 
%   Inclusion  & $84 \%$ \\ 
%   Interface Decohesion  & $76 \% $ \\
%   Evolved & $56\%$ \\
%   Boundary Decohesion & $44 \%$ \\ \bottomrule[2pt]
%
%  \end{tabular}
% \end{center}
% \caption{Agreement for the classification of damage sites by hand. }
% \label{tab:Reliability}
%\end{table}

\section{Data Collection - Second Stage}

Since the information of the location of the damage site inside of the micrograph is not necessary for the training of a classifier, a new method was used for the classification of damage sites. The goal is to eliminate the need to search for damage sites by hand. This was performed by using DBSCAN, as explained in chapter \ref{cha:CNN}. 




undersampling, oversampling, additional weights



A consistently labeled dataset is crucial for the training of a classifier. When elements in the training set are classified using different class definitions or are mislabeled, the convergence and the performance of a network used for classification is hindered. \\

In a first approximation the effect of using different class definitions for labeling or misclassification can be explained as follows. Assuming the network has already been trained to some extent on a correctly labeled dataset, two similar samples, both belonging to class $c_0$ but one of them incorrectly labeled as class $c_1$, will be mapped by the feature extraction to close proximity of each other in the feature space. The fully connected layer in the end of the CNN will then have to learn to map these close points in the feature space to the maximal distance in the output space, requiring large weight corrections, therefore slowing down training. Furthermore, if the two points in the feature space are arbitrarily small to each other the effect of a mislabeled training sample will then nullify the adaptation of the network to a correctly labeled sample. This mapping problem can be seen in figure \ref{fig:featuremapping}.

\begin{figure}[H]
\centering
\includegraphics[width=\textwidth]{featuremapping.pdf}
\caption{The classification of two similar images in the input space to two different classes. Given a feature mapping this will on the one hand lead to a close distance in the feature space, while on the other hand to a desired maximal distance in the output space. The possible outputs are indicated by a dashed line.}
\label{fig:featuremapping}
\end{figure}

%Due to the necessary time for the labeling of damage sites, this task is performed by multiple researchers. Unclear class definitions will lead to different conclusions for similar damage sites, leading the neural network to adapt its weights once in order to learn the mapping $o_0 \mapsto c_0$ and for a similar object $o_1 \mapsto c_1$, while the true class may be either $c_0$ or $c_1$. This will nullify the correction of the networks weights. \\

\section{Reliability of Training Data}

Due to the need of consistently labeled datasets, for the training process of the classifier, a study was performed internally beforehand. Five experts were asked to classify $25$ damage sites given the following descriptions. 
\begin{itemize}[label={}]
\item \textbf{Inclusion}: Inclusions, either holes left from preparation or actual inclusions.
\item \textbf{Martensite Cracking}: Brittle cracked martensite islands.
\item \textbf{Interface Decohesion}: Damage to the martensite/ferrite boundary.
\item \textbf{Boundary Decohesion}: Damage to ferrite grain boundaries.
\item \textbf{Evolved Damage}: More than one active damage mechanism, e.g. martensite cracking evolved into ductile damage in ferrite.
\end{itemize}
The agreement with the damage classes determined beforehand can be seen in table \ref{tab:Reliability}. Major problems arose with boundary decohesions and evolved damage sites. Due to the high concentration of martensite in the dual-phase steel used, few ferrite grains are adjacent to each other. Therefore boundary decohesions will occur rarely, and sites resembling boundary decohesions probably result from other mechanisms. Problems with evolved damage sites come from the lack of a clear threshold after which a damage site falls into this class. Due to this, boundary decohesions and evolved damages were not used for the classification algorithm to distinguish between.\\

\begin{table}[H]
 \begin{center}
  \begin{tabular}{@{} *2l @{}} \toprule[2pt]
   Damage Category & Accuracy \\\midrule
   Martensite Cracking & $85 \%$   \\ 
   Inclusion  & $84 \%$ \\ 
   Interface Decohesion  & $76 \% $ \\
   Evolved & $56\%$ \\
   Boundary Decohesion & $44 \%$ \\ \bottomrule[2pt]

  \end{tabular}
 \end{center}
 \caption{Agreement for the classification of damage sites by hand. }
 \label{tab:Reliability}
\end{table}

Furthermore damage sites located in a special geometrical constellation appeared often. These are classified by being located between the tips of two martensite islands. The underlying mechanisms of the nucleation of these damage sites is unknown, and might result from the cracking of one formerly connected martensite or the decohesion of a ferrite bridge between two martensite islands. Characteristic for this constellation is a stress concentration at this particular point in the microstructure, this class is called notch effect. The final classes used for classification in this work are shown in figure \ref{fig:classes}.

\begin{figure}[H]
\begin{subfigure}{.25\textwidth}
\centering
  \includegraphics[width=.8\linewidth]{MartensiteCracking_abstraction_scalebar.pdf}
  \caption{MC}
  \label{fig:MC}
\end{subfigure}%
\begin{subfigure}{.25\textwidth}
\centering
  \includegraphics[width=.8\linewidth]{InterfaceDecohesion_abstraction_scalebar.pdf}
  \caption{ID}
  \label{fig:Interface_scalebar}
\end{subfigure}%
\centering
\begin{subfigure}{.25\textwidth}
\centering
  \includegraphics[width=.8\linewidth]{NotchEffect_abstraction_scalebar.pdf}
  \caption{NE}
  \label{fig:Notch_scalebar}
\end{subfigure}%
\begin{subfigure}{.25\textwidth}
\centering
  \includegraphics[width=.8\linewidth]{Inclusion_abstraction_scalebar.pdf}
  \caption{Inclusion}
  \label{fig:Inclusion_scalebar}
\end{subfigure}%
\caption{Abstracted damage mechanism for martensite cracking (MC) in subfigure (a), interface decohesion (ID) in subfigure (b), notch effect (NE) in subfigure (c),  and inclusions in subfigure (d). The martensite islands, the ferrite matrix, and the foreign body are labeled as M, F, and I respectively.}
\label{fig:classes}
\end{figure}



\section{Influence of an Incorrectly Labeled Dataset}

The influence of an incorrectly labeled dataset was studied using the CIFAR-10 dataset, with the EERACN network, see \ref{app:EERACN}. The network was trained for $30$ epochs with an Adam optimizer \cite{Sharma2017} with a learning rate of $0.0005$ and other parameters set to default. The baseline accuracy for a correctly labeled dataset with constant seeds among experiments and fluctuations resulting from the use of GPU is $0.73\pm 0.01$. Purposely mislabeling $30 \%$, corresponding to the average mislabeling rate from \ref{tab:Reliability}, of the training data leads to an accuracy of $0.66\pm 0.01$ evaluated on a dataset with correct labels, leading to a drop in the accuracy of $0.07 \pm 0.01$. For a realistic setting the test dataset will also be mislabeled. Evaluating the trained network on a mislabeled dataset leads to a further drop in the accuracy to $0.43\pm 0.01$, since correctly labeled inputs will be evaluated as wrong predictions. The mislabeled data has therefore two effects. Firstly the performance of the network drops significantly. Secondly the measured accuracy of the network will be skewed towards a network using random predictions. 


%The network was once trained with a correctly labeled dataset and then with a dataset with $30\%$ of the datapoints mislabeled, corresponding to the average mislabeling rate from \ref{tab:Reliability}. The baseline accuracy for a correctly labeled dataset with constant seeds among experiments and fluctuations resulting from the use of GPU is $0.73\pm 0.01$. Training the network with mislabeled data and evaluating the trained network on a dataset labeled correctly the accuracy drops to $0.66\pm 0.01$. Because the data used for the training of a network distinguishing between damage mechanisms is completely labeled by hand the test dataset would also be labeled incorrectly. The accuracy of a network evaluated on a test dataset with inconstistent labeling leads to a further drop in the accuracy to $0.43\pm 0.01$, due to correct predictions being labeled as wrong.\\


\section{Datasets}
The datasets were obtained from SEM micrographs by two researchers. 



The datasets were obtained in two ways. At first labelImg \cite{labelImg} was used. With this tool it is possible to mark damage sites in SEM panoramas, requiring the user to search for them by zooming into the micrograph. Later once the localization algorithm was implemented as explained in chapter \ref{cha:Localization}, significantly reducing the required time for the creation of a dataset by negating the need to localize damage sites by hand. \\

\begin{figure}[H]
\begin{subfigure}{.5\textwidth}
\centering
  \includegraphics[width=.8\linewidth]{Artifact_scalebar.png}
\end{subfigure}
\begin{subfigure}{.5\textwidth}
\centering
  \includegraphics[width=.8\linewidth]{Artifact_2_scalebar.png}
\end{subfigure}
\caption{Artifact showing on the surface of the sample.}
\label{fig:artifacts}
\end{figure}

During deformation artifacts can form on the surface of the micrograph. A few examples of such artifacts are shown in figure \ref{fig:artifacts}. Due to the preparation of the sample these do not show in ex-situ experiments and very rarely in early stages of in-situ experiments. The dataset is therefore split into two categories. The first one consists of surface micrographs not containing artifacts on the sample's surface, while the second one consists of surface micrographs containing said artifacts. \\

The available data for each damage mechanism and category after labeling is shown in table \ref{tab:Dataset}.

%The data is split into two categories. The first category are surface micrographs that were taken right after preparing the sample. Due to the preparation these do not contain artifacts from the deformation of the sample. This first category consists of ex-situ experiments and the first stage of in-situ experiments. Later stages of in-situ experiments fall into the second category, 
%The data is split into three parts. Firstly data from ex-situ experiments, with different grades of stress. The dual-phase steel was prepared before each recording with the SEM, therefore no artifacts can be seen on the surface of the sheet. Secondly data from the first stage of in-situ experiments, this data is  The second round of experiments was performed in-situ. Due to the in-situ nature of those experiments at higher rates of deformation, recordings contain artifacts on the surface of the probe. The datasets with the number of damage sites in each category can be seen in table \ref{tab:Dataset}. \\ 

\begin{table}[H]
\begin{center}
\begin{tabular}{@{} *5l @{}} \toprule[2pt]
Training set &  \multicolumn{4}{c}{Damage Mechanism}   \\\midrule
 & Inc & MC & ID & NE   \\ 
ex-situ  & 379 & 691 & 788 & 443\\ 
in-situ  & 193 & 796 & 586 & 418 \\ \bottomrule
all  & 572 & 1487 & 1374 & 861\\\bottomrule[2pt]

\end{tabular}
 \caption{The number of damage sites found in the ex-situ and in-situ experiments per class (inclusion (Inc), martensite cracking (MC), interface decohesion (ID), and notch effect (NE)).}
 \label{tab:Dataset}
\end{center}
\end{table}
%\chapter{Practical Considerations}
\label{cha:PracticalConsiderations}

In this chapter implementation details will be discussed. Firstly, the treatment of imbalanced classes will be discussed. Secondly, a hierarchical classification algorithm consisting of two CNNs will be introduced. Lastly, the basis for the training and the evaluation of CNNs will be treated.

\section{Class Imbalance}

As can be seen in figure \ref{fig:datasets}, the number of samples per class are distributed unevenly, which is called class imbalance. The action of a neural network can be understood to some degree as an approximation of the Bayesian a posteriori probability. If the frequency of the distribution among classes varies heavily, this will propagate to the output. It is now possible for a neural network to minimize the error function by learning the a priori probability, and classes with few examples will not be learned by the network and therefore not predicted. E.g. In the extreme case of two classes, where class $0$ occurs $99$ more often than class $1$, always predicting class $0$ will lead to a low error on the training dataset. In a real-world application, the aim of a classification algorithm is not to minimize the loss, but rather predict individual cases correctly.\\

\begin{figure}[H]
\centering
\includegraphics[width=0.8\textwidth]{/Datasets/Dataset.pdf}
\caption{Distribution of samples among the classes inclusion (Inc), notch effect (NE), interface decohesion (ID), and martensite cracking (MC), in increasing order.}
\label{fig:datasets}
\end{figure}

Multiple methods exist addressing this issue. They can be categorized in sampling methods, artificially modifying the sample distribution in the training dataset, and methods modifying the learning process and prediction of a classifier. In the following a selection of possible methods will be explained

\subsubsection{Oversampling}
Oversampling describes the usage of already available examples of underrepresented classes multiple times. The most basic approach is to randomly select elements of an underrepresented class and add it to the training set until the classes are balanced. Because the information coded in the samples of the underrepresented classes is used multiple times, this method is usually associated with overfitting and poor generalizability. Furthermore due to increased training size, the time a network requires for training is increased. 

\subsubsection{Undersampling}
Undersampling follows the opposite idea of oversampling. Instead of increasing the training dataset with samples from the underrepresented class, samples from overrepresented classes are chosen at random until all classes have the same amount of training data. The obvious downside to this method is that not all available data are used for training, requiring an initial dataset of sufficient size. While balance is restored the possible accuracy for overrepresented classes may not be achieved. As the training size decreases, so does the required training time. \\

\subsubsection{Scaling}
Scaling is a method modifying the learning process of a neural network. The adjustment of the weights due to a sample belonging to a class with lower frequency will be increased by a factor, depending on the a priori probability of that class. Correspondingly this can be understood as the modification of the error function, with the corresponding scaling factor. Wrong classifications of underrepresented classes therefore lead to a larger overall error. 

\subsubsection{Post-Scaling}
Instead of modifying the learning process, this method aims to correct the prediction of a neural network after it has been trained. The output of the network is then scaled with the a priori probability. Both scaling and post-scaling methods heavily rely on the Bayesian a posteriori probability interpretability of the output of the network.\\


Buda et. al. performed experiments on three benchmark datasets, in order to investigate the influence of class imbalance and the performance of methods used to compensate its effects \cite{Buda2017}. While multiple studies exist investigating the effects of class imbalance, the study performed by Buda et al. is one of the few focusing on image classification using CNNs. This study is therefore of particular interest for this thesis. The two major points relevant for this thesis are 
\begin{itemize}
\item The most dominant method is oversampling
\item Oversampling does not imply overfitting
\end{itemize}
In the following oversampling will therefore be the method of choice when dealing with imbalanced class distributions.

\section{Class Hierarchy}
\label{sec:Architecture}

In general imaging parameters can change between experiments. Due to the small size of the training dataset, in this work a controlled environment is created, by keeping the resolution in particular constant among experiments, in order to prevent the need for a classifier to adapt to a changing environment while at the same time learning the relevant features, necessary for a distinction between the different damage mechanisms. Size differences between different damage mechanisms will therefore be coherent among the input of a CNN. The most distinct size difference can be seen between inclusions, and the remaining damage mechanisms, as indicated in figure \ref{fig:SizeDifference}. Using a single CNN for the classification of damage mechanisms, would come with the problem that a window size large enough to encompass an inclusion site, would lead to the other damage mechanisms being underrepresented in the input, while a window size on the typical scale of a martensite cracking, an interface decohesion, or a notch effect would not show a typical inclusion site in its entirety. \\

\begin{figure}[H]
\centering
\begin{subfigure}{.25\textwidth}
\centering
  \includegraphics[width=.8\linewidth]{Inclusion_scalebar_arrow.png}
  \caption{Inclusion}
  \label{fig:Inclusion_scalebar}
\end{subfigure}%
\begin{subfigure}{.25\textwidth}
\centering
  \includegraphics[width=.8\linewidth]{Martensite_scalebar_arrow.png}
  \caption{MC}
  \label{fig:Martensite_scalebar}
\end{subfigure}%
\begin{subfigure}{.25\textwidth}
\centering
  \includegraphics[width=.8\linewidth]{Interface_scalebar_arrow.png}
  \caption{ID}
  \label{fig:Interface_scalebar}
\end{subfigure}%
\begin{subfigure}{.25\textwidth}
\centering
  \includegraphics[width=.8\linewidth]{Notch_scalebar_arrow.png}
  \caption{NE}
  \label{fig:Notch_scalebar}
\end{subfigure}%
\caption{The different categories of damage sites, that the classifiers are trained to distinguish from each other together with a scale bar.}
\label{fig:SizeDifference}
\end{figure}

Therefore a hierarchical structure for the classification task at hand was chosen. Taking advantage of the size difference between inclusions and the remaining damage categories, the first classifier receives an input of a size of a typical inclusion site. It then classifies inclusion sites and passes the remaining sites to the next classifier. The second classifier has a input window  receiving only the relevant features to distinguish the remaining three classes, martensite cracking, interface decohesion, and notch effect. A diagrammatic depiction of this is shown in figure \ref{fig:Architecture}.\\

%Due to the inherent size difference between the different damage categories, as can be seen in figure \ref{fig:SizeDifference}, using a single CNN for the distinction between all possible damage mechanisms, comes with a major drawback. As the resolution is restrict to be the same experiments, in orde
%
%Due to the inherent size difference between the different damage categories, as can be seen in figure \ref{fig:SizeDifference}, using a single CNN for an ad-hoc classification, comes with one major drawback. Choosing a receptive field large enough to encase an entire inclusion, would lead to the void of other damage categories being underrepresented in the image. While choosing a receptive field of the typical size of a void of a category different than inclusion would be too small in order to show the relevant features of an inclusion.\\
%
%Therefore, instead of using a single CNN for the classification task, a hierarchical structure was chosen. Taking advantage of the size difference between inclusions and the remaining damage categories, the first classifier receives an input of a size of a typical inclusion site. It then classifies inclusion sites and passes the remaining sites to the next classifier. The second classifier has a smaller receptive field receiving only the relevant features to distinguish the remaining three classes, martensite cracking, interface decohesion, and notch effect. A diagrammatic depiction of this is shown in figure \ref{fig:Architecture}.\\

\begin{figure}[H]
\begin{center}
  \includegraphics[width=\linewidth]{Architecture.png}
\caption{Hierarchical classifier.}
\label{fig:Architecture}
\end{center}
\end{figure}

\section{Training}

\subsection{Implementation}
The networks were implemented and tested using keras 2.1.5 \cite{chollet2015keras} with tensorflow 1.2.1 \cite{tensorflow2015-whitepaper} and cuDNN 6.0.21 \cite{Chetlur2014} as backend. The clustering algorithms were implented using scikit-learn \cite{scikit-learn}. Training and evaluation was performed on a NVidia GeForce GTX 1070.\\

Both networks were trained using the Adam optimizer with a learning rate of $0.001$ for the first network and a learning rate of $0.0005$ for the second network, the remaining parameters were left at their default values, i.e. $\beta_1=0.9$, $\beta_2=0.999$, $\epsilon=10^{-8}$, and a learning rate decay of $0.0$. The first network was trained for $50$ epochs and the second network for $90$ epochs. The training loss and accuracy during training is exemplary shown in figure \ref{fig:Training}. \\

\begin{figure}
\centering
\includegraphics[width=\textwidth]{training.pdf}
\caption{Loss and accuracy of a network during training.}
\label{fig:Training}
\end{figure}

The available data was split into $80\%$ used for training and $20\%$ used for testing, using oversampling such that each class was equally represented. In order to prevent the networks from having to adapt to the non-zero mean of the input distribution, due to the pixel values taking values between $0$ and $255$, the input was rescaled to the interval $[-1,1]$. \\

\subsection{Statistical Fluctuations}

Through backpropagation a CNNs final weights depend on its initial weights. Due to the pseudo-random nature of initialization networks initialized with different weights will result in different final weights. In order to estimate the performance of a network multiple networks are initialized, and the final results are given through the mean and the standard deviation estimation. Another influencing factor on a CNNs fluctuations are the implemented libraries using the GPU. These employ asynchronous atomic reduction methods, that are non-deterministic in their nature. The distribution of the accuracy of a network initialized and trained $10$ times evaluated on a test dataset are shown in figure \ref{fig:AccuracyFluctuation}.

\begin{figure}[H]
\centering
\includegraphics[width=\textwidth]{StatisticalFluctuations.pdf}
\caption{Different accuracies of a network after training with different initial weights, through the random weight initialization, together with the mean accuracy and standard deviation.}
\label{fig:AccuracyFluctuation}
\end{figure}

%A CNN's final weights depend on the initial weights. Due to the pseudo-random nature of initialization, a network will converge to a different state.  This effect can partly be suppressed by setting the seeds of the pseudo-random number generator. Another influence is the calculation using a GPU. 
%Depending on the initial state of a network, i.e. the values of the weights after initialization, its weights will converge to different values. This can be suppressed by setting the seed of the pseudo-random number generation. Another source of fluctuations 


















%
\chapter{Evaluation and Results} % Main chapter title

\label{Performance} % For referencing the chapter elsewhere, use \ref{Chapter1} 

In this chapter the performance of the different classifiers will be studied. In section \ref{sec:perfectconditions} the performance of the networks will be evaluated given that the localizer returns sites belonging to one of the classes that the classifier was trained on. In section \ref{sec:Robustness} the networks robustness will be tested given any possible site that the localizer can find. 

For the training of each classifier the dataset an $80-20$ split is applied for training of each network and testing. In order to prevent overfitting, the training set is further split into $80\%$ actually used for the adjustments of the networks weights and $20\%$ for validation. \\
Furthermore the pixel values are transformed from the interval $\{0,\dots,255\}$ to the interval $[-1,1]$.\\


\newpage
\section{First Classifier}

As described in chapter \ref{cha:Architecture} the first classifier filters all inclusions out and passes all other sites to the next classifier. The size of its input window is chosen in such a way that even large inclusions can be seen by the network. $250\times 250$ was chosen as an input size. Furthermore in order to prevent the network from learning a bias towards one class the training data consists of sites containing inclusions and sites not containing inclusions, evenly spread among inclusions and sites not containing inclusions. In order to have a more representative training set, the training data is then also distributed equally among martensite cracking, interface decohesion, and notch effect sites.

\subsection{Architecture}
Due to the small amount of available data optimization of a networks architecture is not promising to increase its accuracy. Therefore three networks were chosen that performed well in the ImageNet challenge with similar window sizes, namely the Xception network \cite{Xception}, the InceptionResNetV2 network \cite{InceptionResNetV2}, and the InceptionV3 network \cite{InceptionV3}. Their accuracies can be seen in table \ref{tab:AccuracyComparisonNetworks}. Since the InceptionV3 network performed best it was chosen for the first classifier.\\

\begin{table}[H]
 \begin{center}
  \begin{tabular}{@{} *5l @{}} \toprule[2pt]
   Network &  \multicolumn{3}{c}{Accuracy}  \\\midrule
    & ex-situ  & in-situ  & all   \\ 
   Xception  & 0.868 & 0.878 & 0.866\\ 
   InceptionResNetV2  & 0.854 & 0.849 & 0.888\\
 \boxit{8.46cm}   InceptionV3 & 0.901 & 0.863 & 0.915 \\ \bottomrule[2pt]

   \label{tab:AccuracyComparisonNetworks}
  \end{tabular}
 \end{center}
 \caption{Accuracy of the networks trained on different training sets evaluated on the available datasets.}
\end{table}

\subsection{Generalization to new Datasets}
While working only with the data recorded ex-situ, the network generalized well on this dataset. But applying it to newly recorded in-situ data, leads to drops in the networks accuracy, as can be seen in table \ref{tab:AccuracyComparisonInception}. By including the in-situ data into the training dataset, the networks performance was increased drastically. This behavior is also shown in figure \ref{fig:Inception_ex_vs_in}.

\begin{table}[H]
 \begin{center}
  \begin{tabular}{@{} *5l @{}} \toprule[2pt]
   Training set &  &Test set&  \\\midrule
    & ex-situ  & in-situ  & all   \\ 
   ex-situ  & 0.90 & 0.55 & 0.71\\ 
   all  & 0.94 & 0.90 & 0.92\\\bottomrule[2pt]

  \end{tabular}
 \end{center}
 \caption{Accuracy of the networks trained on different training sets evaluated on the available datasets.}
 \label{tab:AccuracyComparisonInception}
\end{table}

\begin{figure}
  \includegraphics[width=\linewidth]{Inception_ex_vs_in.pdf}
\caption{Comparison of the precision between networks trained excluding in-situ data and including in-situ data. There is a visible drop of the accuracy once the classifier trained on the ex-situ data is applied to the in-situ data to just above a random classifier. Including in-situ data into the training leads to a performance just above the network trained on ex-situ data evaluated on ex-situ data.}
\label{fig:Inception_ex_vs_in}
\end{figure}

%\subsection{Choosing a Threshold}
%The output of the network can be seen as a probability that a certain damage site belongs to a damage class. By choosing a threshold above which the classifier should classify a given damage site, a trade-off has to be made between the desired accuracy and the classification rate. In figure \ref{fig:InceptionACC_EFF_THETA} the accuracy of the first network together with its classification rate plotted against the threshold can be seen. Requiring the accuracy to be $95\%$ would correspond to a threshold of $\theta = 0.7$ and a classification rate of $92\%$. The remaining $8\%$ of damage sites have to be labeled afterwards by hand. 
%
%
%%Due to the first classifier acting as a filter for the next classifier, it is necessary to minimize the number of falsely classified damage sites. By introducing a threshold for the network to only decide for a damage site to be of a category if the returned probability exceeds it, the number of falsely classified damage sites can be minimized, at the cost of a lower efficiency. By labeling the not classified damage sites by hand, those can be later introduced back into the system as new training data, representing points in the input space not learned by the network. The accuracy together with the efficiency against the threshold are shown in figure \ref{fig:InceptionACC_EFF_THETA}. A reasonable choice for this threshold is $\theta=0.7$, resulting in an accuracy of $95\%$ classifying $92\%$ of all damage sites. 
%
%\begin{figure}
%  \includegraphics[width=\linewidth]{Inception_ACC_CLA_THETA.pdf}
%\caption{Accuracy of classified sites plotted together with the ratio of classified sites against the threshold. At $\theta=0.7$ of the $92\%$ classified damage sites $95\%$ were classified correctly.}
%\label{fig:InceptionACC_EFF_THETA}
%\end{figure}
%
%%Due to the first network acting as a filter for inclusions, it is necessary to minimize the number of damage sites falsely classified as inclusions or equivalently minimizing the number of false negatives. The relevant quantity for inspecting the performance under these conditions of a binary classifier, in this case the neural network, is its precision (positive predictive value) defined by
%%\begin{equation}
%%PPV = \frac{TP}{TP+FP}
%%\end{equation}
%%where $TP$ is the number of correctly classified inclusions and $FP$ is the number of other damage sites classified as inclusions. \\
%%On the other hand for the network to be useful, the number of correctly classified inclusions should be maximized, in order to reduce the amount of work necessary to relabel remaining damage sites by hand. 
%
%%\subsection{Ex-situ to in-situ}
%%While the first classifier, responsible for filtering out the inclusion sites, performed well on the ex-situ data sets, problems arose while trying to use it for the classification of in-situ damage sites. By using some of the in-situ data for the training of the network, its performance was substantially increased. The data sets for training and testing of the classifier are shown in the following table. \\
%%
%%\begin{tabular}{| l | c | c | c | c |}
%%\hline
%% & ex-situ train & ex-situ test & in-situ train & in-situ test \\ \hline
%%excluding in-situ & training & training & ignored & testing \\ \hline
%%including in-situ & training & training & training & testing \\ \hline
%%only ex-situ data & training & testing & ignored & ignored \\ \hline
%%\end{tabular}
%%
%%\subsubsection{Training excluding in-situ data}
%%Without including the in-situ data in the training set of the network, it characterized $52$ out of the $62$ inclusion sites correctly, while also classifying $164$ out $409$ sites that aren't inclusions as inclusions. Since the purpose of the first network is to filter out inclusions, this performance would make it inapplicable. While some of the inclusion sites not labeled as inclusions, can pass through the system and have to be labeled afterwards by hand, labeling sites that aren't inclusions with a high confidence as inclusions poses a huge problem for its usage as a filtering system. 
%
%%\subsubsection{Training including in-situ data}
%%By including some of the in-situ data into the training set, $44$ out of the $62$ inclusions sites were classified correctly as inclusions, performing slightly worse than the network trained on only the ex-situ data set. However none of the remaining $409$ sites were classified as inclusions.
%
%%\subsubsection{Training including all data}
%%Due to new data being created, one more test was included. The performance of the network trained on the final dataset is shown in figure \ref{fig:FirstClassifierFinal}. As can be seen the performance of the network 
%
%\begin{figure}
%  \includegraphics[width=\linewidth]{FinalPerformanceInception.pdf}
%\caption{Purity plotted against efficiency of the network trained on the full dataset. As a comparison the performance of randomly assigning categories is shown as well.}
%\label{fig:FirstClassifierFinal}
%\end{figure}
%
%%\subsubsection{Comparison}
%%In figure \ref{fig:TPR_comparison} the precision of both differently trained networks are shown. As one can see the network performed well, while working only with ex-situ data. Transferring the network to be used on in-situ data the networks precision dropped immensely, rendering the network useless for classifying in-situ images. However, by including a small portion of in-situ data in the training of the network, the number of damage sites wrongly classified by the network as inclusions becomes negligible above a certain threshold. 
%
%\subsection{Confusion Matrix}
%
%
%\newpage
%\section{Second Stage}
%Due to the increased number of labeled damage sites, a more complex network, the InceptionV3 network, was tested in order to assess whether the accuracy can be increased by using a different architecture. The comparison between the two networks can be seen in figure \ref{fig:InVsEE}. Eventhough the amount of data was increased significantly, the simpler architecture still performs better than the complex one. 
%
%\begin{figure}
%  \includegraphics[width=\linewidth]{InceptionVsEERACN_all.pdf}
%\caption{Performance comparison between the InceptionV3 network and the EERACN network distinguishing between brittle and ductile mechanisms}
%\label{fig:InVsEE}
%\end{figure}
%
%%\begin{figure}
%%  \includegraphics[width=\linewidth]{InceptionVsEERACN_nothing.pdf}
%%\caption{Accuracy on detected shadows found by the localization algorithm}
%%\label{fig:TPR_comparison}
%%\end{figure}
%
%\subsection{Brittle versus Ductile Damage Mechanisms}
%Furthermore a further split of the class hierarchy was considered. Instead of distinguishing between the three remaining classes, martensite cracking, interface decohesion, and notch effect, the second classifier should distinguish between brittle and ductile damage mechanisms. Therefore interface decohesions and notch effects are grouped into one class. In figure \ref{fig:2vs3Classes} the true positive rate can be seen 
%Furthermore we tested whether it is sensible to train a network capable of only distinguishing between brittle damage mechanisms (Martensite cracking) and ductile damage mechanisms (interface decohesion and notch effects) involved in the formation of voids. As can be seen in figure \ref{fig:2vs3Classes} a network trained to distinguish ductile damage mechanisms in interface decohesion and notch effects performs just as well if not better in distinguishing between brittle and ductile damage mechanisms as a network trained just for that task.
%
%\begin{figure}
%  \includegraphics[width=\linewidth]{EERACN_2vs3Classes.pdf}
%\caption{True positive rates for the EERACN network distinguishing between two classes (Martensite and rest) and between three classes (Martensite, interface decohesion, and notch)}
%\label{fig:2vs3Classes}
%\end{figure}
%
%\begin{figure}
%  \includegraphics[width=\linewidth]{EERACN_differentTrainingSets.pdf}
%\caption{Accuracy plotted against the classification rate for the EERACN network trained on different training sets evaluated on all test sets}
%\label{fig:TPR_comparison}
%\end{figure}
%
%\begin{figure}
%  \includegraphics[width=\linewidth]{EERACN_differentTrainingSets_test_in_situ.pdf}
%\caption{Accuracy plotted against the classification rate for the EERACN network trained on different training sets evaluated on the original in-situ test set}
%\label{fig:TPR_comparison}
%\end{figure}
%
%\begin{figure}
%  \includegraphics[width=\linewidth]{EERACN_differentTrainingSets_test_Stufe0.pdf}
%\caption{Accuracy plotted against the classification rate for the EERACN network trained on different training sets evaluated on the stage zero test set}
%\label{fig:TPR_comparison}
%\end{figure}
%
%\begin{figure}
%  \includegraphics[width=\linewidth]{EERACN_differentTrainingSets_test_Deformed.pdf}
%\caption{Accuracy plotted against the classification rate for the EERACN network trained on different training sets evaluated on the deformed test set}
%\label{fig:TPR_comparison}
%\end{figure}
%
%\begin{figure}
%  \includegraphics[width=\linewidth]{PPV_different_classes_ex_situ.pdf}
%\caption{PPV for all classes trained on ex-situ data}
%\label{fig:TPR_comparison}
%\end{figure}
%
%\begin{figure}
%  \includegraphics[width=\linewidth]{PPV_different_classes_in_situ.pdf}
%\caption{PPV for all classes trained on in-situ data}
%\label{fig:TPR_comparison}
%\end{figure}
%
%\begin{figure}
%  \includegraphics[width=\linewidth]{PPV_different_classes_stage0.pdf}
%\caption{PPV for all classes trained on stage zero data}
%\label{fig:TPR_comparison}
%\end{figure}
%
%\begin{figure}
%  \includegraphics[width=\linewidth]{PPV_different_classes_deformed.pdf}
%\caption{PPV for all classes trained on deformed data }
%\label{fig:TPR_comparison}
%\end{figure}
%
%
\section{Combined Classifier}

\subsection{In-situ Analysis with Position Tracking}

Combining both classifiers


The combined classifier was used in order to investigate the time evolution of damage sites. This is done by firstly cropping the panoramas in such a way that at the borders of the images at different stages of stress the same physical features can be seen. For the first few stages the naive approach to just using the coordinates $x,y$ of a damage site in stage $i$ and rescaling them by the 



















%%%\chapter{Workflow}

Given a desired accuracy the internal thresholds of the CNNs used in the classifier have to be adjusted. This results in the classifier not being able to classify all damage sites in a given panorama. Nevertheless even for high thresholds this algorithm is capable to reduce the required time for the classification of damage sites. In the following a workflow is proposed.


%\include{Chapters/Conclusion}
%\include{Chapters/Outlook}
%\appendix
%% Chapter 2

\chapter{Appendix} % Main chapter title

\label{Appendix} % For referencing the chapter elsewhere, use \ref{Chapter1} 

\section{Stress-Strain Curve of Dual-Phase Steel 800}

\begin{figure}[H]
\centering
  \includegraphics[width=\linewidth]{StressStrain.pdf}
  \caption{Dual-phase steel SEM micrograph}
  \label{fig:DPStressStrain}
\end{figure}

\section{Experimental Details}

\textbf{Preparation} \\
Before the experiments were performed, the surface of the dual-phase steel was prepared in order to be able to distinguish the martensite islands from the ferrite matrix, in surface micrographs. The preparation process consists of grinding the surface with up to $4000$ grit sandpaper, polishing it with oxide polishing suspension in steps of $6\mu m$, $3\mu m$, and $1\mu m$, and finally etching it with $1 \%$ nital. \\

\noindent \textbf{Probe Geometry} \\
The probe geometry was chosen in such a way that the center of the probe experiences almost homogeneous stress. It can be seen in figure \ref{fig:ProbeGeometry}. \\

\noindent \textbf{SEM Specifications} \\
The surface of the material after deformation was recorded using an SEM. Its specifications and settings can be seen in table \ref{tab:SEM}. \\

\begin{table}[H]
 \begin{center}
  \begin{tabular}{@{} *2l @{}} \toprule[2pt]
   Model & Zeiss LEO 1530 \\\midrule
   Horizontal Field Width & $100\mu m$   \\ 
   Vertical Field Width  & $75\mu m$ \\ 
   Resolution  & $3072\times 2304$ \\
   Detector Type & Secondary Electrons \\
   Electron Source & Field-Emitter Cathode \\ \bottomrule[2pt]

  \end{tabular}
 \end{center}
 \caption{Details of the SEM used for the surface micrographs.}
   \label{tab:SEM}
\end{table}

\section{Wrong Label}

In a first approximation the effect of using different class definitions for labeling or misclassification can be explained as follows. Assuming the network has already been trained to some extent on a correctly labeled dataset, two similar samples, both belonging to class $c_0$ but one of them incorrectly labeled as class $c_1$, will be mapped by the feature extraction to close proximity of each other in the feature space. The fully connected layer in the end of the CNN will then have to learn to map these close points in the feature space to the maximal distance in the output space, requiring large weight corrections, therefore slowing down training. Furthermore, if the two points in the feature space are arbitrarily small to each other the effect of a mislabeled training sample will then nullify the adaptation of the network to a correctly labeled sample. This mapping problem can be seen in figure \ref{fig:featuremapping}.

\begin{figure}[H]
\centering
\includegraphics[width=\textwidth]{featuremapping.pdf}
\caption{The classification of two similar images in the input space to two different classes. Given a feature mapping this will on the one hand lead to a close distance in the feature space, while on the other hand to a desired maximal distance in the output space. The possible outputs are indicated by a dashed line.}
\label{fig:featuremapping}
\end{figure}
%
%\section{Cost Functions}
%\begin{itemize}
%\item How a human learns - punishment?
%\item Define what is right and what is wrong
%\item Mathematical foundation to find optimal weights
%\item Properties of cost functions:
%\subitem $C>0$
%\subitem Output of network close to desired output then cost function close to minimum
%\end{itemize}
%\subsection{Squared Error}
%\begin{itemize}
%\item Linear regression
%\item Classical error function
%\end{itemize}
%\subsection{Categorical Cross Entropy}
%% http://neuralnetworksanddeeplearning.com/chap3.html#the_cross-entropy_cost_function
%\begin{itemize}
%\item Problems with squared error: small gradient
%\item Definition of cross entropy
%\item Using definition of sigmoid function shows that the system learns faster the further it is away from the true solution
%\end{itemize}
%
%\section{Activation Functions}
%Some interesting facts check where to put them later
%\begin{itemize}
%\item gap between computational neuroscience models and machine learining models
%\subitem brain: neurons encode information in a sparse and distributed way check http://journals.sagepub.com/doi/pdf/10.1097/00004647-200110000-00001
%\subitem non-linear activation functions: leaky integrate and fire, sigmoid and tanh similar
%\subitem firing rate of sigmoid about $1/2$ 
%\subitem firing rate of tanh about $0$ but antisymmetry absent in biological neurons
%\end{itemize}
%\subsection{Binary Activation Function}
%\begin{itemize}
%\item Inspired from biological neurons
%\item Historically first used
%\item Rosenblatt perceptron
%\end{itemize}
%\subsection{Sigmoid}
%\begin{equation}
%\phi(v) = \frac{1}{1+\exp(-\alpha v)}
%\end{equation}
%\begin{figure}
%%sigmoid plot
%\end{figure}
%\begin{itemize}
%\item Small changes in input should result in small changes in the output
%\item Continuous activation functions
%\item Problems:
%\subitem Saturation for small and large input (gradient very small)
%\subitem Always positive (use functions like arctangent)
%\end{itemize}
%\subsection{Hyperbolic Tangent}
%\begin{equation}
%\phi(v) = \tanh(\alpha x)
%\end{equation}
%\begin{itemize}
%\item Biologically less plausible than sigmoid activation function
%\item Positive and negative
%\item Better suited for the training of multilayer neural networks
%\end{itemize}
%
%\subsection{ReLu}
%\begin{equation}
%\phi(v) = \max(0,x)
%\end{equation}
%\begin{itemize}
%\item Useful paper \cite{DeepSparceRectifierNN}
%\item Inspired by sparsity of activation
%\item Rectifying non-linearity gives rise to real zeros
%\end{itemize}
%
%\subsection{Advanced ReLu}
%\begin{itemize}
%\item PReLu with trainable parameter $a_i$
%\item Leaky ReLu ...
%\end{itemize}
%
%\subsection{Softmax}
%\begin{equation}
%\phi_j^L = \frac{e^{z_j^L}}{\sum_k e^{z_k^L}}
%\end{equation}
%\begin{itemize}
%\item Sumation over output neurons equals $1$
%\item All activations sum up to 1
%\item Can be interpreted as probability distribution
%\end{itemize}
%
%\section{Comparison}
%Paper:
%\begin{itemize}
%\item Empirical Evaluation of Rectified Activations in Convolution Network
%\end{itemize}
%
%
%\subsection{Gradient Descent}
%\begin{equation}
%\Delta w_{ji}(n) = -\eta \frac{\partial E(n)}{\partial w_{ji}(n)}
%\end{equation}
%\begin{itemize}
%\item First order optimization technique
%\item Calculate local gradient of loss hyper surface
%\item Follow path of steepest descent
%\item Adjustable parameter: Learning rate $\eta$
%\end{itemize}
%\begin{figure}
%\caption{Show trajectory of gradient descent with different learning rates}
%\end{figure}
%
%\subsection{Momentum Gradient Descent}
%\begin{itemize}
%\item Gradient descent: taking step always in direction of instantaneous steepest descent
%\item Momentum gradient descent: add momentum (compare to ball rolling down hill
%\item Inertia: smoother, accelerator, dampening oscillations, helps navigate past local minima (Hier aufpassen noch sehr nah an https://distill.pub/2017/momentum/)
%\item Adding momentum corresponds to giving short term memory
%\end{itemize}
%\begin{figure}
%\caption{Show trajectory of gradient descent with momentum with different parameters}
%\end{figure}
%
%\subsection{Stochastic Gradient Descent}
%Paper:
%\begin{itemize}
%\item Efficient Mini-batch Training for Stochastic Optimization
%\end{itemize}
%\begin{itemize}
%\item Instead of using the informationi of the entire training data use only single elements or batch of elements (batch stochastic gradient descent) from the training data
%\end{itemize}
%
%\subsection{Adam}
%\begin{itemize}
%\item Only uses first order gradients
%\item Computes individual adaptive learning rates for different parameters from estimates of first and second moments of gradients
%\item Adaptive moment estimation $\rightarrow$ Adam
%\end{itemize}
%
%\subsection{L-BFGS}
%
%
%\section{Weight Initialization}
%\begin{itemize}
%\item How to initialize Network properly in order to quickly find the optimal configuration
%\item Naive: Set everything to zero $\rightarrow$ bad, have to introduce symmetry breaking
%\item `` Biases can generally be initialized to zero but weights need to be initialized carefully to break the symmetry between hidden units of the same layer. Because different output units receive different gradient signals, this symmetry breaking issue does not concern the output weights (into the output units), which can therefore also be set to zero.''
%\item Standard gradient descent performing poorly on deep neural networks with random initialization
%\end{itemize}
%
%\subsection{Xavier Initialization}
%Paper:
%\begin{itemize}
%\item Understanding the difficulty of training deep feedforward neural networks
%\end{itemize}
%\begin{itemize}
%\item Introduced to train deep neural networks with sigmoids, tanh or softsign
%\item Ensure error signal reaches all layers
%\end{itemize}
%
%\subsection{He Initialization}
%Paper:
%\begin{itemize}
%\item On weight initialization in deep neural networks
%\item Delving Deep into Rectifiers: Surpassing Human-Level Performance on ImageNet Classification
%\end{itemize}
%\begin{itemize}
%\item https://arxiv.org/pdf/1502.01852.pdf
%\item Introduced in order to improve the performance of neural networks with rectified linear units
%\end{itemize}
%
%\subsection{Comparison}
%\begin{itemize}
%\item Show training of deep network with the two different initialization techniques
%\end{itemize}
%
%
%
%\section{Regularization}
%\begin{itemize}
%\item Training set not representative of problem
%\item Overfitting a problem, detects features not relevant for the classification task
%\item How to train the network such that it generalizes well
%\item Generalization of network: performance on data not part of training data
%\end{itemize}
%
%\subsection{Early Stopping}
%\begin{itemize}
%\item Split data into training, validation and test
%\item Performance of network on validation set indicator for how well the network generalizes
%\item Once the performance on the validation set stagnates or decreases stop training, from this point irrelevant features will be detected
%\end{itemize}
%\begin{figure}
%\caption{Show plot of accuracy/loss on training and validation set over epochs}
%\end{figure}
%
%\subsection{Regularization Theory}
%
%\subsection{Dropout}
%Paper:
%\begin{itemize}
%\item Improving neural networks by preventing co-adaptation of feature detectors
%\item Dropout as data augmentation
%\item Dropout: A Simple Way to Prevent Neural Networks from Overfitting
%\end{itemize}
%A different approach, Dropout, to regularization was first introduced by Srivastava et. al. \cite{DropoutOriginal}. Instead of introducing an additional term to the error function keeping the weights between neurons \"small\" or stopping the training early, once the accuracy on the validation set stagnates, neurons are chosen by random and \"droped out\", hence the name. This can be seen in \ref{DropoutDiagram} \\
%Dropping out units from a neural network corresponds to creating a new neural network that shares the existing weights from the original network. After units have been dropped out the network is trained on the training data. This process is repeated and neurons are selected again by random. For a neural network consisting of $n$ neurons there are $2^n$ such possible thinned networks, of which each realization will rarely if at all be trained. The trained weigths will then be averaged resulting in a network of the original size.
%
%\begin{figure}
%	\centering
%	\includegraphics[width=6cm]{dropout.jpeg}
%	 \caption{Dropout Neural Net Model.}
%	 %taken from:http://blog.christianperone.com/2015/08/convolutional-neural-networks-and-feature-extraction-with-python/
%	 %ueberlegen andere grafik zu benutzen
% \label{DropoutDiagram}
%\end{figure}
%
%In a standard neural network using backpropagation the error function is minimized by using the influence of each parameter, leading to neurons adapting to one another and possibly compensating errors made by those neurons. This co-adaption leads to overfitting since it does not generalize to to unseen data. By randomly dropping out units this effect is suppressed. In their original paper Srivastava et. al. showed this by looking at the first level features of a neural network trained on the MNIST \cite{MNIST} with and without dropout. As can be seen in \ref{Dropout_coadption} features learned without dropout co-adapt to one another while the features learned with dropout are seperated more clearly.
%
%\begin{figure}
%	\centering
%	\includegraphics[width=\textwidth]{dropout-coadaption.png}
%	 \caption{Features learned on the MNIST data set with one hidden layer autoencoders having 256 rectified linear units. \cite{DropoutOriginal}}
%	 %taken from:http://blog.christianperone.com/2015/08/convolutional-neural-networks-and-feature-extraction-with-python/
%	 %ueberlegen andere grafik zu benutzen
% \label{Dropout_coadption}
%\end{figure}
%
%\subsection{Dropconnect}
%Paper:
%\begin{itemize}
%\item Regularization of neural networks using dropconnect
%\end{itemize}
%\begin{itemize}
%\item Further development of Dropout
%\item Instead of dropping neurons drop connections
%\item Slight improvement over dropout
%\end{itemize}
%
%\subsection{Batch Normalization}
%Paper:
%\begin{itemize}
%\item Batch Normalization: Accelerating Deep Network Training
%\end{itemize}
%
%
%\section{Optimization of Hyperparameters}
%Paper:
%\begin{itemize}
%\item Practical recommendations for gradient-based training of deep architectures
%\end{itemize}
%\begin{itemize}
%\item Two kinds of parameters:
%\subitem Parameters intrinsic to the model (model selection)
%\subitem Hyperparameters used for the training of the model
%\item Grid search of hyperparameters
%\subitem Logarithmic search
%\subitem Random search (curse of dimensionality)
%\end{itemize}

\bibliography{Citations,ImageNetCitation}{}
\bibliographystyle{plain}
\end{document}