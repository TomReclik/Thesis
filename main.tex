\documentclass[12pt,a4paper]{report}
\usepackage[utf8]{inputenc}
\usepackage[english]{babel}
\usepackage{amsmath}
\usepackage{amsfonts}
\usepackage{amssymb}
\usepackage{graphicx}
\usepackage{subcaption}\usepackage{bm}
\usepackage{booktabs}
\usepackage{float}
\usepackage[bookmarks]{hyperref}
\usepackage{enumitem}
\usepackage{multicol}
\usepackage{relsize}
\usepackage{mathtools}
\usepackage[font={small}]{caption}
\usepackage{fancyhdr}
\usepackage{gensymb}

\pagestyle{fancy} %eigener Seitenstil
\fancyhf{} %alle Kopf- und Fußzeilenfelder bereinigen
\fancyhead[R]{\smaller{\rightmark}} %Kopfzeile links
\fancyhead[C]{} %zentrierte Kopfzeile
\fancyhead[L]{\ifnum\value{chapter}>0 \chaptername\ \thechapter. \fi} %Kopfzeile rechts
\fancyfoot[C]{\thepage}


\DeclarePairedDelimiter\ceil{\lceil}{\rceil}
\DeclarePairedDelimiter\floor{\lfloor}{\rfloor}

\restylefloat{table}
\graphicspath{{./Images/}}
\newcommand\tab[1][0.5cm]{\hspace*{#1}}
\renewcommand{\baselinestretch}{1.5} 

\title{Application of Deep Convolutional Neural Network for the Classification of Damage Sites in Dual-Phase Steels}
\author{Tom Markus Reclik}

\def\boxit#1{%
  \smash{\fboxsep=0pt\llap{\rlap{\fbox{\strut\makebox[#1]{}}}~}}\ignorespaces
}

\begin{document}
%\maketitle
%\begin{abstract}
%To this day, various methods are used in order to reveal the micromechanical mechanisms of damage in materials. Post-mortem analysis at different stages of stress reveals only snapshots of the material, while in-situ methods are spatially limited to observing the evolution of only a few damage events. A limiting factor in all those methods is the amount of work involved in controlling the microscope and the image analysis. \\
%\tab[0.4cm] In this work, we implement different structures and architectures of neural networks for the localization and classification of damages. As a sample material dual-phase steels are chosen, due to the different responses of the ductile ferrite matrix and the brittle martensite islands to stress, resulting in the formation of damage sites belonging to distinct classes at early stages of deformation. \\
%\tab[0.4cm] The developed algorithms can on the one hand be used in order to automate the statistical evaluation of post-mortem micrographs, while on the other hand enabling in-situ experiments to generate statistically relevant data. Due to the computational nature of this method, a high throughput of data is possible, enabling a more complete understanding of failure mechanisms in many materials.
%
%
%
%
%
%%In material science, failure mechanisms in metals are classically studied based on the evolution of damages in micrographs observed at different stages of stress. The collection of statistically meaningful data is therefore an important step in the understanding of these processes. A limiting factor is the amount of work involved in the control of the microscope and image analysis. 
%%The aim of this master thesis is to automate the process of gathering data by employing machine learning algorithms. As a sample application, dual-phase steels are studied. Due to their light weight and high strength, they play an important role in the automotive industry. Consisting of a soft ferrite matrix with islands of hard martensite, different kinds of damages can emerge under stress, e.g. martensite islands can crack, damages between two ferrite grains can emerge, etc. 
%%In a first step, convolutional neural networks, and derivatives thereof, are used in order to find and classify damages. Due to the limited amount of relevant data, two challenges arise. On the one hand, data has to be collected experimentally and afterwards labeled by hand. On the other hand, a network architecture has to be found capable of achieving proper accuracy on small data sets. 
%%While the architecture to be used in the end has to be designed for the experimental data, a few preliminary studies will be performed on standard sets of images, namely the CIFAR-10 data set and the ImageNet data set.
%
%
%%Collecting statistically meaningfull data is limited by the large amount of work involved in the image analysis and control of the microscope. 
%%The aim of this master thesis is to automate this process and overcome those limitations by employing machine learning algorithms. Convolutional neural networks are used in order to classify different categories of damages. Specifically dual-phase steels will be studied in this thesis.
%%Dual-phase steel plays a prominent role in advanced high strength steel. Due to its light weight and high strength they are often used in the automotive industry.
%%Due to limited amount of relevant data, two challanges arise. On the one hand data has to be collected and labeled on the experimental side. On the other hand network architectures have to be found capable of achieving proper accuracy on small data sets. 
%%While the architecture to be used for the classification problem has to be designed on the experimental data, a few preliminary studies will be peformed on standard set of images, namely the CIFAR-10 data set and the ImageNet data set.
%
%%Dual phase steel plays a prominent role in advanced high strength steel. Its light weight and high strength make it interesting for fields like the automotive industry. 
%%On the mesoscopic scale different kinds of damages can emerge under uniaxial and biaxial tension. Gathering statisticly meaningful data is an important means for understanding (and optimizing?) its mechanical properties. Up to this point the process of collecting data is performed, either manually or additionally aided by computational means, by inspecting scanning electron microscope pictures. This poses a bottleneck for the collection of large volumes of data for later analysis. In this thesis we ask whether it is possible to automatize this process of locating and categorizing damages by means of artificial convolutional neural networks.
%%Approaching this task multiple problems emerge. Firstly there is only a low volume of data for the training of the constructed convolutional neural network. In a first effort of tackling this problem, we investigate the influence of the size of the training data on a sample data set, namely the CIFAR-10 \cite{CIFAR-10} data set. This gives us a first idea of what to expect in the later process and what network architectures are most fit to work on small data sets. Secondly damages are defined by their surroundings rather than by themselves, making this an unique challange to tackle. (Here later more about different architectures)
%
%%Different kinds of damages can emerge in dual phase steel under uniaxial and biaxial tension. Gathering statisticly meaningful data is an important means for understanding its mechanical properties. The process of collecting data to this point is performed either manually or partly assisted by computational means. In this thesis we ask whether it is possible to automatize this process of locating and categorizating damages by means of artificial neural networks. Solving this problem we face to problems. On the one hand a low volume of training data compared to classical categorization problems. Secondly damages are objects mostly defined by their surroundings rather than by their own properties.
%\end{abstract}
%\chapter{Introduction}

The continuous drive in the transportation industry to use lighter components that still meet safety regulations, for economical and ecological reasons, requires a deeper understanding of the underlying mechanisms responsible for the irreversible deformation and failure of materials. Due to the emerging complexity of macroscopic properties from microscopic features, ab initio theories only give crude approximations, and experiments are needed to observe the response of the microstructure to external deformation. By understanding the mechanisms behind the irreversible deformation and failure, new materials can be designed to exhibit desired mechanical properties.\\

%Due to the emerging complexity of the macroscopic properties of a material from their microstructure, experiments are needed, exploring the response of the microstructure to the deformation of the macroscopic material. A promising approach to improve the materials properties then is to reversely design a microstructure more resistant to the mechanisms behind irreversible deformations and failure. \\

%In materials with a heterogeneous microstructure, before the onset of failure under tensile stress, voids can nucleate inside the material. For the formation of a critical crack in the material and therefore its failure those voids act as sources for its nucleation and propagation. Understanding the damage mechanisms is therefore crucial in order to improve the materials resilience against failure, by e.g. engineering a microstructure, suppressing the predominant damage mechanism. A material improved in that way can on the one hand be used in order to meet the demand in modern transportation engineering by the reduction of a components weight resulting in lower fuel consumption and on the other hand in general create a component less prone to failure and therefore with a longer lifetime, resulting in lower costs in maintenance.\\

The deformation of a material leads to changes in the microstructure, e.g. emerging voids between different constituents in a material. Classifying those changes without computational aid leads to a focus on a few sites or extended periods of time required for the collection of statistically relevant data. Developing such a classification algorithm is therefore crucial for a deeper understanding of the deformation behavior. \\

%In heterogeneous materials its deformation can result in the formation of voids between the different constituents of the material on a microscopic level, which at higher levels of stress act as sources and propagators of a critical crack leading to the failure of the material. Because of the emerging properties of the macroscopic material from its microscopical features, statistically relevant data on the microstructure's response to the externally applied stress is needed. Evaluating such a statistically relevant dataset without an automated algorithmic aid is very time intensive. \\

While it is easy for humans to classify objects in images, it is extremely difficult to formulate the classification task into strict algorithmic rules for realistic problems. Machine learning algorithms are capable of closing this semantic gap, by classifying images based on extracted features, i.e. properties of the object represented in the image, either by hand engineering them, or by letting the machine learning algorithm learn the relevant features by themselves. \\

%While it is easy for a human to classify objects based on images, formulating strict algorithmic rules for a classification task is infeasible for realistic problems. Usually machine learning is used for classification tasks. There are generally two approaches. In the first features are hand engineered and extracted and then given to some machine learning algorithm, such as an decision tree actually classifying. The second approach is to let the algorithm directly learn the relevant features and classify the object using the features it learned by itself, such as in convolutional neural networks (CNNs). \\


%Gathering the necessary data is very time intensive, consisting of preparing and deforming the material, recording the micrograph, and evaluating these micrographs. The last step is the most promising point of attack for the reduction of needed human supervision using minimal resources. The evaluation itself consists of two steps, localization and classification of damage sites. Automating this process would enable the collection of statistically relevant data for ex-situ and in-situ experiments, and therefore the determination of ,for example, the predominant damage mechanism. Furthermore this makes it possible to track the evolution of a large number of damage sites in in-situ experiments through different stages of stress, gaining a deeper understanding of the physical processes of damage nucleation and growth.\\

%Currently localization is either done by searching manually or using some form of algorithmic rule together with a clustering algorithm reducing the overall space needed to be searched, while the classification has to be performed entirely by hand. This comes with two major problems. Firstly the amount of time needed in order to gather a dataset with statistical relevancy is unjustifiable and secondly the concentration while searching and classifying over long periods of time decreases, resulting in the increasing of the misclassification rate.\\

%The goal of this thesis is to automate this process. Therefore enabling an automatic collection of statistically relevant data for ex-situ and in-situ experiments, finding for example the predominant mechanism during deformation. Additionally making it possible to track the evolution of a large number of damage sites in in-situ experiments through different stages of stress, gaining a deeper understanding of the physical processes of damage nucleation and growth.\\

While other works in material science have implemented algorithms for the classification of damage sites by using hand tailored features \cite{Chen2013}, \cite{Zapata2011}, these methods are restricted to the specific material in use. The aim of this master thesis is to implement an algorithm applicable to a variety of materials. \\

%, e.g. Chen et al. extracted hand tailored features and used a decision tree for the localization and classification of engineering ceramic grinding surface damages \cite{Chen2013} and Zapata et al. similarly extracted features and classified using an artificial neural network (ANN) for the classification of weld defects in radiographic images \cite{Zapata2011}, the aim of this thesis is to implement an algorithm needing minimal preprocessing in order to be easily applicable for damage mechanisms in a variety of materials. \\

%While other works have implemented such an algorithm for certain materials, e.g. Chen et al. extracted hand tailored features and used a decision tree for the localization and classification of engineering ceramic grinding surface damages \cite{Chen2013} and Zapata et al. similarly extracted features and classified using an artificial neural network (ANN) for the classification of weld defects in radiographic images \cite{Zapata2011}, the aim of this thesis is to implement an algorithm needing minimal preprocessing in order to be easily applicable for damage mechanisms in a variety of materials. \\

%While other works have achieved this goal for certain materials, e.g. Chen et al. extracted hand tailored features and used a decision tree for the localization and classification of engineering ceramic grinding surface damages \cite{Chen2013} and Zapata et al. similarly extracted features and classified using an artificial neural network (ANN) for tWhile other works have implemented such an algorithm for certain materials, e.g. Chen et al. extracted hand tailored features and used a decision tree for the localization and classification of engineering ceramic grinding surface damages \cite{Chen2013} and Zapata et al. similarly extracted features and classified using an artificial neural network (ANN) for the classification of weld defects in radiographic images \cite{Zapata2011}, the aim of this thesis is to implement an algorithm needing minimal preprocessing in order to be easily applicable for damage mechanisms in a variety of materials. \\
%he classification of weld defects in radiographic images \cite{Zapata2011}, the aim of this thesis is to implement an algorithm needing minimal preprocessing in order to be easily applicable for damage mechanisms in a variety of materials. \\

Convolutional neural networks (CNNs) are a class of classification algorithms that need minimal to no preprocessing and are capable of automatically learning relevant features for the needed classification task. Furthermore they have proven to be reliable classification tools, e.g. winning the ImageNet Large Scale Visual Recognition Challenge \cite{imagenet_cvpr09} annually since 2012. Therefore they find heavy use in other fields of science and engineering, like in medicine for the automatic classification of skin cancer \cite{CNNSkinCancer}. \\

%In other fields of science and engineering it is commonplace to use convolutional neural networks (CNNs) for the task of object classification in images, e.g. the automatic classification of skin cancer by Esteva et al. \cite{CNNSkinCancer}. The advantage of CNNs is that they do not need preprocessing of inputs and hand engineered features. On common classification challenges like the ImageNet Large Scale Visual Recognition Challenge \cite{imagenet_cvpr09}, conducted annually since 2010, since 2012 CNNs take the first place each year. Showing that they are reliable classification tools. \\

Dual-phase steel was used as a sample material in this work, firstly because it is one of the most widely used advanced high-strength steels and secondly due to the fact that damage sites right after nucleation are clearly distinguishable on the surface of the material. The deformation behavior was investigated by applying uniaxial stress to the material and recording surface micrographs using a scanning electron microscope (SEM). Even though different methods exist that gain a deeper insight into the material than surface micrographs, e.g. electron backscatter diffraction, these methods are limited by the time required to record the necessary data. \\

The goal of this thesis is to automate the process of localizing and classifying damage sites in surface micrographs of dual-phase steels under stress, using a combination of machine learning algorithms and CNNs. This tool would enable the automated collection of statistically relevant data, for determining the predominant damage mechanism. Furthermore making it possible to track the evolution of a large number of damage sites, resulting in a deeper understanding of the underlying mechanisms. Ultimately this algorithm can be easily applied to the classification of damage sites in different materials, that appear as voids in the microstructure. \\

%Furthermore making it possible to aid the tracking of the evolution of damage sites, thereby 
%
% and also the automated tracking of the evolution of damage sites, gaining a deeper understanding into the underlying mechanisms, while also being easily applicable to classifying damage sites in materials, that show themselves as voids in the microstructure. \\

%The goal of this thesis is to automate the process of localizing and classifying damage sites in ex-situ and in-situ experiments using a combination of machine learning algorithms and CNNs. As a sample material dual-phase steel is used, since the damage mechanisms right after nucleation are clearly distinguishable from each other. But ultimately the created tool should be applicable to other materials as well. Due to the time intensity of data collection, one of the main challenges is the collection of a suitable dataset for the training of the CNNs. \\


%In the following ....
%\chapter{Dual-Phase Steel}
\label{cha:DualPhaseSteels}

\section{Steel}
Steel is one of the longest used materials in the history of mankind. Consisting of an alloy of iron and carbon, and other possible elements, it exhibits high ductility and hardness. The crystal structure of steel depending on the alloys temperature and carbon content are shown in figure \ref{fig:phaseDiagram}. At low temperatures and low carbon content the only thermodynamically stable configuration of the crystal structure is a body centered cubic structure, called ferrite. This phase exhibits a high ductility while being relatively soft. At higher temperatures the atoms in the crystal take the a face centered cubic structure, which is harder but less ductile. This phase is called austenite. While generating a fcc lattice at room temperature is adiabatically not possible, it is possible to start with austenite and rapidly cool it, during this process atoms do not have enough time to change the crystal structure completely, resulting in a tetragonal crystal structure. \\


Steel is one of the oldest materials used by mankind. It is an alloy consisting of iron and carbon, that can also contain other elements. Due to the wide variety of properties depending on the alloys composition, it finds many commercial applications. 


\begin{itemize}
\item Iron with carbon
\item Phase diagram
\item Ferrite
\item Martensite
\end{itemize}

\section{Dual-Phase Steel}
\begin{itemize}
\item Short introduction
\item Properties
\item Production
\end{itemize}



Dual-phase steels are a class of advanced high-strength steels. They combine a high ultimate tensile strength, a low initial yielding stress, high fracture strain, and a high hardening ratio. These advantageous properties together with their rather straightforward thermodynamical processing and light weight make them particularly interesting for the automotive industry, where they are mostly used for sheet metal parts. \\

%They exhibit multiple advantageous properties besides its rather straightforward thermodynamical processing.Combined with their low weight they are of particular interest in the automotive industry, where they are used for steel sheets. \\
%Dual-phase steels are a class of advanced high-strength steels. On a microstructural level they consist of a ductile ferrite matrix with islands of brittle martensite. Furthermore they can contain, austenite remaining from the processing of the material, pearlite, bainite carbides, and acicular ferrite. The hard martensite is responsible for the materials high ultimate strength, while the ductile ferrite results in a low initial yield strength. Those advantageous properties together with a high fracture strain, high hardening ratio, and their light weight makes them interesting from a engineering point of view, especially in the automotive industry, where dual-phase steels are used for steel sheets. \\


On a microstructural level dual-phase steels consist of a ductile ferrite matrix with islands of brittle martensite. While the microstructure is seemingly simple, its influence on the materials mechanical properties is not fully understood and research since the patent of dual-phase steels in 1968 \cite{dualphassteel} is still ongoing. The microstructural parameters include the martensite volume fraction, the size of martensite islands, and the overall morphology. The configuration of these parameters depends on the chemical composition and processing of the material. 

%On a microstructural level dual-phase steels consist of a ductile ferrite matrix with islands of brittle martensite. They furthermore can contain austenite, pearlite, bainite, carbides and acicular ferrites. While the microstructure of dual-phase steels is seemingly simple, consisting mostly of two While the microstructure of dual-phase steels is seemingly simple, its influence on the mechanical properties of dual-phase steels is only partially understood and research is ongoing PAPER. This is due to the fact that multiple microstructural parameters, like the martensite volume ratio, the size of martensite islands, the morphology, etc. play an important role in the macroscopic behavior. The microstructure at the surface of a sample is shown in figure \ref{fig:DPMicrostructure}. \\

\begin{figure}[H]
\centering
  \includegraphics[width=\linewidth]{DPMicrograph.png}
  \caption{Dual-phase steel SEM micrograph.}
  \label{fig:DPMicrostructure}
\end{figure}

\section{Damage Mechanisms}



 

%In dual-phase steels, with two constituents, these result in damage sites at the interface between ferrite and ferrite grains, called boundary decohesion, and at the interface between ferrite and martensite, called interface decohesion. Since the martensite is embedded in a ferrite matrix, grain boundaries between two martensite grains do not occur and therefore no voids accumulate at the grain boundary between two martensite grains. Due to the hardness of the martensite, cracks can occur in a single martensite grain, called martensite cracking. These are heavily influenced by the stress distribution and the geometry of the martensite grain.\\


%During deformation dual-phase steels firstly deform strictly elastically. If the applied stress is released the sample returns to its original state. Increasing the stress further up to the yield stress the material will be deformed permanently but upon releasing the applied stress it will return elastically to a different geometry. Beyond the yield stress damages nucleate inside the material permanently damaging the material. The yield stress together with the tensile strength and ultimate elongation are marked in the stress strain that can be seen in the appendix \ref{fig:DPStressStrain}.


%While under tensile stress, the material exhibits irreversible deformation at stresses above the yield point. A stress-strain curve is shown in figure \ref{fig:DPStressStrain}. Beyond this point voids can nucleate in the microstructure. These voids are classified based on their relative position to the constituents of the dual-phase steel. The classes are as follows
During deformation, materials exhibit different behaviors depending on the applied stress. At first, a material reacts elastically, returning to its original state when the stress is released. Increasing the stress further, the regime of plastic deformation is reached, and permanent changes to the microstructure of the material are introduced. Additionally to the plastic deformation, in dual-phase steels voids nucleate in the microstructure, becouse of the stress and strain partitioning between the two phases and their plastic incompatibility. These voids can act as sources and propagators of a critical crack leading to the failure of the material if the stress is further increased. The emerging voids in the materials microstructure are classified based on their relative position to the constituents of the dual-phase steel. A common classification is as follows
\begin{itemize}[label={}]
\item \textbf{Martensite Cracking}: A brittle crack nucleating inside a martensite island.
\item \textbf{Interface Decohesion}: Nucleation of a void at the interface between a martensite island and a ferrite grain.
\item \textbf{Boundary Decohesion}: Nucleation of a void at the grain boundary between two adjacent ferrite grains.
\end{itemize}
Furthermore, important for the materials behavior are foreign particles. These do not nucleate under stress but are introduced to the material during its processing. In SEM micrographs, these inclusions can also be seen as voids, albeit commonly of larger size.
\begin{itemize}[label={}]
\item \textbf{Inclusions}: A foreign particle remaining from the processing of the material. 
\end{itemize}
Abstracted examples of the possible damage mechanisms are shown in figure \ref{fig:DamageCategories_abstraction}. \\


%Due to the high concentration of martensite in the dual-phase steel used in this study, boundary decohesions occur rarely or not at all. In in-situ experiments, most of the sites classified at a certain stage, without knowledge of the previous state, as boundary decohesions turned out to be evolved notch effect sites. Examples for each damage category can be seen in figure \ref{fig:DamageCategories}.\\

\begin{figure}[H]
\begin{subfigure}{.25\textwidth}
\centering
  \includegraphics[width=.8\linewidth]{MartensiteCracking_abstraction_scalebar.pdf}
  \caption{MC}
  \label{fig:MC}
\end{subfigure}%
\begin{subfigure}{.25\textwidth}
\centering
  \includegraphics[width=.8\linewidth]{InterfaceDecohesion_abstraction_scalebar.pdf}
  \caption{ID}
  \label{fig:Interface_scalebar}
\end{subfigure}%
\centering
\begin{subfigure}{.25\textwidth}
\centering
  \includegraphics[width=.8\linewidth]{BoundaryDecohesion_abstraction_scalebar.pdf}
  \caption{BD}
  \label{fig:Inclusion_scalebar}
\end{subfigure}%
%\begin{subfigure}{.2\textwidth}
%\centering
%  \includegraphics[width=.8\linewidth]{NotchEffect_abstraction_scalebar.pdf}
%  \caption{NE}
%  \label{fig:Notch_scalebar}
%\end{subfigure}%
\centering
\begin{subfigure}{.25\textwidth}
\centering
  \includegraphics[width=.8\linewidth]{Inclusion_abstraction_scalebar.pdf}
  \caption{Inclusion}
  \label{fig:Inclusion_scalebar}
\end{subfigure}%
\caption{Abstracted damage mechanism for martensite cracking (MC) in subfigure (a), interface decohesion (ID) in subfigure (b), boundary decohesion (BD) in subfigure (c),  and inclusions in subfigure (d). The martensite islands, the ferrite matrix, and the foreign body are labeled as M, F, and I respectively.}
\label{fig:DamageCategories_abstraction}
\end{figure}

\section{Experimental Setup}

In order to study the mechanisms behind the deformation of a material and its failure, experiments have to be performed investigating the materials response to different kinds and levels of plastic strain under a certain stress state. By studying the response of the microstructure, specifically the formation of voids, a deeper understanding of the materials mechanic properties can be generated. While it is possible to use three dimensional imaging, these methods are limited by their time intensity and are usually limited to small portions of a probe. Another possibility is to use surface micrographs, these are limited by voids nucleating at or evolving to the surface of the specimen, but larger portions of the probe can be imaged. Furthermore two types of experiments can be performed, explained in the following. \\

\noindent \textbf{Ex-Situ}\\
In ex-situ experiments a sample is deformed, then metallographically prepared in order to be able to distinguish the constituents on the surface of the material. Afterwards, the surface of the specimen is scanned and investigated. This method is restricted to temporal snapshots of damage mechanisms. However, a view of the damage in the bulk of the material is possible, as material is ground during the metallographic preparation.\\

\noindent \textbf{In-Situ}\\
In in-situ experiments, the surface of the sample is prepared beforehand and the deformation experiment is performed afterwards with different stages of stress, enabling the study of the time evolution of damage sites. The downside to this method is the influence of the preparation of the materials surface on its deformation behavior. Furthermore, the nucleation and evolution of damage sites at the surface of the sample differs from the bulk properties as the stress states at th surface are altered. \\

The data used in this thesis stems from uniaxial tensile tests with surface micrographs from both ex-situ and in-situ experiments. Detail of the experimental setup can be found in the appendix \ref{app:experiment}.

%In order to investigate the mechanisms behind the deformation of a material and its failure, the material is put under stress and its response is investigated on a microstructural level. Such tests can for example involve putting a probe of the material uniaxial or biaxial tensile tests, or bending tests. The data recorded for this thesis stems from unixial tensile tests. While it is possible to investigate the microstructures response with three dimensional imaging, these methods are very time intensive and therefore restricted to a small portion of the probe. A simpler method is to record the surface of the probe using an electron microscope, restricting insights to damages either nucleating or evolving to the surface at the benefit of being able to investigating a larger portion of the probe. \\

%Furthermore it is possible to either deform a material and then prepare its surface in order to enable the distinction between its constituents at the cost of recording only snapshots of damage sites, or to prepare the material and record micrographs of the same material at different levels of stress at the cost of influencing the materials behavior through the preparation of its surface, but enabling the recording of the evolution of damage sites. These types of experiments are called ex-situ and in-situ experiments respectively. \\

%The details of the experiments are explained in the appendix \ref{app:experiment}



%\textbf{Preparation} \\
%Before the experiments were performed, the surface of the dual-phase steel was prepared in order to be able to distinguish the martensite islands from the ferrite matrix, in surface micrographs. The preparation process consists of grinding the surface with up to $4000$ grit sandpaper, polishing it with oxide polishing suspension in steps of $6\mu m$, $3\mu m$, and $1\mu m$, and finally etching it with $1 \%$ nital. \\
%
%\noindent \textbf{Probe Geometry} \\
%The probe geometry was chosen in such a way that the center of the probe experiences almost homogeneous stress. It can be seen in figure \ref{fig:ProbeGeometry}. \\
%
%\noindent \textbf{Tensile Tests}\\
%In order to study the damage mechanisms inside the materials microstructure, stresses were applied beyond its yield stress. \\
%
%\noindent \textbf{SEM Specifications} \\
%The surface of the material after deformation was recorded using an SEM. Its specifications and settings can be seen in table \ref{tab:SEM}. \\
%
%\begin{table}[H]
% \begin{center}
%  \begin{tabular}{@{} *2l @{}} \toprule[2pt]
%   Model & Zeiss LEO 1530 \\\midrule
%   Horizontal Field Width & $100\mu m$   \\ 
%   Vertical Field Width  & $75\mu m$ \\ 
%   Resolution  & $3072\times 2304$ \\
%   Detector Type & Secondary Electrons \\
%   Electron Source & Field-Emitter Cathode \\ \bottomrule[2pt]
%
%  \end{tabular}
% \end{center}
% \caption{Details of the SEM used for the surface micrographs.}
%   \label{tab:SEM}
%\end{table}

%
\chapter{Machine Learning for Image Analysis} % Main chapter title

\label{CNN} % For referencing the chapter elsewhere, use \ref{Chapter1} 

\section{Introduction}

Artificial neural networks (ANNs) are generally able to approximate any smooth function, as proven by Hornik et al. in 1989 \cite{Hornik1989}. In many cases the underlying function of a problem is unknown and the function can only be probed at certain points. ANNs can be trained on those samples and are particularly capable of finding the underlying non-linearities. Crucial for the performance of an ANN is firstly the collection of a representative training dataset. Once a sufficiently large dataset is available the selection of an appropriate architecture becomes more and more important. \\
%One crucial step in finding an ANN performing well for this task is the selection of an architecture and a representative training dataset. \\

In the following, the basic building blocks of ANNs the neurons will be introduced in section \ref{sec:Neurons}. Then the basic structure of ANNs is explained in section \ref{sec:ANN}. In section \ref{sec:LinearImageFilter} a motivation for the choice of a particular topology of ANNs is given using linear image filter. Afterwards from the methods used in linear image filters, a topology 


\section{Neurons}
\label{sec:Neurons}
The basic building block of every ANN is the neuron. For real valued inputs it maps $n: \mathbb{R}^n \rightarrow \mathbb{R}$ given internal parameters $w \in \mathbb{R}^d$ and $b \in \mathbb{R}$.
\begin{equation}
x \mapsto g(w \cdot x + b)
\end{equation}
with some non-linear activation function $g$. As can be seen from the its diagrammatic depiction in figure \ref{fig:Neuron} this closely resembles the action of its biological inspiration, which receives signals from adjacent neurons and fires if the internal potential exceeds its threshold. While this would be more accurately represented using a step function as the non-linear activation function, which was historically the first to be implemented by Rosenblatt \cite{Perceptron}, problems arise, e.g. the lacking of a training algorithm for neurons arranged in layers. Today in most deep architectures the most common choice for activation functions are rectifying linear units (ReLU) and derivatives thereof like the exponential linear unit (ELU) \cite{Clevert2015} or the leaky rectified linear unit (leakyReLU) \cite{Maas2013}.
\begin{figure}
\centering
  \includegraphics[width=0.9\linewidth]{Neuron.pdf}
  \caption{Perceptron}
  \label{fig:Neuron}
\end{figure}
%%The basic building block of each artificial neural network is the neuron. Given an input vector $x$ it performs a weighted sum and adds a bias
%%\begin{equation}
%%u_i = W_i \cdot x + b_i 
%%\end{equation}
%%afterwards applies a non-linear activation function
%%\begin{equation}
%%z_i = g(u_i)
%%\end{equation}
%%This closely resembles the inner workings of a biological neuron. It receives signals from adjacent neurons, the weights correspond to the dendrites connecting the neurons. If the potential inside the neuron exceeds a certain potential, here represented by a bias, the neuron fires. The firing of the neuron in the artificial case is performed by applying the activation function. While the closest analogy to a biological neuron would be by using a step function, this approach was discarded, mostly due to the lacking of an automated training algorithm for neurons arranged in a layered fashion. Today one of the most common activation function is the rectifying linear units and derivatives thereof.

\section{Feedforward Neural Networks}
\label{sec:ANN}
In feedforward neural networks neurons are arranged in a layered fashion. Each layer processes the output of its preceding layer, where the first layer constitutes the input to the network. Overall the network acts as a function $N: \mathbb{R}^n \rightarrow \mathbb{R}^m$ with internal parameters $\theta$, where the dimensionality of $\theta$ depends on the internal structure of the network.\\

The action of a layer $l$ can best be described by using matrix notation. 
\begin{equation}
W_l = 
\begin{pmatrix}
w_{1,1}^l & w_{1,2}^l & \dots & w_{1,n_{l-1}}^l \\
w_{2,1}^l & w_{2,2}^l & \dots & w_{2,n_{l-1}}^l \\
\vdots & \vdots & \vdots & \vdots \\
w_{n_l,1}^l & w_{n_l,2}^l & \dots & w_{n_l,n_{l-1}}^l \\
\end{pmatrix}
\end{equation}
where $w_i^l$ is the weight vector of neuron $i$ in layer $l$, $n_l$ is the number of neurons in layer $l$ and $n_{l-1}$ is the number of neurons in its previous layer. $W_l$ is matrix of real valued numbers of dimension $n_l \times n_{l-1}$. Furthermore the biases of all neurons in layer $l$ are collected in a bias vector. The mapping of layer $l$ can then be expressed as 
%By now also collecting the biases of all neurons in layer $l$ in a bias vector $b_l$.  and letting the layer dependent activation function act elementwise the action of a layer on its input can be defined as
\begin{equation}
x_{l-1} \mapsto g_l(W_l x_{l-1} + b_l)
\end{equation}
where $g_l$ is the layer dependent activation function that is applied elementwise to the input vector. A rather simple ANN is depicted in figure \ref{fig:ANN}. \\
\begin{figure}
\centering
  \includegraphics[width=0.8\linewidth]{SimpleNeuron-crop.pdf}
  \caption{An ANN with an input vector of $5$ elements, $3$ hidden layers, containing $6$ neurons each, outputting a single real number. Each edge corresponds to the weight connection between the connected neurons. }
  \label{fig:ANN}
\end{figure}

Feedforward networks are commonly used to approximate the mapping from some object represented in an input space to its corresponding class. Assuming such a mapping exists these networks are generally capable of approximating the mapping, as mentioned before. While this mapping can be realized using only one hidden layer with a large number of neurons, this approach is infeasible since each possible object in the input space has to be used in order to train the network. Choosing a topology for a neural network becomes crucial if the input space itself has in intrinsic topology. E.g. the information of an object represented in an image is not only represented by the pixel values but also their position inside the image.\\

\subsection{Weight Initialization}
The weights inside the network are usually initialized randomly. Choosing the right distribution and parameters is crucial for rate at which the network learns. For deep neural networks using sigmoid activation functions Xavier initialization is the most efficient while for ReLUs He initialization has proven to be the most efficient. The distribution from which the weights are chosen are the uniform distribution or the normal distribution. The details for the parameter initialization can be seen in table \ref{tab:initialization}

\begin{table}[H]
 \begin{center}
  \begin{tabular}{@{} *2l @{}} \toprule[2pt]
   Xavier Initialization\\\midrule
   Weights & Accuracy \\
   Bias & $85 \%$   \\ 
   Xavier Initialization\\\midrule
   Weights & Accuracy \\
   Bias & $85 \%$   \\ 
  \end{tabular}
 \end{center}
 \caption{Agreement for the classification of damage sites by hand. }
 \label{tab:Reliability}
\end{table}

\subsection{Training}
At first an error function $C$ has to be defined, that satisfies $C>0$ and is minimal only when the network returns the desired output. By doing so a measure is introduced for how far the prediction is from the truth. $C$ depends on all weights and biases in the network. By using backpropagation it is possible to calculate the contribution of each weight and bias to the prediction. 
%\section{Cost Functions}
%\begin{itemize}
%\item How a human learns - punishment?
%\item Define what is right and what is wrong
%\item Mathematical foundation to find optimal weights
%\item Properties of cost functions:
%\subitem $C>0$
%\subitem Output of network close to desired output then cost function close to minimum
%\end{itemize}
%\subsection{Squared Error}
%\begin{itemize}
%\item Linear regression
%\item Classical error function
%\end{itemize}
%\subsection{Categorical Cross Entropy}
%% http://neuralnetworksanddeeplearning.com/chap3.html#the_cross-entropy_cost_function
%\begin{itemize}
%\item Problems with squared error: small gradient
%\item Definition of cross entropy
%\item Using definition of sigmoid function shows that the system learns faster the further it is away from the true solution
%\end{itemize}

A naive approach to constructing a topology in a network would be for example to connect each neuron in one layer with every neuron in its preceding layer, a so called fully connected neural network. Applying this network to the classification of images comes with two major downsides. Firstly the number of trainable parameters grows quickly as the size of the image and the size of the network grows, e.g. classifying an image of size $256\times 256$ with $100$ neurons in the networks first layer would already lead to $6553600$ trainable parameters. Secondly this network is not inherently shift invariant and does not regard the topology of the input space. In the next section a motivation for a special kind of topology, a convolutional neural network, will be motivated using practices for image processing using discrete convolutions.

\section{Discrete Convolutions}\label{sec:DiscreteConvolutions}
For humans it is possible to decide to which object class it belongs. Therefore it is safe to assume that there exists an underlying function, mapping from the input space to the class space. By now adjusting the architecture of our network we hope that the constructed network will be more effective at approximating this function. 

While inspiration was originally taken from the inner workings of the visual cortex of mammals, based on the pioneering work D. Hubel and T. Wiesel \cite{Hubel1959}, in this work CNNs will be motivated by techniques used in image processing, specifically discrete convolutions. \\
%An image is represented as an array with shape $M\times N\times C$ with $M$, height $N$ and channels $C$. In an RGB image the number of channels equals $3$ representing the different colours, while in a black and white only one channel is present. Each element takes a value between $0$ and $255$ representing the intensity at its position inside the image. \\


A convolution is a linear operation acting on two functions $f$ and $g$, resulting in a new function. Commonly one of the functions is the convolution kernel, say $g$, and the resulting function is a modification of the original function $f$. The operation is given by
\begin{equation} \label{eq:convolutionContinuous}
(f*g)(x) = \int_{\mathbb{R}^n} f(x)g(x-y)dy
\end{equation}
where $f$ and $g$ act on $\mathbb{R}^n$. It finds applications in different fields of science, engineering, and pure mathematics. E.g. given a differential equation, $g$ can be the Green's function corresponding to the differential operator, $f$ are the initial conditions, then the convolved function is the solution of the differential equation at arbitrary coordinates. \\

While working with digitalized data, equation \ref{eq:convolutionContinuous} needs to be adjusted for functions acting on the discrete space $\mathbb{Z}^n$, resulting in
\begin{equation}\label{eq:convolutionDiscrete}
(f*g)(i_1,\dots ,i_n) = \sum_{j_1} \cdots \sum_{j_n} f(i_1,\dots ,i_n) g(i_1-j_1,\dots ,i_n-j_n)
\end{equation}
$g$ is often times restricted to have non-zero values only in a window of a certain size. For a two-dimensional object this is shown in figure \ref{fig:Convolution}. \\

In image processing discrete convolutions are used in order to find or amplify features in images. One example is the Sobel operator, which is used for the detection of edges in images. It approximates the gradient of the pixel values of an image. Assuming there is an underlying continuous function and an image $I$ is its discretization, the gradient in horizontal direction can be approximated by $G_x$ corresponding to edges in vertical direction and the derivative in vertical direction by $G_y$ corresponding to edges in horizontal direction. $G_x$ and $G_y$ are given by
\begin{align}
  \begin{split}
G_x =
\begin{pmatrix}
+1 & 0 & -1 \\
+2 & 0 & -2 \\
+1 & 0 & -1 \\
\end{pmatrix}
\end{split}
\begin{split}
G_y = 
\begin{pmatrix}
+1 & +2 & +1 \\
0 & 0 & 0 \\
-1 & -2 & -1
\end{pmatrix}
\end{split}
\end{align}
The overall gradient can then be calculated by
\begin{equation}
G = \sqrt{G_x^2+G_y^2}
\end{equation}
in the sense that $G_x$ and $G_y$ are used to convolve the original image, the resulting elements are squared, added, and the square root is applied elementwise. A sample application can be seen in figure \ref{fig:Sobel}. By doing so the relevant feature, the position of the edges, is extracted from the original image. \\
%In the next section adjustments to ANNs will be described in order to get a topology, that reflects the extraction of features in images using convolutions.
\begin{figure}
\centering
\begin{subfigure}{.5\textwidth}
  \centering
  \includegraphics[width=\linewidth]{Bike.png}
  \caption{Original image}
  \label{fig:sub1}
\end{subfigure}%
\begin{subfigure}{.5\textwidth}
  \centering
  \includegraphics[width=\linewidth]{Bike_Sobel.png}
  \caption{Edge amplified image}
  \label{fig:sub2}
\end{subfigure}
\caption{Edge amplification}
\label{fig:Sobel}
\end{figure}

\section{Convolutional Neural Networks}

Starting from a fully connected neural network the action of a discrete convolution can be replicated by rearranging the connections inside the neural network. At first this transformation will be described for the first layer of the network extracting only one feature for an image with only spatial dimensions. The generalization to an image with multiple channels and a layer extracting more than one feature then becomes straightforward. Having the action of one such convolutional layer a convolutional neural network can constructed. \\

In order to preserve the topology of the image the input now is an tensor $I_{m,n}$ with $m=0,\dots ,M-1$ and $n=0,\dots ,N-1$. The connections of a neuron therefore also becomes tensor $W_{m,n}^i$ with $m=0,\dots ,M-1$, $n=0,\dots ,N-1$, and $i$ being the index of the neuron. The bias of neuron $i$ is $b^i$. The output of neuron $i$ then is

\begin{equation}
v^i = g\left( \sum_{m=0}^{M-1} \sum_{n=0}^{N-1} I_{m,n} W_{m,n}^i + b^i \right)
\end{equation} \\
By then restricting the perceptive field of neuron $i=0$ to a window of size $M'\times N'$ anchored at $i'=0,j'=0$ its weight matrix takes the form

\begin{align}
\begin{split}
W^0 = 
\begin{pmatrix}
\tilde{W}^0 & \boldsymbol{0} \\
\boldsymbol{0} & \boldsymbol{0} \\
\end{pmatrix}
\end{split}
\begin{split}
\tilde{W}^0 = 
\begin{pmatrix}
\tilde{W}_{0,0}^0 & \tilde{W}_{0,1}^0 & \dots & \tilde{W}_{0,N'-1}^0 \\
\tilde{W}_{1,0}^0 & \tilde{W}_{1,1}^0 & \dots & \tilde{W}_{1,N'-1}^0 \\
\vdots & \vdots & \ddots & \vdots \\
\tilde{W}_{M'-1,0}^0 & \tilde{W}_{M'-1,1}^0 & \dots & \tilde{W}_{M'-1,N'-1}^0 \\
\end{pmatrix}
\end{split}
\end{align}
Creating copies of this neuron having the same weights but anchored at different positions, such that the entire image is covered, the action of a discrete convolution is recreated. Generalizing the input to have color channels or multiple features, the input takes the form of a tensor of rank $3$ $I_{m,n,c}$ and the weights of have to be adjusted as well to be tensors of rank $3$. The output of this layer now takes the form of equation \ref{eq:convolutionDiscrete}. Such a layer is then capable to learn, by adjusting its weights, what features to extract from its input. Furthermore multiple features can be extracted in each feature, by using multiple convolutions, which are independent from each other. \\

Using discrete convolutions comes with new hyperparameters to choose. These are listed in the following.

\subsubsection{Window Size}
One of the most important hyperparameters is the size of the convolution window $M' \times N'$. Usually $M' = N'$ is chosen. In a CNN layers can have different values of $N'$. Commonly these have the values of $1$ for bottleneck layers reducing the dimensionality inside the CNN \cite{Bottleneck}, $2,3,5,7$ where it has been argued that kernel sizes larger than $3$ should be replaced by subsequent kernels of size $3$ \cite{InceptionV3}. 
 
\subsubsection{Number of Convolutional Kernels}
The number of features $c$ to be extracted in each layer. For rather straightforward networks as the image size decreases during processing this fact is compensated by increasing the number of feature maps. 

\subsubsection{Padding}
Furthermore the treatment of the boundary of the input is of particular interest. The main ways are to either ignore the boundary, leading to an output smaller in size $M\times N \mapsto M-M' \times N-N'$, to expand the input with zeros (zero padding), with the outermost values (same padding), or to introduce periodic boundary conditions at the border (reflect padding).

\subsubsection{Stride}
The image can be covered by moving the convolution kernel by increments of $1$ in each direction or to move it in larger increments $s_1,s_2$ called the strides. Most commonly $s_1 = s_2 = s$ is chosen. Furthermore the stride is restricted by the size of the convolution kernel $s<N'$ such that the entire input is still covered. $s=1$ and $s=2$ are popular choices for the stride.

\subsection{Additional Layer}
Besides convolutional layers CNNs also consist of other layers. These will be described in the following.

\subsubsection{Pooling}
In between convolutional layers pooling layers are often used. These work similar to a convolutional layer except that the action of this layer is predetermined. Just like the convolutional layer pooling layers also have the hyperparameters window size $N^P_1 \times N^P_2$, and stride $s_1^P,s_2^P$, where again $N^P_1 = N^P_2 = N^P$ and $s_1^P=s_2^P=s^P$ are most commonly chosen. An input with the  dimensions $M \times N \times c$ is then mapped to  $M \mapsto \ceil{(M-N^P)/s^P} $ and $N \mapsto \ceil{(N-N^P)/s^P}$ while the channel dimension remains unchanged $c \mapsto c$.\\

The most popular choices for the pooling function are either to take the maximum value inside the window (max pooling) or to take the average. An example of max pooling for an input of size $6\times 4$ and $N^P = 2, s^P = 2$  can be seen in figure \ref{fig:MaxPooling}.\\

\begin{figure}[H]
\centering
\includegraphics[width=0.8\linewidth]{MaxPooling.pdf}
\caption{Max pooling}
\label{fig:MaxPooling}
\end{figure}

Pooling layers are used for multiple reasons, the main one being to downsample the input and save memory on the GPU. Furthermore by using pooling layers the network becomes increasingly invariant to spatial translations.

\subsubsection{Dropout}
Multiple approaches exist to prevent overfitting, e.g. introducing additional penalties in the error function, requiring the weights in the network not to grow indefinitely. The most popular approach nowadays is to use dropout during training. First introduced by Srivastava et. al. \cite{DropoutOriginal}, this approach selects a portion of neurons in each layer during training at random and removes them from the network temporarily. Dropping out units from a neural network corresponds to creating a new neural network that shares the existing weights from the original network. After units have been dropped out the network is trained on the training data. This process is repeated and neurons are selected again by random. For a neural network consisting of $n$ neurons there are $2^n$ such possible thinned networks, of which each realization will rarely if at all be trained. The trained weights will then be averaged resulting in a network of the original size. \\

\begin{figure}[H]
	\centering
	\includegraphics[width=0.8\linewidth]{dropout.jpeg}
	 \caption{Dropout Neural Net Model.}
	 %taken from:http://blog.christianperone.com/2015/08/convolutional-neural-networks-and-feature-extraction-with-python/
	 %ueberlegen andere grafik zu benutzen
 \label{DropoutDiagram}
\end{figure}

In a standard neural network using backpropagation the error function is minimized by using the influence of each parameter, leading to neurons adapting to one another and possibly compensating errors made by those neurons. This co-adaption leads to overfitting since it does not generalize to to unseen data. By randomly dropping out units this effect is suppressed. 

%In their original paper Srivastava et. al. showed this by looking at the first level features of a neural network trained on the MNIST \cite{MNIST} with and without dropout.


\subsubsection{Normalization}


\subsubsection{Softmax Layer}



\subsection{Training}
Having an networks architecture, its weights have to be initialized and adjusted. Initialization is performed usually by choosing the weights at random. For deep neural networks using ReLU's activation functions, this is done by using either a normal or uniform distribution with standard deviation $\sigma = \sqrt{2/n_in}$ where $n_in$ is the number of input units


Having a network architecture, its weights have to initialized. These is usually done by initializing them by random. For deep networks using ReLU's He initialization is an appropriate choice. 

%\subsection{He Initialization}
%Paper:
%\begin{itemize}
%\item On weight initialization in deep neural networks
%\item Delving Deep into Rectifiers: Surpassing Human-Level Performance on ImageNet Classification
%\end{itemize}
%\begin{itemize}
%\item https://arxiv.org/pdf/1502.01852.pdf
%\item Introduced in order to improve the performance of neural networks with rectified linear units
%\end{itemize}

\subsubsection{Cost Function}
In order to penalize a wrong prediction of the netwo
%\section{Cost Functions}
%\begin{itemize}
%\item How a human learns - punishment?
%\item Define what is right and what is wrong
%\item Mathematical foundation to find optimal weights
%\item Properties of cost functions:
%\subitem $C>0$
%\subitem Output of network close to desired output then cost function close to minimum
%\end{itemize}
%\subsection{Squared Error}
%\begin{itemize}
%\item Linear regression
%\item Classical error function
%\end{itemize}
%\subsection{Categorical Cross Entropy}
%% http://neuralnetworksanddeeplearning.com/chap3.html#the_cross-entropy_cost_function
%\begin{itemize}
%\item Problems with squared error: small gradient
%\item Definition of cross entropy
%\item Using definition of sigmoid function shows that the system learns faster the further it is away from the true solution
%\end{itemize}

%\subsection{Gradient Descent}
%\begin{equation}
%\Delta w_{ji}(n) = -\eta \frac{\partial E(n)}{\partial w_{ji}(n)}
%\end{equation}
%\begin{itemize}
%\item First order optimization technique
%\item Calculate local gradient of loss hyper surface
%\item Follow path of steepest descent
%\item Adjustable parameter: Learning rate $\eta$
%\end{itemize}


%
%\section{Convolutional Neural Networks - Architecture}
%
%Reflecting the topology of the input the neurons are now arranged in rectangular grid. 
%
%
%
%
%While at first each neuron is still connected to each input neuron, those connections are now restricted to a smaller window of size $m\times n$, in such a manner that the entire input is covered in a regular way. This can be seen in figure \ref{fig:CNN_topology}.
%Due to the special topology of images the neurons have to be rearranged as can be seen in figure \ref{fig:CNN_topology_notrestricted}. By restricting the connections of each neuron to a specific window of size $m\times n$, such that each neuron sees a different part of the original image and heavy weight sharing between all neurons, this layer of neurons performs a convolution on the original image. The result then is a feature map. In a CNN multiple such feature extractors are used in each layer. Additionally to the convolution operation, each to each output a bias is added and a non-linear activation function is applied.\\
%\begin{figure}
%\centering
%\begin{subfigure}{.5\textwidth}
%  \centering
%  \includegraphics[width=\linewidth]{CNN_topology.pdf}
%  \caption{Original image}
%  \label{fig:CNN_topology_notrestricted}
%\end{subfigure}%
%\begin{subfigure}{.5\textwidth}
%  \centering
%  \includegraphics[width=\linewidth]{CNN_topology_restricted.pdf}
%  \caption{Edge amplified image}
%  \label{fig:sub2}
%\end{subfigure}
%\caption{Edge amplification}
%\label{fig:Sobel}
%\end{figure}
%
%
%\begin{figure}
%\includegraphics[width=\linewidth]{CNN_topology_multiple_feature_maps.pdf}
%  \caption{Original image}
%  \label{fig:CNN_topology_notrestricted}
%\caption{Edge amplification}
%\label{fig:Sobel}
%\end{figure}
%
%\section{Different Layers Used In CNNs}
%While the convolutional layer is the most essential part of a network to be defined as a CNN, other layers are used in them that will be described in this section.
%
%\subsection{Pooling Layer}
%Usually a pooling layer follows a convolutional layer. A pooling layer takes, similarly to the convolutional layer, a rectangular as its input. It performs a predefined operation on this window, e.g. it returns the average or the maximum value inside the window. This results in a network that is shift invariant to minor local changes
%
%\subsection{Normalization Layer}
%Batch Normalization

\section{Performance Analysis}

The performance of a classifier can be assessed in multiple ways. In the following confusion matrices and 

\subsection{Confusion Matrix}
A confusion matrix is a matrix $c$ of shape $k\times k$ where $k$ is the number of possible classes. The elements of this matrix are the number of cases in which an object belonging to class $i$ was classified as class $j$. The diagonal $i=j$ are then objects correctly classified while the remainders are objects that have been misclassified. \\

From the confusion matrix the accuracy results from $ACC = \sum_i c_{i,i} / \sum_{i,j} c_{i,j}$.


\subsection{Purity vs Efficiency}

When using a classifier whose accuracy is not sufficient it is possible to increase the number of correctly classified objects at the cost that the classifier will not classify all instances. This is performed by introducing a threshold $\theta$ and only classify if the confidence of the network, the outputted probability like value, exceeds it. Purity is the achieved accuracy on the instances that were classified while efficiency is the ratio of actually classified objects. In figure \ref{fig:PURvsEFF} purity efficiency curves for a classifier working optimally, a classifier assigning classes at random, and a realistic classifier are shown. 

\begin{figure}
\begin{center}
\includegraphics[width=\linewidth]{ACCvsEFF_example.pdf}

\end{center}
\caption{PurityEfficiency}
\label{fig:PURvsEFF}
\end{figure}

\subsection{LIME}






















%\chapter{Preliminary Studies}

Before the necessary data for the training of a classifier, distinguishing between different damage mechanisms, was available, preliminary studies on different datasets were performed. The aim was to find an architecture performing well on a relatively small dataset distinguishing between up to five classes. Therefore the influence of the size of the training dataset, different network architectures, and architectural details were investigated. \\

\noindent The datasets were chosen from the most popular available datasets. \\
\noindent \textbf{CIFAR-10} \cite{Krizhevsky2009}: (Canadian Institute for Advanced Research 10) consisting of $60,000$ $32\times 32$ colour images distributed among 10 classes. \\
\noindent \textbf{ILSVRC} \cite{imagenet_cvpr09}: (ImageNet Large Scale Visual Recognition Challenge) consisting of over $14,000,000$ colour images of varying sizes distributed among over $20,000$ classes.

\section{CIFAR-10}

\subsection{Architectures}
Due to the small window size of the images in the CIFAR-10 datasets, some rather simple architectures were tested. The networks decided upon are \\
\noindent\textbf{Classical CNN}: A CNN with a few convolutional layers with a fully connected layer at the end. \\
\noindent\textbf{Graham simplified}: A CNN modeled after the winning CNN in the 2015 CIFAR-10 competition. Two convolutional layers alternate with a pooling layer, where the number of starts with $320$ and increases in each layer by $320$. In order to avoid excessive padding the number of layers was reduced. \\
\noindent\textbf{EERACN}: A CNN that uses convolutional layers followed by bottleneck layers, inspired by \cite{Xu2015}. \\
The details of each architecture can be seen in \ref{cha:Appendix_architectures}

\subsection{Influence of the Size of the Dataset}
Expecting to collect datasets with up to $1000$ examples per damage mechanism, $5$ categories of the CIFAR-10 dataset were chosen randomly and the networks performance was evaluated with up to $5000$ samples overall. The resulting accuracies can be seen in figure \ref{fig:Accuracy_Comparison_CIFAR10}. \\
\begin{figure}
  \includegraphics[width=\linewidth]{Accuracy_CIFAR10.pdf}
\caption{Accuracy on training sets of different sizes.}
\label{fig:Accuracy_Comparison_CIFAR10}
\end{figure}

While the classical CNN, performs comparably well for small datasets, its accuracy does not seem to improve much with an increasing amount of data. This is possibly caused by the rather small number of trainable parameters, with most trainable parameters in the fully connected last layer. The Graham simplified network, oscillates and has a rather low accuracy, probably caused by the large number of trainable parameters. The EERACN network has a smaller standard deviation from its mean and achieves an appropriate accuracy with an increasing trend for larger training sizes. With a similar amount of trainable parameters as the classical CNN, but located in its convolutional layers instead of the fully connected layer. \\

\section{ILSVRC}



%\chapter{Datasets}

A consistently labeled dataset is crucial for the performance of a classifier. Damage sites resulting from the same mechanism but classified differently will result in the network incorrectly adjusting its weights and rendering it useless. 


The datasets were obtained in two ways. At first labelImg \cite{labelImg} was used. With this tool it is possible to mark damage sites in SEM panoramas, requiring the user to search for them by zooming into the micrograph. Later once the localization algorithm was implemented as explained in chapter \ref{cha:Localization}, significantly reducing the required time for the creation of a dataset by negating the need to localize damage sites by hand. The two methods are shown in figure \ref{fig:labelimg}. \\

\section{Reliability of Training Data}

Due to the need of consistently labeled datasets, for the training process of the classifier, a study was performed internally beforehand. Five experts were asked to classify $25$ damage sites given the following descriptions. 
\begin{itemize}[label={}]
\item \textbf{Inclusion}: Inclusions, either holes left from preparation or actual inclusions.
\item \textbf{Martensite Cracking}: Brittle cracked martensite islands.
\item \textbf{Interface Decohesion}: Damage to the martensite/ferrite boundary.
\item \textbf{Boundary Decohesion}: Damage to ferrite grain boundaries.
\item \textbf{Evolved Damage}: More than one active damage mechanism, e.g. martensite cracking evolved into ductile damage in ferrite.
\end{itemize}
The agreement with the damage classes determined beforehand can be seen in table \ref{tab:Reliability}. Major problems arose with boundary decohesions and evolved damage sites. Due to the high concentration of martensite in the dual-phase steel used in this work boundary decohesions occur very rarely. Furthermore another internal in-situ study revealed that damage sites that appear to be boundary decohesions, are actually evolved notch effect sites. Problems with evolved damage sites come from the lack of a clear threshold after which a damage site falls into the class of evolved damage sites. Due to this, boundary decohesions and evolved damages were not used for the classification algorithm to distinguish between. Furthermore the class notch effect was introduced. \\

\begin{table}[H]
 \begin{center}
  \begin{tabular}{@{} *2l @{}} \toprule[2pt]
   Damage Category & Accuracy \\\midrule
   Martensite Cracking & $85 \%$   \\ 
   Inclusion  & $84 \%$ \\ 
   Interface Decohesion  & $76 \% $ \\
   Evolved & $56\%$ \\
   Boundary Decohesion & $44 \%$ \\ \bottomrule[2pt]

  \end{tabular}
 \end{center}
 \caption{Agreement for the classification of damage sites by hand. }
 \label{tab:Reliability}
\end{table}


\section{Datasets}
During deformation artifacts can form on the surface of the micrograph. A few examples of such artifacts are shown in figure \ref{fig:artifacts}. Due to the preparation of the sample these do not show in ex-situ experiments and very rarely in early stages of in-situ experiments. The dataset is therefore split into two categories. The first one consists of surface micrographs not containing artifacts on the probes set, while the second one consists of surface micrographs containing said artifacts. \\

The available data for each damage mechanism and category after labeling is shown in table \ref{tab:Dataset}.

%The data is split into two categories. The first category are surface micrographs that were taken right after preparing the sample. Due to the preparation these do not contain artifacts from the deformation of the sample. This first category consists of ex-situ experiments and the first stage of in-situ experiments. Later stages of in-situ experiments fall into the second category, 
%The data is split into three parts. Firstly data from ex-situ experiments, with different grades of stress. The dual-phase steel was prepared before each recording with the SEM, therefore no artifacts can be seen on the surface of the sheet. Secondly data from the first stage of in-situ experiments, this data is  The second round of experiments was performed in-situ. Due to the in-situ nature of those experiments at higher rates of deformation, recordings contain artifacts on the surface of the probe. The datasets with the number of damage sites in each category can be seen in table \ref{tab:Dataset}. \\ 

\begin{table}[H]
\begin{center}
\begin{tabular}{@{} *5l @{}} \toprule[2pt]
Training set &  \multicolumn{4}{c}{Damage Mechanism}   \\\midrule
 & Inc & MC & ID & NE   \\ 
ex-situ  & 379 & 691 & 788 & 449\\ 
in-situ  & 192 & 823 & 586 & 392 \\ \bottomrule
all  & 571 & 1514 & 1374 & 841\\\bottomrule[2pt]

\end{tabular}
 \caption{The number of damage sites found in the ex-situ and in-situ experiments per class (inclusion (Inc), martensite cracking (MC), interface decohesion (ID), and notch effect (NE)).}
 \label{tab:Dataset}
\end{center}
\end{table}
%\chapter{Architecture}

\label{cha:Architecture}

\section{Hierarchical Classifier}
Due to the small size of the training dataset, the parameters of the SEM are kept constant among experiments. This is done in order to prevent the need for a classifier to adapt to a changing environment while at the same time learning the relevant features, necessary for a distinction between the different damage mechanisms. Size differences between different damage mechanisms will therefore be coherent among the input of a CNN. The most distinct size difference can be seen between inclusions, and the remaining damage mechanisms, as indicated in figure \ref{fig:SizeDifference}. Using a single CNN for the classification of damage mechanisms, would come with the problem that a window size large enough to encompass an inclusion site, would lead to the other damage mechanisms being underrepresented in the input, while a window size on the typical scale of a martensite cracking, an interface decohesion, or a notch effect would not show a typical inclusion site in its entirety. \\

\begin{figure}[H]
\centering
\begin{subfigure}{.25\textwidth}
\centering
  \includegraphics[width=.8\linewidth]{Inclusion_scalebar.png}
  \caption{Inclusion}
  \label{fig:Inclusion_scalebar}
\end{subfigure}%
\begin{subfigure}{.25\textwidth}
\centering
  \includegraphics[width=.8\linewidth]{Martensite_scalebar.png}
  \caption{MC}
  \label{fig:Martensite_scalebar}
\end{subfigure}%
\begin{subfigure}{.25\textwidth}
\centering
  \includegraphics[width=.8\linewidth]{Interface_scalebar.png}
  \caption{ID}
  \label{fig:Interface_scalebar}
\end{subfigure}%
\begin{subfigure}{.25\textwidth}
\centering
  \includegraphics[width=.8\linewidth]{Notch_scalebar.png}
  \caption{NE}
  \label{fig:Notch_scalebar}
\end{subfigure}%
\caption{The different categories of damage sites, that the classifiers are trained to distinguish from each other together with a scale bar.}
\label{fig:SizeDifference}
\end{figure}

Therefore a hierarchical structure for the classification task at hand was chosen. Taking advantage of the size difference between inclusions and the remaining damage categories, the first classifier receives an input of a size of a typical inclusion site. It then classifies inclusion sites and passes the remaining sites to the next classifier. The second classifier has a smaller receptive field receiving only the relevant features to distinguish the remaining three classes, martensite cracking, interface decohesion, and notch effect. A diagrammatic depiction of this is shown in figure \ref{fig:Architecture}.\\

%Due to the inherent size difference between the different damage categories, as can be seen in figure \ref{fig:SizeDifference}, using a single CNN for the distinction between all possible damage mechanisms, comes with a major drawback. As the resolution is restrict to be the same experiments, in orde
%
%Due to the inherent size difference between the different damage categories, as can be seen in figure \ref{fig:SizeDifference}, using a single CNN for an ad-hoc classification, comes with one major drawback. Choosing a receptive field large enough to encase an entire inclusion, would lead to the void of other damage categories being underrepresented in the image. While choosing a receptive field of the typical size of a void of a category different than inclusion would be too small in order to show the relevant features of an inclusion.\\
%
%Therefore, instead of using a single CNN for the classification task, a hierarchical structure was chosen. Taking advantage of the size difference between inclusions and the remaining damage categories, the first classifier receives an input of a size of a typical inclusion site. It then classifies inclusion sites and passes the remaining sites to the next classifier. The second classifier has a smaller receptive field receiving only the relevant features to distinguish the remaining three classes, martensite cracking, interface decohesion, and notch effect. A diagrammatic depiction of this is shown in figure \ref{fig:Architecture}.\\

\begin{figure}
\begin{center}
  \includegraphics[width=\linewidth]{Architecture.png}
\caption{Comparison of the precision between networks trained excluding in-situ data and including in-situ data. There is a visible drop of the accuracy once the classifier trained on the ex-situ data is applied to the in-situ data to just above a random classifier.}
\label{fig:Architecture}
\end{center}
\end{figure}

%\begin{enumerate}
%\item Inclusions larger than rest of damage categories
%\item $\rightarrow$ use two networks first deciding whether the image contains an inclusion, second one decides whether it belongs to martensite cracking, interface decohesion or notch effect
%\item small amount of data
%\item $\rightarrow$ use a priori knowledge
%\item aq
%\end{enumerate}




%A network directly distinguishing between all classes, performed very poorly in earlier tests. This is mainly due to the small amount of data and the inherent size difference between inclusions and other types of damage sites. While taking a perceptive field large enough such that features of inclusions can typically be seen leads to a very small part of the perceptive field being relevant for the classification of other damage sites. \\
%By keeping the resolution of the microscope constant across different experiments this size difference can be taken advantage of, by using multiple CNNs working sequentially. Such an architecture can be seen in figure \ref{fig:Architecture}. The first network distinguishes whether the input shows an inclusion. If the network returns a value larger than some predetermined threshold $\theta_1$, for either belonging to the inclusion class or other damage mechanisms, the input is labeled as an inclusion or processed further respectively. If neither output exceeds $\theta_1$ the input is labeled as not classified and has to be labeled by hand. \\
%Before the second network processes th






%Inclusions are typically larger than other types of damage mechanisms. By keeping the resolution of the microscope constant, this can be taken advantage of by training two different CNNs with a different architecture and a different perceptive field. The first classifier then distinguishes between whether the damage site is an inclusion and passes it on to the next classifier it is not. This architecture together with internal confidence levels is shown in figure \ref{fig:Architecture}
%Due to the inherent size difference between inclusions and other types of damage mechanisms, the overall architecture consists of a first classifier deciding between inclusion sites with a larger perceptive field and a second classifier distinguishing between all other damage categories. 


\chapter{Evaluation and Results} % Main chapter title

\label{Performance} % For referencing the chapter elsewhere, use \ref{Chapter1} 

In this chapter firstly the performance of the two components of the architecture described in chapter \ref{cha:PracticalConsiderations} will be studied, as well as the performance of the combined classifier. In the end of this chapter the combined classifier is applied to the tracking of the evolution of damage sites in in-situ experiments, and its issues with generalization are discussed.

\section{Performance}
In the following the performance of the two components of the combined classifier will be studied. This is performed by using the methods introduced in \ref{sec:Metrics}, with baseline metrics referring to the classification without the use of confidence levels.

\subsection{First Classifier}

Due to the relatively small size of the datasets, optimization of a networks architecture is not promising for an increased performance. Instead three networks performing well on the ImageNet challenge \cite{imagenet_cvpr09} with input sizes similar to a typical inclusion site. The networks tested are the Xception network \cite{Xception}, the InceptionResNetV2 network \cite{InceptionResNetV2}, and the InceptionV3 network \cite{inception}. The achieved accuracies on the different datasets of each network can be seen in table \ref{tab:AccuracyComparisonNetworks}. Because the InceptionV3 network performed best, it was chosen as the architecture for the first CNN classifying inclusion sites. \\


%Due to the relative small size of the available data, optimization of a networks architecture is not promising to increase its performance. Therefore three networks were chosen that performed well on the ImageNet challenge \cite{imagenet_cvpr09} with input sizes comparable to the typical size of inclusion sites,  Their accuracies on the different datasets can be seen in table \ref{tab:AccuracyComparisonNetworks}. Because the InceptionV3 Network performed best on the available data it was chosen for the classification of inclusion sites.

%As explained in chapter \ref{cha:Architecture}, the first network of the overall architecture is trained to distinguish between inclusions and all other possible damage mechanisms. The networks were trained for $70$ epochs using an Adam optimizer with a learning rate of $0.001$, $\beta_1=0.9$, $\beta_2=0.999$, $\epsilon=10^{-8}$, and a learning rate decay of $0$. The dataset is split into $80\%$ used for training and $20\%$ for testing. 

%\subsection{Architecture}

%Due to the small amount of available data, optimization of a networks architecture is not promising to increase its accuracy. Therefore three networks were chosen that performed well in the ImageNet challenge with similar window sizes, namely the Xception network \cite{Xception}, the InceptionResNetV2 network \cite{InceptionResNetV2}, and the InceptionV3 network \cite{InceptionV3}. Their accuracies can be seen in table \ref{tab:AccuracyComparisonNetworks}. Since the InceptionV3 network performed best it was chosen for the first classifier.\\

\begin{table}[H]
 \begin{center}
  \begin{tabular}{@{} *5l @{}} \toprule[2pt]
   Network &  \multicolumn{3}{c}{Accuracy}  \\\midrule
    & ex-situ  & in-situ  & all   \\ 
   Xception  & 0.868 & 0.878 & 0.866\\ 
   InceptionResNetV2  & 0.854 & 0.849 & 0.888\\
 \boxit{8.46cm}   InceptionV3 & 0.901 & 0.863 & 0.915 \\ \bottomrule[2pt]

  \end{tabular}
 \end{center}
 \caption{The accuracies of the different network architectures.}
   \label{tab:AccuracyComparisonNetworks}
\end{table}

\subsubsection{Ex-Situ Data}
Training and evaluating the network on the ex-situ dataset results in a baseline accuracy of $0.947\pm 0.008$ without the use of a confidence level. In figure \ref{fig:InceptionExSituCLAvsACC} the accuracy, the true positive and true negative rate depending on the classification are shown. From this plot it can be seen that:
\begin{itemize}
\item The accuracy for the classification of sites not containing inclusion sites is higher than the accuracy when classifying inclusion sites. 
\item The classification rate does not drop below $50\%$ indicating that the network output distribution favors values close to $1$ or $0$.
\end{itemize}

\begin{figure}[H]
\includegraphics[width=\textwidth]{/FirstClassifier/InceptionExSituCLAvsACC_TPR_FPR_withLines_corrected.pdf}
\caption{The precision plotted against the efficiency, for the inception network trained and evaluated on the ex-situ dataset.}
\label{fig:InceptionExSituCLAvsACC}
\end{figure}

The baseline confusion matrix is shown in table \ref{tab:FirstClassifierConfusionMatrixExSitu}. Out of $76$ inclusion sites $66$ were correctly identified, and out of $75$ other sites $70$ were correctly identified as not being inclusions. In the following samples of damage sites are shown.

\begin{table}
 \begin{center}
  \begin{tabular}{@{} *3l @{}} \toprule[2pt]
   predicted &  \multicolumn{2}{c}{true}  \\\midrule
    & Inclusion  & Rest   \\ 
   Inclusion  & 66 & 5 \\ 
   Rest  & 10 & 70 \\ \bottomrule[2pt]
   \label{tab:FirstClassifierConfusionMatrixExSitu}
  \end{tabular}
 \end{center}
 \caption{Confusion matrix.}
\end{table}

In figure \ref{fig:InceptionExSituPredictedIncTrueRest} damage sites incorrectly classified as inclusion sites are shown, together with the class probability for the assigned class returned by the network. In figure \ref{sub:MasInc} a martensite cracking is shown. In figure \ref{sub:IDasInc1} an interface decohesion site is shown. In figure \ref{sub:IDasInc2} a notch effect site is shown. In the bottom left an inclusion site can be seen, leading the classifier to conclude that the excerpt depicts an inclusion site. 

%In figure \ref{fig:InceptionExSituPredictedIncTrueRest} sites are shown that the classifier misclassified damage sites that stem from martensite cracking, interface decohesion, or notch effect. The damage site shown in \ref{sub:IDasInc1} shows itself as a large void, probably leading the network to assign the class inclusion. In the immediate surrounding of the damage site shown in figure \ref{sub:NasInc} is an inclusion in the lower left part, leading to the classifier interpreting the input as belonging to a site showing an inclusion. For the two damage sites \ref{sub:MasInc} and \ref{sub:IDasInc2} a justification of the assignment of the class inclusion is difficult. \\


\begin{figure}[H]
\centering
\begin{subfigure}{.3\textwidth}
\includegraphics[width=0.8\linewidth]{/FirstClassifier/FirstClassifier_ExSitu_P0_T1_0_93039989.png}
\caption{$0.930$}
\label{sub:MasInc}
\end{subfigure}
\centering
\begin{subfigure}{.3\textwidth}
\includegraphics[width=0.8\linewidth]{/FirstClassifier/FirstClassifier_ExSitu_P0_T2_0_92554379.png}
\caption{$0.926$}
\label{sub:IDasInc1}
\end{subfigure}
\centering
\begin{subfigure}{.3\textwidth}
\includegraphics[width=0.8\linewidth]{/FirstClassifier/FirstClassifier_ExSitu_P0_T3_0_75197071.png}
\caption{$0.742$}
\label{sub:IDasInc2}
\end{subfigure}
\caption{Damage sites incorrectly classified as inclusion sites together with the networks confidence. In (a) a martensite cracking site is shown, in (b) an interface decohesion site is shown, and in (c) a notch effect site is shown.
 In figure (a) a martensite cracking is shown classified as an inclusion, in figure (b) interface decohesion is shown classified as an inclusion, and in figure (c) a notch effect site is shown classified as an inclusion.} 
\label{fig:InceptionExSituPredictedIncTrueRest}
\end{figure}

In figure \ref{fig:InceptionExSituPredictedRestTrueInc} sites not recognized as inclusion sites are shown, together with the class probability for the assigned class returned by the network. The inclusion site shown in figure \ref{sub:IncAsRest_1} looks similar to an interface decohesion site together with a martensite cracking. In figures \ref{sub:IncAsRest_2} and \ref{sub:IncAsRest_3} two inclusion sites are shown without the characteristic surrounding voids. 
%In figure \ref{fig:InceptionExSituPredictedRestTrueInc} damage sites are shown that are inclusions but have not been recognized by the network as such. As can be seen for figure \ref{sub:IncAsRest1} and \ref{sub:IncAsRest4} the void is at the edge of the window. The damage sites shown in figures \ref{sub:IncAsRest2} and \ref{sub:IncAsRest3} the included foreign particle resembles a martensite grain closely. \\

\begin{figure}[H]
\centering
\begin{subfigure}{.3\textwidth}
\includegraphics[width=0.8\linewidth]{/FirstClassifier/FirstClassifier_ExSitu_P1_T0_0_6985262.png}
\caption{$0.699$}
\label{sub:IncAsRest_1}
\end{subfigure}
\centering
\begin{subfigure}{.3\textwidth}
\includegraphics[width=0.8\linewidth]{/FirstClassifier/FirstClassifier_ExSitu_P1_T0_0_92622524.png}
\caption{$0.926$}
\label{sub:IncAsRest_2}
\end{subfigure}
\centering
\begin{subfigure}{.3\textwidth}
\includegraphics[width=0.8\linewidth]{/FirstClassifier/FirstClassifier_ExSitu_P1_T0_0_99065292.png}
\caption{$0.742$}
\label{sub:IncAsRest_3}
\end{subfigure}
\caption{Inclusion sites not recognized by the classifier.}
\label{fig:InceptionExSituPredictedRestTrueInc}

%In figure (a) a martensite cracking is shown classified as an inclusion, in figures (b) and (c) interface decohesions are shown classified as inclusions, and in figure \ref{sub:NasInc} a notch effect site is shown classified as an inclusion}. \\
\end{figure}

%\begin{figure}
%\centering
%\begin{subfigure}{.24\textwidth}
%\includegraphics[width=0.8\linewidth]{/FirstClassifier/Ex-situ_predicted1_true0_0.png}
%\caption{}
%\label{sub:IncAsRest1}
%\end{subfigure}
%\centering
%\begin{subfigure}{.24\textwidth}
%\includegraphics[width=0.8\linewidth]{/FirstClassifier/Ex-situ_predicted1_true0_4.png}
%\caption{}
%\label{sub:IncAsRest2}
%\end{subfigure}
%\centering
%\begin{subfigure}{.24\textwidth}
%\includegraphics[width=0.8\linewidth]{/FirstClassifier/Ex-situ_predicted1_true0_6.png}
%\caption{}
%\label{sub:IncAsRest3}
%\end{subfigure}
%\centering
%\begin{subfigure}{.24\textwidth}
%\includegraphics[width=0.8\linewidth]{/FirstClassifier/Ex-situ_predicted1_true0_7.png}
%\caption{}
%\label{sub:IncAsRest4}
%\end{subfigure}
%\caption{Different misclassified damage sites. All these sites were not recognized by the classifier as inclusions.}
%\label{fig:InceptionExSituPredictedRestTrueInc}
%\end{figure}

\subsubsection{Generalization to In-Situ Data}

While working only with the data recorded ex-situ, the network generalized well on this dataset, the networks performance drops significantly to just above a classifier randomly assigning class labels, the accuracy on the different datasets can be seen in table \ref{tab:AccuracyComparisonInception}. By including the in-situ data into the training dataset, the networks performance was increased drastically. \\

\begin{table}[H]
 \begin{center}
  \begin{tabular}{@{} *5l @{}} \toprule[2pt]
   Training set &  &Test set&  \\\midrule
    & ex-situ  & in-situ  & all   \\ 
   ex-situ  & 0.90 & 0.55 & 0.71\\ 
   all  & 0.94 & 0.90 & 0.92\\\bottomrule[2pt]

  \end{tabular}
 \end{center}
 \caption{Accuracy of the networks trained on different training sets evaluated on the available datasets.}
 \label{tab:AccuracyComparisonInception}
\end{table}


%\begin{figure}
%  \includegraphics[width=\linewidth]{Inception_ex_vs_in.pdf}
%\caption{Comparison of the precision between networks trained excluding in-situ data and including in-situ data.}
%\label{fig:Inception_ex_vs_in}
%\end{figure}

\subsubsection{Overall Performance}

Training and evaluating the network on the ex-situ dataset results in a baseline accuracy of $0.948\pm 0.006$. While this accuracy is close to the accuracy on the ex-situ dataset, the behavior of the accuracy, true positive and negative rates are different as can be seen in figure \ref{fig:FirstClassifierInSitu}. From this plot it can be seen that:
\begin{itemize}
\item The accuracy for the classification of sites not containing inclusion sites is higher than the accuracy when classifying inclusion sites. 
\item The classification rate does not drop below $82\%$ indicating that the network output distribution favors values close to $1$ or $0$.
\end{itemize}
The different metrics of the first classifier are shown in table \ref{tab:FirstClassifierMetrics}.

%The different metrics of the first classifier are shown in table \ref{tab:FirstClassifierMetrics} for the classification using the classes with the highest probability ($\theta =0$) and for the classification using a confidence level of $\theta = 0.7$.

\begin{table}[H]
 \begin{center}
  \begin{tabular}{@{} *3l @{}} \toprule[2pt]
   Threshold & Metric &   \\ \midrule
   $\theta=0$ & ACC & $0.948\pm0.06$ \\
   &TPR  & $0.84\pm 0.06$ \\
   &TNR  & $0.98\pm 0.01$ \\
   &PPV  & $0.90\pm0.04$ \\
   &NPV  & $0.97\pm0.01$ \\ \midrule
   $\theta=0.7$& ACC & $0.956 \pm 0.006$ \\
   &TNR  & $0.95\pm 0.09$ \\
   &TPR  & $0.85\pm 0.07$ \\
   &PPV  & $0.92\pm 0.04$ \\
   &NPV  & $0.97 \pm0.01$ \\ 
   &CLA  & $0.983\pm 0.007$ \\ \bottomrule[2pt]
  \end{tabular}
 \end{center}
 \caption{{The different achieved metrics for the case of classification using the classes with the highest probability and a confidence level of $\theta =0.7$. The true positive rate (TPR), true negative rate (TNR) positive predictive value (PPV), negative predictive value (NPV), the accuracy (ACC), and the classification rate (CLA) are shown.}}
   \label{tab:FirstClassifierMetrics}
\end{table}

\begin{figure}[H]
\includegraphics[width=\textwidth]{/FirstClassifier/InceptionInSituCLAvsACC_TPR_FPR_withLines_corrected.pdf}
\caption{Precision vs Recall together with the true positive rates for each class. }
\label{fig:FirstClassifierInSitu}
\end{figure}

The baseline confusion matrix is shown in table \ref{tab:FirstClassifierConfusionMatrixInSitu}. Out of $113$ inclusion sites $94$ were correctly identified, and out of $111$ other sites $107$ were correctly identified as not being inclusions. In the following samples of damage sites are shown.

\begin{table}[H]
 \begin{center}
  \begin{tabular}{@{} *3l @{}} \toprule[2pt]
   predicted &  \multicolumn{2}{c}{true}  \\\midrule
    & Inclusion  & Rest   \\ 
   Inclusion  & 94 & 4 \\ 
   Rest  & 19 & 107 \\ \bottomrule[2pt]
  \end{tabular}
 \end{center}
 \caption{Confusion matrix.}
   \label{tab:FirstClassifierConfusionMatrixInSitu}
\end{table}

In figure \ref{fig:FirstClassifierInSituP0T1} damage sites incorrectly classified as inclusions can be seen, together with the class probability for the assigned class returned by the network. In figure \ref{sub:InSituP0T1} a martensite cracking was classified as an inclusion site. Due to the high damage site density in the input of the network, different damage sites, than the one intended (in the middle of the window), influence the prediction of the network. In figure \ref{sub:InSituP0T2} an interface decohesion site was classified as an inclusion site, similar to \ref{sub:InSituP0T1}, multiple damage sites are in the input, influencing the networks prediction. Lastly, in figure \ref{sub:InSituP0T3}, a notch effect site was classified as an inclusion site. It can be seen that that the martensite grains are very bright in the image. These bright spots are characteristic for inclusion sites.

\begin{figure}[H]
\centering
\begin{subfigure}{.3\textwidth}
\includegraphics[width=0.8\linewidth]{/FirstClassifier/FirstClassifier_InSitu_P0_T1_0_958.png}
\caption{0.958}
\label{sub:InSituP0T1}
\end{subfigure}
\centering
\begin{subfigure}{.3\textwidth}
\includegraphics[width=0.8\linewidth]{/FirstClassifier/FirstClassifier_InSitu_P0_T2_0_999.png}
\caption{0.999}
\label{sub:InSituP0T2}
\end{subfigure}
\centering
\begin{subfigure}{.3\textwidth}
\includegraphics[width=0.8\linewidth]{/FirstClassifier/FirstClassifier_InSitu_P0_T3_0_654.png}
\caption{0.654}
\label{sub:InSituP0T3}
\end{subfigure}
\caption{Damage sites incorrectly classified as inclusions, together with the networks confidence. In (a) a martensite cracking site is shown, in (b) an interface decohesion site is shown, and in (c) a notch effect site is shown.}
\label{fig:FirstClassifierInSituP0T1}
\end{figure}

In figure \ref{fig:FirstClassifierInSituP1T0} damage sites incorrectly classified as damage mechanisms other than inclusions can be seen, together with the class probability for the assigned class returned by the network. While the damage site shown in figure \ref{sub:InSituP1T0_1} was not recognized as an inclusion by the classifier, the confidence in its prediction is just above $0.5$, i.e. just above the threshold for a two class problem without a confidence level. By using confidence levels this site would be not classified, needing classification by hand but preventing errors by the network. In figure \ref{sub:InSituP1T0_2} the inclusion site shows itself without the characteristic dark surroundings. The inclusion site shown in figure \ref{sub:InSituP1T0_3} is atypical due to its small size.\\
\begin{figure}[H]
\centering
\begin{subfigure}{.3\textwidth}
\includegraphics[width=0.8\linewidth]{/FirstClassifier/FirstClassifier_InSitu_P1_T0_0_540.png}
\caption{0.540}
\label{sub:InSituP1T10_1}
\end{subfigure}
\centering
\begin{subfigure}{.3\textwidth}
\includegraphics[width=0.8\linewidth]{/FirstClassifier/FirstClassifier_InSitu_P1_T0_0_797.png}
\caption{0.797}
\label{sub:InSituP1T0_2}
\end{subfigure}
\centering
\begin{subfigure}{.3\textwidth}
\includegraphics[width=0.8\linewidth]{/FirstClassifier/FirstClassifier_InSitu_P1_T0_0_999.png}
\caption{0.999}
\label{sub:InSituP1T0_3}
\end{subfigure}
\caption{Inclusion sites not recognized by the classifier, together with the networks confidence.}
\label{fig:FirstClassifierInSituP1T0}
\end{figure}

By using LIME as discussed in \ref{sec:LIME} the decisive regions for the classification of a damage site can be found. In figure \ref{fig:InSituP1T0_2_LIME} it can be seen that while the classifier correctly identified the inclusion, other influences lead to the misclassification. \\

%\begin{figure}[H]
%\centering
%\includegraphics[width=\textwidth]{/FirstClassifier/FirstClassifier_InSitu_P0_T1_0_958LIME.png}
%\caption{0.540}
%\label{fig:InSituP0T1_LIME}
%\end{figure}

\begin{figure}[H]
\centering
\includegraphics[width=\textwidth]{/FirstClassifier/FirstClassifier_InSitu_P1_T0_0_797LIME.png}
\caption{0.797}
\label{fig:InSituP1T0_2_LIME}
\end{figure}
In figure \ref{sub:InSituP1T10_1_LIME} part of the inclusion was recognized by the classifier. The foreign particle itself was recognized as not belonging to the inclusion.
%In figure \ref{sub:InSituP1T10_1_LIME} as discussed earlier the bright regions of the micrograph were decisive for the classification of this damage site as an inclusion
%\begin{figure}[H]
%\centering
%\includegraphics[width=\textwidth]{/FirstClassifier/FirstClassifier_InSitu_P1_T0_0_999LIME.png}
%\caption{0.999}
%\label{fig:InSituP1T0_3_LIME}
%\end{figure}
%
\begin{figure}
\centering
\includegraphics[width=\textwidth]{/FirstClassifier/FirstClassifier_InSitu_P1_T0_0_540LIME.png}
\caption{}
\label{sub:InSituP1T10_1_LIME}
\end{figure}
%%
%\begin{figure}
%\centering
%\includegraphics[width=\textwidth]{/FirstClassifier/FirstClassifier_InSitu_P1_T0_0_797LIME.png}
%\caption{}
%\label{sub:InSituP1T0_2_LIME}
%\end{figure}
%\begin{figure}
%\centering
%\includegraphics[width=\textwidth]{/FirstClassifier/FirstClassifier_InSitu_P1_T0_0_999LIME.png}
%\caption{}
%\label{sub:InSituP1T0_3_LIME}
%\end{figure}


%\begin{figure}
%
%\end{figure}


%\subsection{Choosing a Threshold}
%The output of the network can be seen as a probability that a certain damage site belongs to a damage class. By choosing a threshold above which the classifier should classify a given damage site, a trade-off has to be made between the desired accuracy and the classification rate. In figure \ref{fig:InceptionACC_EFF_THETA} the accuracy of the first network together with its classification rate plotted against the threshold can be seen. Requiring the accuracy to be $95\%$ would correspond to a threshold of $\theta = 0.7$ and a classification rate of $92\%$. The remaining $8\%$ of damage sites have to be labeled afterwards by hand. 
%
%
%%Due to the first classifier acting as a filter for the next classifier, it is necessary to minimize the number of falsely classified damage sites. By introducing a threshold for the network to only decide for a damage site to be of a category if the returned probability exceeds it, the number of falsely classified damage sites can be minimized, at the cost of a lower efficiency. By labeling the not classified damage sites by hand, those can be later introduced back into the system as new training data, representing points in the input space not learned by the network. The accuracy together with the efficiency against the threshold are shown in figure \ref{fig:InceptionACC_EFF_THETA}. A reasonable choice for this threshold is $\theta=0.7$, resulting in an accuracy of $95\%$ classifying $92\%$ of all damage sites. 
%
%\begin{figure}
%  \includegraphics[width=\linewidth]{Inception_ACC_CLA_THETA.pdf}
%\caption{Accuracy of classified sites plotted together with the ratio of classified sites against the threshold. At $\theta=0.7$ of the $92\%$ classified damage sites $95\%$ were classified correctly.}
%\label{fig:InceptionACC_EFF_THETA}
%\end{figure}

%\subsubsection{Comparison}
%In figure \ref{fig:TPR_comparison} the precision of both differently trained networks are shown. As one can see the network performed well, while working only with ex-situ data. Transferring the network to be used on in-situ data the networks precision dropped immensely, rendering the network useless for classifying in-situ images. However, by including a small portion of in-situ data in the training of the network, the number of damage sites wrongly classified by the network as inclusions becomes negligible above a certain threshold. 

\newpage
\subsection{Second Classifier}
Due to the smaller nature of the remaining classes, compared to inclusions, for the second classifier a simpler network architecture was decided upon. The networks architecture can be found in \ref{app:Architecture}.


%In order to compare the performance of such a simple architecture to the performance of a complex architecture, the Inception network, their respective precision recall plots are shown in \ref{fig:InVsEE}.  As can be seen for these smaller sites the naive architecture outperforms the InceptionV3 network. It was therefor decided upon to use the EERACN network as described in ... .
%
%
%\begin{figure}[H]
%  \includegraphics[width=\linewidth]{InceptionVsEERACN_all.pdf}
%\caption{Performance comparison between the InceptionV3 network and the EERACN network distinguishing between brittle and ductile mechanisms}
%\label{fig:InVsEE}
%\end{figure}

\subsubsection{Ex-Situ Data}
The network trained on the ex-situ dataset evaluated on the ex-situ dataset resulted in an accuracy of $0.71$. The networks accuracy and true positive rates of each class are shown in figure \ref{fig:EERACNExSituCLAvsACC_allClasses}. From this plot it can be seen that:
\begin{itemize}
\item The networks true positive rate rate for martensite cracking is higher than the remaining classes.
\item There is a drop in the true precision rate of notch effects. This drop results from the small number of notch effects.
\item The minimum classification rate is $19.5\%$. 
\end{itemize}

%The network trained on the ex-situ dataset evaluated on the ex-situ dataset resulted in an accuracy of $0.71$. In figure \ref{fig:EERACNExSituCLAvsACC_allClasses} the overall accuracy and the true positive rates are plotted against the classification rate. As can be seen the classifier was able to classify martensite crackings, but had difficulties classifying interface decohesion sites and notch effect sites. \\
\begin{figure}[H]
\centering
\includegraphics[width=\linewidth]{/SecondClassifier/EERACNExSituCLAvsACC_allClasses_corrected.pdf}
\caption{The accuracy and the different true precision rates plotted against the classification rate.}
\label{fig:EERACNExSituCLAvsACC_allClasses}
\end{figure}
The confusion matrix shows that the classifier had problems distinguishing between interface decohesions and martensite crackings, where more interface decohesion sites were classified as martensite crackings as vice versa. Furthermore problems arose distinguishing between interface decohesions and notch effects. \\

\begin{table}[H]
 \begin{center}
  \begin{tabular}{@{} *4l @{}} \toprule[2pt]
   predicted &  \multicolumn{3}{c}{true}  \\\midrule
    & MC  & ID & NE   \\ 
   MC  & 126 & 38 & 19 \\ 
   ID  & 8 & 101 & 23 \\ 
   NE  & 5 & 19 & 47 \\ \bottomrule[2pt]
  \end{tabular}
 \end{center}
 \caption{Confusion matrix.}
   \label{tab:SecondClassifierConfusionMatrixExSitu}
\end{table}

%\begin{equation}
%\begin{pmatrix}
%126 & 38 & 19 \\
%8 & 101 & 23 \\
%5 & 19 & 47 \\
%\end{pmatrix}
%\end{equation}

In figure \ref{fig:ExSituMartensiteSamples} sites classified as martensite cracking are shown, together with the class probability for the assigned class returned by the network. In figure \ref{sub:ExSituMartensiteSamplesM} a typical martensite cracking is shown correctly classified with high a high confidence. In figure \ref{sub:ExSituMartensiteSamplesI} an interface decohesion is shown. The damage site is located between two martensite grains and has an elongated shape, probably leading to the network recognizing an martensite cracking. In figure \ref{sub:ExSituMartensiteSamplesN} a notch effect site is shown. 
% the some martensite cracking sites are shown that were classified differently, were the sites were chosen that were classified with the highest confidence. In figure \ref{sub:ExSituMartensiteSamplesM} a typical martensite cracking is shown that was classified correctly, with a confidence of $1.000$. In figure \ref{sub:ExSituMartensiteSamplesI} is shown that the classifier identified with a confidence of $0.973$ as interface decohesion, this might result from the blurry nature of this particular site and the wide shape of the void. In figure \ref{sub:ExSituMartensiteSamplesN} the classifier assigned the label of notch effect with a confidence of $1.000$ ,...... 

\begin{figure}[H]
\centering
\begin{subfigure}{0.3\textwidth}
\includegraphics[width=\linewidth]{/SecondClassifier/SecondClassifierExSitu_T0P0score0_999962.png}
\caption{$1.000$}
\label{sub:ExSituMartensiteSamplesM}
\end{subfigure}
\begin{subfigure}{0.3\textwidth}
\includegraphics[width=\linewidth]{/SecondClassifier/SecondClassifierClassifier_ExSitu_P0_T1_0_95029753.png}
\caption{$0.950$}
\label{sub:ExSituMartensiteSamplesI}
\end{subfigure}
\begin{subfigure}{0.3\textwidth}
\includegraphics[width=\linewidth]{/SecondClassifier/SecondClassifierClassifier_ExSitu_P0_T2_0_9859941.png}
\caption{$0.986$}
\label{sub:ExSituMartensiteSamplesN}
\end{subfigure}
\caption{Damage sites classified as martensite cracking, together with the confidence of the network in its prediction. In (a) a martensite cracking site is shown, in (b) an interface decohesion site is shown, and in (c) a notch effect site is shown.}
\label{fig:ExSituMartensiteSamples}
\end{figure}

In figure \ref{fig:ExSituInterfaceDecohesionSamples} damage site classified as interface decohesion are shown, together with the class probability for the assigned class returned by the network. In figure \ref{sub:ExSituInterfaceDecohesionSamplesM} a martensite cracking site is shown. In figure \ref{sub:ExSituInterfaceDecohesionSamplesID} an interface decohesion site correctly classified is shown. In figure \ref{sub:ExSituInterfaceDecohesionSamplesN} a notch effect site is shown. While this site is located between two martensite grains, the initializing mechanism is interface decohesion. 

% interface decohesion sites are shown that were classified with a high confidence by the network. In figure \ref{sub:ExSituInterfaceDecohesionSamplesM} The damage sites was classified as a martensite cracking, with a confidence of $0.997$. This can be explained by the fact that the interface decoheison is located between two martensite grains. In figure \ref{sub:ExSituInterfaceDecohesionSamplesID} an interface decohesion site is shown classified correctly with a confidence of $0.988$. Finally in figure \ref{sub:ExSituInterfaceDecohesionSamplesN} a damage site is shown that was classified as a notch effect with a confidence of $0.924$. This site shows the characteristics of a notch effect as it is located between two martensite grains, therefore the classification as a notch effect is justifiable. 

\begin{figure}[H]
\centering
\begin{subfigure}{0.3\textwidth}
\includegraphics[width=\linewidth]{/SecondClassifier/SecondClassifierClassifier_ExSitu_P1_T0_0_90312707.png}
\caption{$0.903$}
\label{sub:ExSituInterfaceDecohesionSamplesM}
\end{subfigure}
\begin{subfigure}{0.3\textwidth}
\includegraphics[width=\linewidth]{/SecondClassifier/SecondClassifierExSitu_T1P1score0_987877.png}
\caption{$0.988$}
\label{sub:ExSituInterfaceDecohesionSamplesID}
\end{subfigure}
\begin{subfigure}{0.3\textwidth}
\includegraphics[width=\linewidth]{/SecondClassifier/SecondClassifierClassifier_ExSitu_P1_T2_0_82387412.png}
\caption{$0.823$}
\label{sub:ExSituInterfaceDecohesionSamplesN}
\end{subfigure}
\caption{Damage sites classified as interface decohesion, together with the confidence of the network in its prediction. In (a) the site was identified as an martensite cracking, in (b) as an interface decohesion and in (c) as a notch effect. }
\label{fig:ExSituInterfaceDecohesionSamples}
\end{figure}

In figure \ref{fig:ExSituNotchEffectSamples} damage sites classified as notch effect are shown, together with the class probability for the assigned class returned by the network. In figure \ref{sub:ExSituNotchEffectSamplesM} a martensite cracking site is shown. While the site is typical for a martensite cracking site, it is relatively small. In figure \ref{sub:ExSituNotchEffectSamplesI} the classifier identified an interface decohesion site as the underlying damage mechanism. This site is very similar to a notch effect site, as it shows itself as a small void in the viccinity of two martensite grains. Finally in figure \ref{sub:ExSituNotchEffectSamplesN} the notch effect site was classified correctly. 

\begin{figure}[H]
\centering
\begin{subfigure}{0.3\textwidth}
\includegraphics[width=\linewidth]{/SecondClassifier/SecondClassifierClassifier_ExSitu_P2_T0_0_74333996.png}
\caption{$0.743$}
\label{sub:ExSituNotchEffectSamplesM}
\end{subfigure}
\begin{subfigure}{0.3\textwidth}
\includegraphics[width=\linewidth]{/SecondClassifier/SecondClassifierClassifier_ExSitu_P2_T1_0_71155632.png}
\caption{$0.712$}
\label{sub:ExSituNotchEffectSamplesI}
\end{subfigure}
\begin{subfigure}{0.3\textwidth}
\includegraphics[width=\linewidth]{/SecondClassifier/SecondClassifierExSitu_T2P2score0_953783.png}
\caption{$0.985$}
\label{sub:ExSituNotchEffectSamplesN}
\end{subfigure}
\caption{Damage sites classified as notch effect, together with the confidence of the network in its prediction. In (a) the site was identified as an martensite cracking, in (b) as an interface decohesion and in (c) as a notch effect. }
\label{fig:ExSituNotchEffectSamples}
\end{figure}


\subsubsection{Transition to In-Situ Data}
Similarly to the generalization issues that arose for the first classifier, the accuracy of the second classifier when evaluated on the in-situ data, dropped noticeably from $0.71$ to $0.65$, while not as significantly as for the first classifier.

\begin{figure}[H]
\centering
\includegraphics[width=\textwidth]{/SecondClassifier/EERACNTransitionExToInSitu_acc.pdf}
\end{figure}

%\begin{figure}[H]
%\centering
%\includegraphics[width=\textwidth]{/SecondClassifier/EERACNInSituCLAvsACC_allClasses.pdf}
%\end{figure}

\subsubsection{Overall Performance}
The network trained on the ex-situ and in-situ training data evaluated on the ex-situ and in-situ test dataset resulted in an accuracy of $0.71$. The networks accuracy and true positive rates of each class are shown in figure \ref{fig:SecondClassifierOverallAllClasses}. From this plot it can be seen that:
\begin{itemize}
\item The networks true positive rate rate for martensite cracking is higher than the remaining classes.
\item The true positive rate for notch effects stabilized compared to the network only trained on the ex-situ dataset. 
\end{itemize}
The different metrics of the first classifier are shown in table \ref{tab:SecondClassifierMetrics}.

\begin{table}[H]
 \begin{center}
  \begin{tabular}{@{} *5c @{}} \toprule[2pt]
   Threshold & Metric &  MC & ID & NE \\ \midrule
   $\theta=0$ & ACC & \multicolumn{3}{c}{$0.767\pm0.009$} \\
   &TPR  & $0.91\pm 0.02$ & $0.63 \pm 0.05 $ & $0.72\pm 0.05$ \\
   &PPV  & $0.81\pm 0.02$ & $0.82\pm 0.03$ & $0.62\pm 0.04$ \\ \midrule
   $\theta=0.7$ & ACC & \multicolumn{3}{c}{ $0.80\pm0.01$} \\
   &CLA  & \multicolumn{3}{c}{ $0.91\pm 0.01$} \\ 
   &TPR  & $0.94\pm 0.02$ & $0.65 \pm 0.06$ & $0.76\pm 0.06$  \\
   &PPV  & $0.84 \pm 0.02$ & $0.85 \pm 0.03$ & $0.66 \pm 0.04$  \\   \bottomrule[2pt]
  \end{tabular}
 \end{center}
 \caption{The different achieved metrics for the case of classification using the classes with the highest probability and a confidence level of $\theta =0.7$. The true positive rate (TPR), predictive positive value (PPV) for each class, martensite cracking (MC), interface decohesion (ID), and notch effect (NE), and the overall accuracy (ACC) and classification rate (CLA) are shown.}
   \label{tab:SecondClassifierMetrics}
\end{table}

\begin{figure}
\centering
\includegraphics[width=\textwidth]{/SecondClassifier/EERACNInSituCLAvsACC_allClasses_corrected.pdf}
\caption{}
\label{fig:SecondClassifierOverallAllClasses}
\end{figure}

In figure \ref{fig:InSituMartensiteSamples} damage sites classified as martensite cracking are shown, together with the class probability for the assigned class returned by the network. Site \ref{sub:InSituMartensiteSamplesM} shows the classifier recognizing a martensite cracking. In figure \ref{sub:InSituMartensiteSamplesI} an interface decohesion sites was misclassified as a martensite cracking. While this site is located between two martensite grains, a ferrite bridge can be seen to the left of the void. Furthermore the void has an elongated shape. In figure \ref{sub:InSituMartensiteSamplesN} a notch effect site was mas misclassified as a martensite cracking. Similar to figure \ref{sub:InSituMartensiteSamplesI}, the void is located between two martensite islands. It can be seen that the martensite islands were not connected prior to the void nucleating.

%In figure \ref{sub:InSituMartensiteSamplesI} the damage site was classified as an interface decohesion. While it can be seen that the martensite island was cracked, a void exists between the martensite islands and the ferrite matrix. This void was probably the cause for the classifier at arriving at interface decohesion as the underlying damage mechanism. In the last site \ref{sub:InSituMartensiteSamplesN}, the damage site was classified as a notch effect. As the void is between two martensite islands, that might have been connected before the cracking, this sites shows the characteristics of a notch effect. The main difference between this site and typical notch effect damage sites is the elongated shape of the void.

\begin{figure}[H]
\centering
\begin{subfigure}{0.3\textwidth}
\includegraphics[width=\linewidth]{/SecondClassifier/SecondClassifierInSitu_T0P0score0_999788.png}
\caption{$1.000$}
\label{sub:InSituMartensiteSamplesM}
\end{subfigure}
\begin{subfigure}{0.3\textwidth}
\includegraphics[width=\linewidth]{/SecondClassifier/SecondClassifierInSitu_T1P0score0_975219.png}
\caption{$0.975$}
\label{sub:InSituMartensiteSamplesI}
\end{subfigure}
\begin{subfigure}{0.3\textwidth}
\includegraphics[width=\linewidth]{/SecondClassifier/SecondClassifierInSitu_T2P0score0_958659.png}
\caption{$0.959$}
\label{sub:InSituMartensiteSamplesN}
\end{subfigure}
\caption{Damage sites classified as martensite cracking, together with the confidence of the network in its prediction. In (a) the martensite cracking site was correctly identified. In (b) the damage site was identified as interface decohesion and in (c) as notch effect. }
\label{fig:InSituMartensiteSamples}
\end{figure}

In figure \ref{fig:InSituInterfaceDecohesionSamples} different sites classified as interface decohesion are shown, together with the class probability for the assigned class returned by the network. In figure \ref{sub:InSituInterfaceDecohesionSamplesM} a martensite cracking site can be seen. While this site might have resulted from an interface decohesion coupled with other mechanisms, a martensite cracking is most likely as the three martensite islands borders have a very similar shape. In figure \ref{sub:InSituInterfaceDecohesionSamplesID} an interface decohesion was classified correctly. In figure \ref{sub:InSituInterfaceDecohesionSamplesN} a notch effect site was classified as interface decohesion. While it is possible that the underlying mechanism was an interface decohesion, the specific local geometry is characteristic of an notch effect.
%In figure \ref{fig:InSituInterfaceDecohesionSamples} interface decohesion sites are shown together with assigned classes, where each site is the sample with the highest confidence. In figure \ref{sub:InSituInterfaceDecohesionSamplesM} the same site as shown previously in the ex-situ example was classified wrongly. In figure \ref{sub:InSituInterfaceDecohesionSamplesID}

\begin{figure}[H]
\centering
\begin{subfigure}{0.3\textwidth}
\includegraphics[width=\linewidth]{/SecondClassifier/SecondClassifierInSitu_T0P1score0_906806.png}
\caption{$0.907$}
\label{sub:InSituInterfaceDecohesionSamplesM}
\end{subfigure}
\begin{subfigure}{0.3\textwidth}
\includegraphics[width=\linewidth]{/SecondClassifier/SecondClassifierInSitu_T1P1score0_986926.png}
\caption{$0.987$}
\label{sub:InSituInterfaceDecohesionSamplesID}
\end{subfigure}
\begin{subfigure}{0.3\textwidth}
\includegraphics[width=\linewidth]{/SecondClassifier/SecondClassifierInSitu_T2P1score0_859208.png}
\caption{$0.859$}
\label{sub:InSituInterfaceDecohesionSamplesN}
\end{subfigure}
\caption{Damage sites classified as interface decohesion, together with the confidence of the network in its prediction. In (a) the site was identified as an martensite cracking, in (b) as an interface decohesion and in (c) as a notch effect. }
\label{fig:InSituInterfaceDecohesionSamples}
\end{figure}

In figure \ref{fig:InSituNotchEffectSamples} different sites classified as notch effect can be seen, together with the class probability for the assigned class returned by the network. Figure \ref{sub:InSituNotchEffectSamplesM} depicts a martensite cracking site. While this site shows similarities to a notch effect site, i.e. a small void in between two martensite islands, the underlying damage mechanism can clearly be identified as a martensite cracking. In figure \ref{sub:InSituNotchEffectSamplesI} an interface decohesion site was misclassified as a notch effect site. Due to imaging it is not clear whether the surrounding martensite grains are a single martensite grain, but the position of the void suggests that the underlying mechanism is interface decohesion. In figure \ref{sub:InSituNotchEffectSamplesN} a notch effect site was correctly identified by the classifier.

\begin{figure}[H]
\centering
\begin{subfigure}{0.3\textwidth}
\includegraphics[width=\linewidth]{/SecondClassifier/SecondClassifierInSitu_T0P2score0_657974.png}
\caption{$0.658$}
\label{sub:InSituNotchEffectSamplesM}
\end{subfigure}
\begin{subfigure}{0.3\textwidth}
\includegraphics[width=\linewidth]{/SecondClassifier/SecondClassifierInSitu_T1P2score0_961009.png}
\caption{$0.961$}
\label{sub:InSituNotchEffectSamplesI}
\end{subfigure}
\begin{subfigure}{0.3\textwidth}
\includegraphics[width=\linewidth]{/SecondClassifier/SecondClassifierInSitu_T2P2score0_993681.png}
\caption{$0.994$}
\label{sub:InSituNotchEffectSamplesN}
\end{subfigure}
\caption{Damage sites classified as notch effect, together with the confidence of the network in its prediction. In (a) the site was identified as an martensite cracking, in (b) as an interface decohesion and in (c) as a notch effect. }
\label{fig:InSituNotchEffectSamples}
\end{figure}

%\subsection{Brittle versus Ductile Damage Mechanisms}
%Furthermore a further split of the class hierarchy was considered. Instead of distinguishing between the three remaining classes, martensite cracking, interface decohesion, and notch effect, the second classifier should distinguish between brittle and ductile damage mechanisms. Therefore interface decohesions and notch effects are grouped into one class. In figure \ref{fig:2vs3Classes} the true positive rate can be seen 
%Furthermore we tested whether it is sensible to train a network capable of only distinguishing between brittle damage mechanisms (Martensite cracking) and ductile damage mechanisms (interface decohesion and notch effects) involved in the formation of voids. As can be seen in figure \ref{fig:2vs3Classes} a network trained to distinguish ductile damage mechanisms in interface decohesion and notch effects performs just as well if not better in distinguishing between brittle and ductile damage mechanisms as a network trained just for that task.
%
%\begin{figure}
%  \includegraphics[width=\linewidth]{EERACN_2vs3Classes.pdf}
%\caption{True positive rates for the EERACN network distinguishing between two classes (Martensite and rest) and between three classes (Martensite, interface decohesion, and notch)}
%\label{fig:2vs3Classes}
%\end{figure}

%\begin{figure}
%  \includegraphics[width=\linewidth]{EERACN_differentTrainingSets.pdf}
%\caption{Accuracy plotted against the classification rate for the EERACN network trained on different training sets evaluated on all test sets}
%\label{fig:TPR_comparison}
%\end{figure}

%\begin{figure}
%  \includegraphics[width=\linewidth]{EERACN_differentTrainingSets_test_in_situ.pdf}
%\caption{Accuracy plotted against the classification rate for the EERACN network trained on different training sets evaluated on the original in-situ test set}
%\label{fig:TPR_comparison}
%\end{figure}

%\begin{figure}
%  \includegraphics[width=\linewidth]{EERACN_differentTrainingSets_test_Stufe0.pdf}
%\caption{Accuracy plotted against the classification rate for the EERACN network trained on different training sets evaluated on the stage zero test set}
%\label{fig:TPR_comparison}
%\end{figure}

%\begin{figure}
%  \includegraphics[width=\linewidth]{EERACN_differentTrainingSets_test_Deformed.pdf}
%\caption{Accuracy plotted against the classification rate for the EERACN network trained on different training sets evaluated on the deformed test set}
%\label{fig:TPR_comparison}
%\end{figure}

%\begin{figure}
%  \includegraphics[width=\linewidth]{PPV_different_classes_ex_situ.pdf}
%\caption{PPV for all classes trained on ex-situ data}
%\label{fig:TPR_comparison}
%\end{figure}

%\begin{figure}
%  \includegraphics[width=\linewidth]{PPV_different_classes_in_situ.pdf}
%\caption{PPV for all classes trained on in-situ data}
%\label{fig:TPR_comparison}
%\end{figure}

%\begin{figure}
%  \includegraphics[width=\linewidth]{PPV_different_classes_stage0.pdf}
%\caption{PPV for all classes trained on stage zero data}
%\label{fig:TPR_comparison}
%\end{figure}

%\begin{figure}
%  \includegraphics[width=\linewidth]{PPV_different_classes_deformed.pdf}
%\caption{PPV for all classes trained on deformed data }
%\label{fig:TPR_comparison}
%\end{figure}

%\subsection{Ex-Situ Data}
%\begin{itemize}
%\item Accuracy
%\item Confusion Matrix
%\item Precision vs Recall
%\item Problems
%\item 2 vs 3 classes
%\end{itemize}

%\subsection{In-Situ Data}
%\begin{itemize}
%\item Accuracy
%\item Confusion Matrix
%\item Precision vs Recall
%\item Problems
%\item 2 vs 3 classes
%\end{itemize}

\section{Combined Classifier}
%
%\subsection{Combined Accuracy}

\subsection{Classification of Shadows}
\label{sec:Robustness}
As explained in section \ref{sec:Shadows}, the localizer can find sites not resulting from damage mechanisms but from the topography of the surface. In the following the predictions of the combined classifier using a threshold of $0.7$ for both CNNs of sites showing shadows will be studied. Out of $55$ sites showing shadows, $2$ were classified as inclusion sites, shown in figure \ref{fig:shadowAsInc}, $6$ as martensite cracking, with a selection shown in figure \ref{fig:shadowAsMC}, $19$ as interface decohesions with a selection shown in figure \ref{fig:shadowAsID}, none as notch effects, and the rest were not classified.

\begin{figure}[H]
\centering
\begin{subfigure}{0.24\textwidth}
\includegraphics[width=0.8\linewidth]{/Shadows/Nothing_97_as_Inc.png}
\caption{}
\end{subfigure}
\centering
\begin{subfigure}{0.24\textwidth}
\includegraphics[width=0.8\linewidth]{/Shadows/Nothing_246_as_Inc.png}
\caption{}
\end{subfigure}
\caption{Shadows classified as inclusions.}
\label{fig:shadowAsInc}
\end{figure}

\begin{figure}[H]
\centering
\begin{subfigure}{0.24\textwidth}
\includegraphics[width=0.8\linewidth]{/Shadows/Shadow_6_As_0.png}
\caption{}
\end{subfigure}
\centering
\begin{subfigure}{0.24\textwidth}
\includegraphics[width=0.8\linewidth]{/Shadows/Shadow_8_As_0.png}
\caption{}
\end{subfigure}
\centering
\begin{subfigure}{0.24\textwidth}
\includegraphics[width=0.8\linewidth]{/Shadows/Shadow_17_As_0.png}
\caption{}
\end{subfigure}
\centering
\begin{subfigure}{0.24\textwidth}
\includegraphics[width=0.8\linewidth]{/Shadows/Shadow_19_As_0.png}
\caption{}
\end{subfigure}
\caption{Shadows classified as martensite cracking.}
\label{fig:shadowAsMC}
\end{figure}

\begin{figure}[H]
\centering
\begin{subfigure}{0.24\textwidth}
\includegraphics[width=0.8\linewidth]{/Shadows/Shadow_3_As_1.png}
\caption{}
\end{subfigure}
\centering
\begin{subfigure}{0.24\textwidth}
\includegraphics[width=0.8\linewidth]{/Shadows/Shadow_7_As_1.png}
\caption{}
\end{subfigure}
\centering
\begin{subfigure}{0.24\textwidth}
\includegraphics[width=0.8\linewidth]{/Shadows/Shadow_10_As_1.png}
\caption{}
\end{subfigure}
\centering
\begin{subfigure}{0.24\textwidth}
\includegraphics[width=0.8\linewidth]{/Shadows/Shadow_13_As_1.png}
\caption{}
\end{subfigure}
\caption{Shadows classified as interface decohesion.}
\label{fig:shadowAsID}
\end{figure}


\subsection{In-Situ Tracking}
In order to track the evolution of damage sites, both classifiers were trained on all available data with the training parameters as described previously. \\

The damage sites were tracked from one panorama to the next using a rather primitive algorithm. At first the expected position of a damage site was calculated by using the coordinates of the damage sites and correcting it by the elongation of the panorama. By now extracting a window around the damage site and a search window around the expected position in previous and/or following panoramas of size ... . In order to reduce the required computational resources both the window containing the damage site and the search window were reduced in size using max-pooling with a window size of $2\times 2$ and a stride of $2$. Then the window containing the damage site was moved across the search window calculating the difference between pixel values, and the position was determined by the position at which the deviation was minimal. In the following some selected damage sites and their evolution will be shown. \\

In figures \ref{fig:MCEV1} and \ref{fig:MCEV2} the evolution of two martensite cracking sides are shown. In figures \ref{fig:NEEV1} and \ref{fig:NEEV2} the evolution of two notch effect sides are shown. Two effects can be seen, firstly one can see that the voids increase in size and secondly that the surface deepens around the void. \\

\begin{figure}
\includegraphics[width=\textwidth]{/InSituTracking/Martensite.png}
\caption{The evolution of a martensite cracking found in stage 0.}
\label{fig:MCEV1}
\end{figure}

\begin{figure}
\includegraphics[width=\textwidth]{/InSituTracking/Martensite2.png}
\caption{The evolution of a martensite cracking found in stage 0.}
\label{fig:MCEV2}
\end{figure}


%\begin{figure}
%\includegraphics[width=\textwidth]{/InSituTracking/Martensite4.png}
%\caption{The evolution of a martensite cracking found in stage 0.}
%\label{fig:MCEV3}
%\end{figure}

\begin{figure}
\includegraphics[width=\textwidth]{/InSituTracking/Notch.png}
\caption{The evolution of a martensite cracking found in stage 0.}
\label{fig:NEEV1}
\end{figure}

\begin{figure}
\includegraphics[width=\textwidth]{/InSituTracking/Notch2.png}
\caption{The evolution of a martensite cracking found in stage 0.}
\label{fig:NEEV2}
\end{figure}



For early damage sites in early stages of deformation the classification algorithm found $15$ inclusions of which $13$ where classified correctly, corresponding to an accuracy of about $0.86$ for inclusion sites. Of the $54$ found martensite cracking damage sites $42$ were classified correctly, an accuracy of about $0.78$, of the $16$ interface decohesion damage sites $12$ were classified correctly, corresponding to an accuracy of about $0.75$, of the $8$ notch effect damage sites $6$ were classified correctly, corresponding to an accuracy of about $0.75$, and $32$ sites were not classified. While the set of samples is barely statistically relevant, the trend described beforehand was reproduced. \\

For higher stages of deformation the classifier rarely classified damage sites correctly. Furthermore the localization algorithm performed very poorly finding many shadows. This effect can be explained by the formation of surface reliefs, and the evolved nature of many damage sites in later evolutions, as newly nucleated damage sites rarely occur. 
While the classification of damage sites in early stages of deformation worked reasonably well, in later stages the misclassification rate increased drastically. This can partly be explained due to the formation of surface reliefs at higher stages of deformation. Another possibility is that 

\subsection{Generalization}

Applying the combined classifier to a dual-phase steel, that is different from the one the classifier has been trained on, the accuracy drops significantly. In figure \ref{fig:newMaterial} a different dual-phase steel microstructure is shown. As can be seen the martensite islands have a very different appearance. Standing out is the the fine surface structure of martensite and the bright border. The performance of the classifier drops significantly, i.e. $127$ inclusion sites were found of which $33$ were identified correctly corresponding to an accuracy of $0.26$, of $60$ found martensite cracking sites $15$ were correctly identified corresponding to an accuracy of $0.25$, of $137$ interface decohesion sites $80$ were identified correctly corresponding to an accuracy of $0.58$, and of $24$ notch effect sites $16$ were identified correctly corresponding to an accuracy of $0.67$. \\

\begin{figure}
\centering
\begin{subfigure}{0.48\textwidth}
\includegraphics[width=\linewidth]{NewMaterial/Interface10.png}
\end{subfigure}
\centering
\begin{subfigure}{0.48\textwidth}
\includegraphics[width=\linewidth]{NewMaterial/Interface37.png}
\end{subfigure}
\caption{Surface micrograph of a different dual-phase steel.}
\label{fig:newMaterial}
\end{figure}


%
%%Due to the first network acting as a filter for inclusions, it is necessary to minimize the number of damage sites falsely classified as inclusions or equivalently minimizing the number of false negatives. The relevant quantity for inspecting the performance under these conditions of a binary classifier, in this case the neural network, is its precision (positive predictive value) defined by
%%\begin{equation}
%%PPV = \frac{TP}{TP+FP}
%%\end{equation}
%%where $TP$ is the number of correctly classified inclusions and $FP$ is the number of other damage sites classified as inclusions. \\
%%On the other hand for the network to be useful, the number of correctly classified inclusions should be maximized, in order to reduce the amount of work necessary to relabel remaining damage sites by hand. 
%
%%\subsection{Ex-situ to in-situ}
%%While the first classifier, responsible for filtering out the inclusion sites, performed well on the ex-situ data sets, problems arose while trying to use it for the classification of in-situ damage sites. By using some of the in-situ data for the training of the network, its performance was substantially increased. The data sets for training and testing of the classifier are shown in the following table. \\
%%
%%\begin{tabular}{| l | c | c | c | c |}
%%\hline
%% & ex-situ train & ex-situ test & in-situ train & in-situ test \\ \hline
%%excluding in-situ & training & training & ignored & testing \\ \hline
%%including in-situ & training & training & training & testing \\ \hline
%%only ex-situ data & training & testing & ignored & ignored \\ \hline
%%\end{tabular}
%%
%\subsubsection{Training excluding in-situ data}
%Without including the in-situ data in the training set of the network, it characterized $52$ out of the $62$ inclusion sites correctly, while also classifying $164$ out $409$ sites that aren't inclusions as inclusions. Since the purpose of the first network is to filter out inclusions, this performance would make it inapplicable. While some of the inclusion sites not labeled as inclusions, can pass through the system and have to be labeled afterwards by hand, labeling sites that aren't inclusions with a high confidence as inclusions poses a huge problem for its usage as a filtering system. 

%\subsubsection{Training including in-situ data}
%By including some of the in-situ data into the training set, $44$ out of the $62$ inclusions sites were classified correctly as inclusions, performing slightly worse than the network trained on only the ex-situ data set. However none of the remaining $409$ sites were classified as inclusions.

%\subsubsection{Training including all data}
%Due to new data being created, one more test was included. The performance of the network trained on the final dataset is shown in figure \ref{fig:FirstClassifierFinal}. As can be seen the performance of the network 













%\chapter{Workflow}

Given a desired accuracy the internal thresholds of the CNNs used in the classifier have to be adjusted. This results in the classifier not being able to classify all damage sites in a given panorama. Nevertheless even for high thresholds this algorithm is capable to reduce the required time for the classification of damage sites. In the following a workflow is proposed.


%\include{Chapters/Conclusion}
%\include{Chapters/Outlook}
%\appendix
%% Chapter 2

\chapter{Appendix} % Main chapter title

\label{Appendix} % For referencing the chapter elsewhere, use \ref{Chapter1} 

\section{Stress-Strain Curve of Dual-Phase Steel 800}

\begin{figure}[H]
\centering
  \includegraphics[width=\linewidth]{StressStrain.pdf}
  \caption{Dual-phase steel SEM micrograph}
  \label{fig:DPStressStrain}
\end{figure}

\section{Experimental Details}

\textbf{Preparation} \\
Before the experiments were performed, the surface of the dual-phase steel was prepared in order to be able to distinguish the martensite islands from the ferrite matrix, in surface micrographs. The preparation process consists of grinding the surface with up to $4000$ grit sandpaper, polishing it with oxide polishing suspension in steps of $6\mu m$, $3\mu m$, and $1\mu m$, and finally etching it with $1 \%$ nital. \\

\noindent \textbf{Probe Geometry} \\
The probe geometry was chosen in such a way that the center of the probe experiences almost homogeneous stress. It can be seen in figure \ref{fig:ProbeGeometry}. \\

\noindent \textbf{SEM Specifications} \\
The surface of the material after deformation was recorded using an SEM. Its specifications and settings can be seen in table \ref{tab:SEM}. \\

\begin{table}[H]
 \begin{center}
  \begin{tabular}{@{} *2l @{}} \toprule[2pt]
   Model & Zeiss LEO 1530 \\\midrule
   Horizontal Field Width & $100\mu m$   \\ 
   Vertical Field Width  & $75\mu m$ \\ 
   Resolution  & $3072\times 2304$ \\
   Detector Type & Secondary Electrons \\
   Electron Source & Field-Emitter Cathode \\ \bottomrule[2pt]

  \end{tabular}
 \end{center}
 \caption{Details of the SEM used for the surface micrographs.}
   \label{tab:SEM}
\end{table}

\section{Wrong Label}

In a first approximation the effect of using different class definitions for labeling or misclassification can be explained as follows. Assuming the network has already been trained to some extent on a correctly labeled dataset, two similar samples, both belonging to class $c_0$ but one of them incorrectly labeled as class $c_1$, will be mapped by the feature extraction to close proximity of each other in the feature space. The fully connected layer in the end of the CNN will then have to learn to map these close points in the feature space to the maximal distance in the output space, requiring large weight corrections, therefore slowing down training. Furthermore, if the two points in the feature space are arbitrarily small to each other the effect of a mislabeled training sample will then nullify the adaptation of the network to a correctly labeled sample. This mapping problem can be seen in figure \ref{fig:featuremapping}.

\begin{figure}[H]
\centering
\includegraphics[width=\textwidth]{featuremapping.pdf}
\caption{The classification of two similar images in the input space to two different classes. Given a feature mapping this will on the one hand lead to a close distance in the feature space, while on the other hand to a desired maximal distance in the output space. The possible outputs are indicated by a dashed line.}
\label{fig:featuremapping}
\end{figure}
%
%\section{Cost Functions}
%\begin{itemize}
%\item How a human learns - punishment?
%\item Define what is right and what is wrong
%\item Mathematical foundation to find optimal weights
%\item Properties of cost functions:
%\subitem $C>0$
%\subitem Output of network close to desired output then cost function close to minimum
%\end{itemize}
%\subsection{Squared Error}
%\begin{itemize}
%\item Linear regression
%\item Classical error function
%\end{itemize}
%\subsection{Categorical Cross Entropy}
%% http://neuralnetworksanddeeplearning.com/chap3.html#the_cross-entropy_cost_function
%\begin{itemize}
%\item Problems with squared error: small gradient
%\item Definition of cross entropy
%\item Using definition of sigmoid function shows that the system learns faster the further it is away from the true solution
%\end{itemize}
%
%\section{Activation Functions}
%Some interesting facts check where to put them later
%\begin{itemize}
%\item gap between computational neuroscience models and machine learining models
%\subitem brain: neurons encode information in a sparse and distributed way check http://journals.sagepub.com/doi/pdf/10.1097/00004647-200110000-00001
%\subitem non-linear activation functions: leaky integrate and fire, sigmoid and tanh similar
%\subitem firing rate of sigmoid about $1/2$ 
%\subitem firing rate of tanh about $0$ but antisymmetry absent in biological neurons
%\end{itemize}
%\subsection{Binary Activation Function}
%\begin{itemize}
%\item Inspired from biological neurons
%\item Historically first used
%\item Rosenblatt perceptron
%\end{itemize}
%\subsection{Sigmoid}
%\begin{equation}
%\phi(v) = \frac{1}{1+\exp(-\alpha v)}
%\end{equation}
%\begin{figure}
%%sigmoid plot
%\end{figure}
%\begin{itemize}
%\item Small changes in input should result in small changes in the output
%\item Continuous activation functions
%\item Problems:
%\subitem Saturation for small and large input (gradient very small)
%\subitem Always positive (use functions like arctangent)
%\end{itemize}
%\subsection{Hyperbolic Tangent}
%\begin{equation}
%\phi(v) = \tanh(\alpha x)
%\end{equation}
%\begin{itemize}
%\item Biologically less plausible than sigmoid activation function
%\item Positive and negative
%\item Better suited for the training of multilayer neural networks
%\end{itemize}
%
%\subsection{ReLu}
%\begin{equation}
%\phi(v) = \max(0,x)
%\end{equation}
%\begin{itemize}
%\item Useful paper \cite{DeepSparceRectifierNN}
%\item Inspired by sparsity of activation
%\item Rectifying non-linearity gives rise to real zeros
%\end{itemize}
%
%\subsection{Advanced ReLu}
%\begin{itemize}
%\item PReLu with trainable parameter $a_i$
%\item Leaky ReLu ...
%\end{itemize}
%
%\subsection{Softmax}
%\begin{equation}
%\phi_j^L = \frac{e^{z_j^L}}{\sum_k e^{z_k^L}}
%\end{equation}
%\begin{itemize}
%\item Sumation over output neurons equals $1$
%\item All activations sum up to 1
%\item Can be interpreted as probability distribution
%\end{itemize}
%
%\section{Comparison}
%Paper:
%\begin{itemize}
%\item Empirical Evaluation of Rectified Activations in Convolution Network
%\end{itemize}
%
%
%\subsection{Gradient Descent}
%\begin{equation}
%\Delta w_{ji}(n) = -\eta \frac{\partial E(n)}{\partial w_{ji}(n)}
%\end{equation}
%\begin{itemize}
%\item First order optimization technique
%\item Calculate local gradient of loss hyper surface
%\item Follow path of steepest descent
%\item Adjustable parameter: Learning rate $\eta$
%\end{itemize}
%\begin{figure}
%\caption{Show trajectory of gradient descent with different learning rates}
%\end{figure}
%
%\subsection{Momentum Gradient Descent}
%\begin{itemize}
%\item Gradient descent: taking step always in direction of instantaneous steepest descent
%\item Momentum gradient descent: add momentum (compare to ball rolling down hill
%\item Inertia: smoother, accelerator, dampening oscillations, helps navigate past local minima (Hier aufpassen noch sehr nah an https://distill.pub/2017/momentum/)
%\item Adding momentum corresponds to giving short term memory
%\end{itemize}
%\begin{figure}
%\caption{Show trajectory of gradient descent with momentum with different parameters}
%\end{figure}
%
%\subsection{Stochastic Gradient Descent}
%Paper:
%\begin{itemize}
%\item Efficient Mini-batch Training for Stochastic Optimization
%\end{itemize}
%\begin{itemize}
%\item Instead of using the informationi of the entire training data use only single elements or batch of elements (batch stochastic gradient descent) from the training data
%\end{itemize}
%
%\subsection{Adam}
%\begin{itemize}
%\item Only uses first order gradients
%\item Computes individual adaptive learning rates for different parameters from estimates of first and second moments of gradients
%\item Adaptive moment estimation $\rightarrow$ Adam
%\end{itemize}
%
%\subsection{L-BFGS}
%
%
%\section{Weight Initialization}
%\begin{itemize}
%\item How to initialize Network properly in order to quickly find the optimal configuration
%\item Naive: Set everything to zero $\rightarrow$ bad, have to introduce symmetry breaking
%\item `` Biases can generally be initialized to zero but weights need to be initialized carefully to break the symmetry between hidden units of the same layer. Because different output units receive different gradient signals, this symmetry breaking issue does not concern the output weights (into the output units), which can therefore also be set to zero.''
%\item Standard gradient descent performing poorly on deep neural networks with random initialization
%\end{itemize}
%
%\subsection{Xavier Initialization}
%Paper:
%\begin{itemize}
%\item Understanding the difficulty of training deep feedforward neural networks
%\end{itemize}
%\begin{itemize}
%\item Introduced to train deep neural networks with sigmoids, tanh or softsign
%\item Ensure error signal reaches all layers
%\end{itemize}
%
%\subsection{He Initialization}
%Paper:
%\begin{itemize}
%\item On weight initialization in deep neural networks
%\item Delving Deep into Rectifiers: Surpassing Human-Level Performance on ImageNet Classification
%\end{itemize}
%\begin{itemize}
%\item https://arxiv.org/pdf/1502.01852.pdf
%\item Introduced in order to improve the performance of neural networks with rectified linear units
%\end{itemize}
%
%\subsection{Comparison}
%\begin{itemize}
%\item Show training of deep network with the two different initialization techniques
%\end{itemize}
%
%
%
%\section{Regularization}
%\begin{itemize}
%\item Training set not representative of problem
%\item Overfitting a problem, detects features not relevant for the classification task
%\item How to train the network such that it generalizes well
%\item Generalization of network: performance on data not part of training data
%\end{itemize}
%
%\subsection{Early Stopping}
%\begin{itemize}
%\item Split data into training, validation and test
%\item Performance of network on validation set indicator for how well the network generalizes
%\item Once the performance on the validation set stagnates or decreases stop training, from this point irrelevant features will be detected
%\end{itemize}
%\begin{figure}
%\caption{Show plot of accuracy/loss on training and validation set over epochs}
%\end{figure}
%
%\subsection{Regularization Theory}
%
%\subsection{Dropout}
%Paper:
%\begin{itemize}
%\item Improving neural networks by preventing co-adaptation of feature detectors
%\item Dropout as data augmentation
%\item Dropout: A Simple Way to Prevent Neural Networks from Overfitting
%\end{itemize}
%A different approach, Dropout, to regularization was first introduced by Srivastava et. al. \cite{DropoutOriginal}. Instead of introducing an additional term to the error function keeping the weights between neurons \"small\" or stopping the training early, once the accuracy on the validation set stagnates, neurons are chosen by random and \"droped out\", hence the name. This can be seen in \ref{DropoutDiagram} \\
%Dropping out units from a neural network corresponds to creating a new neural network that shares the existing weights from the original network. After units have been dropped out the network is trained on the training data. This process is repeated and neurons are selected again by random. For a neural network consisting of $n$ neurons there are $2^n$ such possible thinned networks, of which each realization will rarely if at all be trained. The trained weigths will then be averaged resulting in a network of the original size.
%
%\begin{figure}
%	\centering
%	\includegraphics[width=6cm]{dropout.jpeg}
%	 \caption{Dropout Neural Net Model.}
%	 %taken from:http://blog.christianperone.com/2015/08/convolutional-neural-networks-and-feature-extraction-with-python/
%	 %ueberlegen andere grafik zu benutzen
% \label{DropoutDiagram}
%\end{figure}
%
%In a standard neural network using backpropagation the error function is minimized by using the influence of each parameter, leading to neurons adapting to one another and possibly compensating errors made by those neurons. This co-adaption leads to overfitting since it does not generalize to to unseen data. By randomly dropping out units this effect is suppressed. In their original paper Srivastava et. al. showed this by looking at the first level features of a neural network trained on the MNIST \cite{MNIST} with and without dropout. As can be seen in \ref{Dropout_coadption} features learned without dropout co-adapt to one another while the features learned with dropout are seperated more clearly.
%
%\begin{figure}
%	\centering
%	\includegraphics[width=\textwidth]{dropout-coadaption.png}
%	 \caption{Features learned on the MNIST data set with one hidden layer autoencoders having 256 rectified linear units. \cite{DropoutOriginal}}
%	 %taken from:http://blog.christianperone.com/2015/08/convolutional-neural-networks-and-feature-extraction-with-python/
%	 %ueberlegen andere grafik zu benutzen
% \label{Dropout_coadption}
%\end{figure}
%
%\subsection{Dropconnect}
%Paper:
%\begin{itemize}
%\item Regularization of neural networks using dropconnect
%\end{itemize}
%\begin{itemize}
%\item Further development of Dropout
%\item Instead of dropping neurons drop connections
%\item Slight improvement over dropout
%\end{itemize}
%
%\subsection{Batch Normalization}
%Paper:
%\begin{itemize}
%\item Batch Normalization: Accelerating Deep Network Training
%\end{itemize}
%
%
%\section{Optimization of Hyperparameters}
%Paper:
%\begin{itemize}
%\item Practical recommendations for gradient-based training of deep architectures
%\end{itemize}
%\begin{itemize}
%\item Two kinds of parameters:
%\subitem Parameters intrinsic to the model (model selection)
%\subitem Hyperparameters used for the training of the model
%\item Grid search of hyperparameters
%\subitem Logarithmic search
%\subitem Random search (curse of dimensionality)
%\end{itemize}

%\bibliography{Citations,ImageNetCitation}{}
%\bibliographystyle{plain}
\end{document}